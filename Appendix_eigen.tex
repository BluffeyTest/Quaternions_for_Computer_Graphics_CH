
\appendix
%\renewcommand{\thetable}{A.\arabic{table}} %在表格序号前面加上章节号,如:A-1,B-1 
%\renewcommand{\theequation}{A.\arabic{equation}} %在公式序号前加上章节号,

% 修改附录的章节显示格式
\titleformat{\chapter}{
    \Huge\bfseries}{\begin{CJK}{UTF8}{gkai}
        %第\,\thechapter\,章
        \hsp\textcolor{gray75}{|}\hsp\end{CJK}}{0pt}{\Huge\bfseries}

%更改附录的目录显示
\titlecontents{chapter}[0pt]{\addvspace{1.5pt}\filright\bf}%
        {}%
        {}{\titlerule*[8pt]{.}\contentspage}

% 修改附录章节的页眉页脚格式
\renewcommand{\chaptermark}[1]{\markboth{\begin{CJK}{UTF8}{gkai}#1\end{CJK}}{}}
\renewcommand{\sectionmark}[1]{\markright{\thesection\, #1}}

\begin{CJK}{UTF8}{gkai}
    % \pdfbookmark{附录:特征值和特征向量}{eigenvector}
    \chapter{附录:特征值和特征向量}
\end{CJK}

这个附录展示了如何计算旋转矩阵的特征向量。

给定一个方阵$\mathbf{A}$,非零向量$\mathbf{v}$是一个特征向量,$\lambda$是对应的特征值如果
$$
\mathbf{A v}=\lambda \mathbf{v}, \quad \lambda \in \mathbb{R} .
$$

德语单词eigen的意思是特征的,拥有的,潜在的或特殊的,而eigenvector的意思是一个与变换相关的特殊向量。决定任何特征向量存在的方程称为方阵的特征方程,由
\begin{align}
    \operatorname{det}(\mathbf{A}-\lambda \mathbf{I})=0 .
\end{align}
为了说明我们如何求解特征方程(A.1),我们从三个联立方程开始:
$$
\begin{aligned}
3 x+z & =0 \\
-x+3 y+3 z & =0 \\
x+3 z & =0
\end{aligned}
$$
可以用矩阵形式表示为
$$
\left[\begin{array}{ccc}
3 & 0 & 1 \\
-1 & 3 & 3 \\
1 & 0 & 3
\end{array}\right]\left[\begin{array}{l}
x \\
y \\
z
\end{array}\right]=\left[\begin{array}{l}
0 \\
0 \\
0
\end{array}\right]
$$
若
$$
\mathbf{A}=\left[\begin{array}{ccc}
3 & 0 & 1 \\
-1 & 3 & 3 \\
1 & 0 & 3
\end{array}\right]
$$
其特征方程为
$$
\left|\begin{array}{ccc}
3-\lambda & 0 & 1 \\
-1 & 3-\lambda & 3 \\
1 & 0 & 3-\lambda
\end{array}\right|=0
$$
用第一行展开行列式
$$
\begin{aligned}
(3-\lambda)\left|\begin{array}{cc}
3-\lambda & 3 \\
0 & 3-\lambda
\end{array}\right|-0+\left|\begin{array}{cc}
-1 & 3-\lambda \\
1 & 0
\end{array}\right| & =0 \\
(3-\lambda)(3-\lambda)^{2}-(3-\lambda) & =0 \\
(3-\lambda)\left[(3-\lambda)^{2}-1\right] & =0 \\
(3-\lambda)\left(\lambda^{2}-6 \lambda+8\right) & =0 \\
(3-\lambda)(\lambda-4)(\lambda-2) & =0
\end{aligned}
$$
它有解$\lambda=2,3,4$。让我们把这些$\lambda$的值代入原始方程来揭示特征向量。
$$
\left\{
\begin{aligned}
    (3-\lambda)x& &+  &&z         &=0\\
    -x & +(3-\lambda)y& +&&3z    &=0\\
    x& &+       &&(3-\lambda)z          &=0
\end{aligned}
\right.
$$

对于$\lambda=2$,我们从第一个方程得到$z=-x$。将它代入第二个方程,我们有$y= 4x $,这允许我们声明相关的特征向量是这样的形式$\begin{bmatrix} k & 4 k & -k\end{bmatrix}^{\mathrm{T}}$。

对于$\lambda=3$,我们从第一个方程得到$z=0$,从第三个方程得到$x=0$,这允许我们声明相关的特征向量的形式为$\begin{bmatrix}0 & k & 0\end{bmatrix}^{\mathrm{T}}$。

对于$\lambda=4$,我们从第一个方程得到$z=x$。将它代入第二个方程,我们有$y=2 x$,这允许我们声明相关的特征向量是这样的形式$\begin{bmatrix}k & 2 k & k\end{bmatrix}^{\mathrm{T}}$。

因此,特征向量和特征值是
$$
\begin{array}{rrr}
{\left[\begin{array}{rrr}
k & 4 k & -k
\end{array}\right]^{\mathrm{T}}} & \lambda=2 \\
{\left[\begin{array}{lll}
0 & k & 0
\end{array}\right]^{\mathrm{T}}} & \lambda=3 \\
{\left[\begin{array}{lll}
k & 2 k & k
\end{array}\right]^{\mathrm{T}}} & \lambda=4
\end{array}
$$
其中 $k \neq 0$.

这种技术的主要问题是,它需要仔细分析来解开特征向量,理想情况下,我们需要一个确定性算法来揭示结果,这将在下文中讨论。

我们从这样一个事实开始,旋转矩阵总是有一个实特征值$\lambda=1$,这允许我们写出
$$
\begin{aligned}
\mathbf{A} \mathbf{v} & =\lambda \mathbf{v} \\
\mathbf{A v} & =\lambda \mathbf{I} \mathbf{v}=\mathbf{I} \mathbf{v} \\
(\mathbf{A}-\mathbf{I}) \mathbf{v} & =\mathbf{0}
\end{aligned}
$$
因此,
\begin{align}
\left[\begin{array}{ccc}
\left(a_{11}-1\right) & a_{12} & a_{13} \\
a_{21} & \left(a_{22}-1\right) & a_{23} \\
a_{31} & a_{32} & \left(a_{33}-1\right)
\end{array}\right]\left[\begin{array}{l}
x_{v} \\
y_{v} \\
z_{v}
\end{array}\right]=\left[\begin{array}{l}
0 \\
0 \\
0
\end{array}\right]
\end{align}
展开 (A.2) 我们得到
$$
\begin{aligned}
& \left(a_{11}-1\right) x_{v}+a_{12} y_{v}+a_{13} z_{v}=0 \\
& a_{21} x_{v}+\left(a_{22}-1\right) y_{v}+a_{23} z_{v}=0 \\
& a_{31} x_{v}+a_{32} y_{v}+\left(a_{33}-1\right) z_{v}=0 .
\end{aligned}
$$
存在$x_{v}=y_{v}=z_{v}=0$的平凡解,但为了发现更有用的东西,我们可以松弛$v$中的任何一项,它给出了两个未知数中的三个方程。令$x_{v}=0$:
\begin{align}
a_{12} y_{v}+a_{13} z_{v} & =-\left(a_{11}-1\right) \\
\left(a_{22}-1\right) y_{v}+a_{23} z_{v} & =-a_{21} \\
a_{32} y_{v}+\left(a_{33}-1\right) z_{v} & =-a_{31} .
\end{align}
现在我们面临着选择一对方程来分离$y_{v}$和$z_{v}$。事实上,我们必须考虑所有三种配对,因为未来的旋转矩阵可能包含一个有两个零元素的列,这可能与我们在此阶段所做的任何配对冲突。

让我们从选择(A.3)和(A.4)开始。求解采用以下策略:给定以下矩阵方程
$$
\left[\begin{array}{ll}
a_{1} & b_{1} \\
a_{2} & b_{2}
\end{array}\right]\left[\begin{array}{l}
x \\
y
\end{array}\right]=\left[\begin{array}{l}
c_{1} \\
c_{2}
\end{array}\right]
$$
然后
$$
\frac{x}{\left|\begin{array}{ll}
c_{1} & b_{1} \\
c_{2} & b_{2}
\end{array}\right|}=\frac{y}{\left|\begin{array}{ll}
a_{1} & c_{1} \\
a_{2} & c_{2}
\end{array}\right|}=\frac{1}{\left|\begin{array}{ll}
a_{1} & b_{1} \\
a_{2} & b_{2}
\end{array}\right|} .
$$
因此,利用第1和第2方程(A.3)和(A.4),我们有
$$
\begin{aligned}
    \frac{x}{\begin{vmatrix}
        -(a_{11}-1) & a_{13} \\
        -a_{21} & a_{23}
        \end{vmatrix}}&=\frac{y}{\begin{vmatrix}
        a_{12} & -(a_{11}-1) \\
        (a_{22}-1) & -a_{21}
        \end{vmatrix}}=\frac{1}{\begin{vmatrix}
        a_{12} & a_{13} \\
        (a_{22}-1) & a_{23}
        \end{vmatrix}} \\
 x_{v}&=a_{12} a_{23}-a_{13}\left(a_{22}-1\right) \\
 y_{v}&=a_{13} a_{21}-a_{23}\left(a_{11}-1\right) \\
 z_{v}&=\left(a_{11}-1\right)\left(a_{22}-1\right)-a_{12} a_{21} .
\end{aligned}
$$
类似地,使用第1和第3个方程(A.3)和(A.5),我们有
$$
\begin{aligned}
& x_{v}=a_{12}\left(a_{33}-1\right)-a_{13} a_{32} \\
& y_{v}=a_{13} a_{31}-\left(a_{11}-1\right)\left(a_{33}-1\right) \\
& z_{v}=a_{32}\left(a_{11}-1\right)-a_{12} a_{31}
\end{aligned}
$$
利用第2和第3个方程(A.4)和(A.5),我们得到
$$
\begin{aligned}
& x_{v}=\left(a_{22}-1\right)\left(a_{33}-1\right)-a_{23} a_{32} \\
& y_{v}=a_{23} a_{31}-a_{21}\left(a_{33}-1\right) \\
& z_{v}=a_{21} a_{32}-a_{31}\left(a_{22}-1\right) .
\end{aligned}
$$
现在我们有9个方程来应对任何可能的情况。事实上,没有什么能阻止我们选择任何三个我们喜欢的,例如这三个方程看起来很有趣和合理:
\begin{align}
& x_{v}=\left(a_{22}-1\right)\left(a_{33}-1\right)-a_{23} a_{32} \\
& y_{v}=\left(a_{33}-1\right)\left(a_{11}-1\right)-a_{31} a_{13} \\
& z_{v}=\left(a_{11}-1\right)\left(a_{22}-1\right)-a_{12} a_{21} .
\end{align}
因此,特征向量的解为 $\left[\begin{array}{lll}x_{v} & y_{v} & z_{v}\end{array}\right]^{\mathrm{T}}$。注意$y_{v}$的符号被颠倒以保持对称性。

