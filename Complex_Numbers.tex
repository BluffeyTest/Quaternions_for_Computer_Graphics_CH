\chap{复数}
\section{介绍}
在本章中,我们将发现没有实根的方程如何产生虚数$i$,其平方为$-1$。这反过来又把我们引向复数以及它们是如何被代数处理的。与四元数相关的许多性质都是在复数中发现的,这就是为什么它们值得仔细研究的原因。

\section{虚数}
虚数的发明是为了解决方程没有实根的问题,比如$x^{2}+16=0$。声明一个量$i$的存在,使得$i^{2}=-1$,允许我们将这个方程的解表示为
$$
x=\pm 4 i .
$$
试图发现$i$到底是什么是毫无意义的-我真的只是一个平方到$-1$的东西。尽管如此,它确实适合于图形化的解释,我们将在下一章进行研究。

1637年,法国数学家René Descartes(1596-1650)发表了La Géométrie[11],他在其中指出包含$\sqrt{-1}$的数字是“虚数”,几个世纪以来这个标签一直存在。不幸的是,这是一个贬义的评论,$i$并不是虚构的——它实际上只是一个平方到$-1$的东西,当嵌入代数时,会产生一些惊人的模型。

虚数集由$\mathbb{I}$表示,它允许我们将虚数定义为
$$
i b \in \mathbb{I}, \quad b \in \mathbb{R}, \quad i^{2}=-1 .
$$

\section{$i$的幂}
当$i^{2}=-1$时,应该可以将$i$提高到其他幂。例如,
$$
i^{4}=i^{2} i^{2}=1
$$
和
$$
i^{5}=i i^{4}=i .
$$
因此,我们有这样的序列:
\begin{center}
\begin{tabular}{lllllll}
\hline
$i^{0}$ & $i^{1}$ & $i^{2}$ & $i^{3}$ & $i^{4}$ & $i^{5}$ & $i^{6}$ \\
\hline
1 & $i$ & $-1$ & $-i$ & 1 & $i$ & $-1$ \\
\hline
\end{tabular}
\end{center}
循环模式$(1,i,-1,-i, 1, \ldots)$非常引人注目,并让我们想起类似的模式$(x, y,-x,-y, x, \ldots)$,它是通过逆时针方向绕笛卡尔轴旋转而产生的。这种相似性是不可忽视的,因为当实数轴与垂直虚轴结合时,就会产生所谓的复平面。稍后再讲。

上述顺序总结为:
$$
\begin{gathered}
i^{4 n}=1 \\
i^{4 n+1}=i \\
i^{4 n+2}=-1 \\
i^{4 n+3}=-i
\end{gathered}
$$
其中 $n \in \mathbb{N}$.

那么负幂呢?当然,它们也是可能的。考虑$i^{-1}$,它的计算如下:
$$
i^{-1}=\frac{1}{i}=\frac{1(-i)}{i(-i)}=\frac{-i}{1}=-i
$$
类似的,
$$
i^{-2}=\frac{1}{i^{2}}=\frac{1}{-1}=-1
$$
和
$$
i^{-3}=i^{-1} i^{-2}=-i(-1)=i .
$$
与负幂增加相关的顺序是:
\begin{center}
\begin{tabular}{lcccccc}
\hline
$i^{0}$ & $i^{-1}$ & $i^{-2}$ & $i^{-3}$ & $i^{-4}$ & $i^{-5}$ & $i^{-6}$ \\
\hline
1 & $-i$ & $-1$ & $i$ & 1 & $-i$ & $-1$ \\
\hline
\end{tabular}
\end{center}
这一次循环模式被反转为$(1,-i,-1, i, 1, \ldots)$,并且类似于模式$(x,-y,-x, y, x, \ldots)$,后者是通过顺时针旋转笛卡尔轴而生成的。

也许最奇怪的幂是它本身:$i^{i}$,它恰好等于$e^{-\pi / 2}=$ $0.207879576 \ldots$,这在第四章中有解释。回顾了虚数$i$的某些特性之后,让我们看看当它与实数结合时会发生什么。

\section{复数}
根据定义,复数是实数和虚数的和,其形式表示为
$$
z=a+c \quad a \in \mathbb{R}, \quad c \in \mathbb{I} .
$$
我们也可以写成
$$
z=a+b i \quad a, b \in \mathbb{R}, \quad i^{2}=-1
$$
这个复数集合被标记为$\mathbb{C}$,这允许我们写作$z \in\mathbb{C}$。例如,$3+ 4i $是一个复数,其中3是实部,$ 4i $是虚部。以下都是复数:
$$
3, \quad 3+4 i, \quad-4-6 i, \quad 7 i, \quad 5.5+6.7 i .
$$
实数也是复数——只是没有虚数部分。这就引出了实数和虚数是复数的子集的观点,具体表述如下:
$$
\begin{aligned}
& \mathbb{R} \subset \mathbb{C} \\
& \mathbb{I} \subset \mathbb{C}
\end{aligned}
$$
其中$\subset$表示是子集。

虽然一些数学家将$i$放在乘数之前:$ i4 $,但其他人将它放在乘数之后:$ 4i $,这是本书中使用的惯例。然而,当$i$与三角函数相关联时,最好将它放在函数的前面,以避免与函数的角度混淆。例如,$\sin \alpha i$可以表示这个角是虚数,而$i \sin \alpha$则表示$ sin \alpha$的值是虚数。

因此,复数可以用各种方式来构造:
$$
\sin \alpha+i \cos \beta, \quad 2-i \tan \alpha, \quad 23+x^{2} i
$$

一般来说,我们把一个复数写成$a+ bi $,并使它服从真实代数的正常规则。我们需要记住的是,每当遇到$i^{2}$时,它都会被$-1$替换。例如:
$$
\begin{aligned}
(2+3 i)(3+4 i) & =2 \times 3+2 \times 4 i+3 i \times 3+3 i \times 4 i \\
& =6+8 i+9 i+12 i^{2} \\
& =6+17 i-12 \\
& =-6+17 i .
\end{aligned}
$$

\section{复数的加减法}
给出两个复数
$$
\begin{aligned}
& z_{1}=a_{1}+b_{1} i \\
& z_{2}=a_{2}+b_{2} i
\end{aligned}
$$
然后
$$
z_{1} \pm z_{2}=\left(a_{1} \pm a_{2}\right)+\left(b_{1} \pm b_{2}\right) i
$$
实部和虚部分别加或减。只要$a_{1}, b_{1}, a_{2}, b_{2} \in \mathbb{R}$。

示例:
$$
\begin{aligned}
z_{1} & =2+3 i \\
z_{2} & =4+2 i \\
z_{1}+z_{2} & =6+5 i \\
z_{1}-z_{2} & =-2+i .
\end{aligned}
$$

\section{复数乘以标量}
用普通代数规则将复数与标量相乘。例如,复数$a+ bi $乘以标量$\lambda$,如下所示:
$$
\lambda(a+b i)=\lambda a+\lambda b i
$$
一个具体的例子是
$$
3(2+5 i)=6+15 i .
$$

\section{复数乘积}
给出两个复数
$$
\begin{aligned}
& z_{1}=a_{1}+b_{1} i \\
& z_{2}=a_{2}+b_{2} i
\end{aligned}
$$
他们的乘积是
$$
\begin{aligned}
z_{1} z_{2} & =\left(a_{1}+b_{1} i\right)\left(a_{2}+b_{2} i\right) \\
& =a_{1} a_{2}+a_{1} b_{2} i+b_{1} a_{2} i+b_{1} b_{2} i^{2} \\
& =\left(a_{1} a_{2}-b_{1} b_{2}\right)+\left(a_{1} b_{2}+b_{1} a_{2}\right) i
\end{aligned}
$$
这是另一个复数,确认运算是闭合的。例如:
$$
\begin{aligned}
z_{1} & =3+4 i \\
z_{2} & =3-2 i \\
z_{1} z_{2} & =(3+4 i)(3-2 i) \\
& =9-6 i+12 i-8 i^{2} \\
& =9+6 i+8 \\
& =17+6 i .
\end{aligned}
$$
请注意,复数的加法、减法和乘法遵循普通的代数公理。
\subsection{复数的平方}
给定一个复数$z$,它的平方$z^{2}$为:
$$
\begin{aligned}
z & =a+b i \\
z^{2} & =(a+b i)(a+b i) \\
& =\left(a^{2}-b^{2}\right)+2 a b i
\end{aligned}
$$
举例:
$$
\begin{aligned}
z & =4+3 i \\
z^{2} & =(4+3 i)(4+3 i) \\
& =\left(4^{2}-3^{2}\right)+2 \times 4 \times 3 i \\
& =7+24 i
\end{aligned}
$$

\section{复数的范数}
复数$z$的范数、模或绝对值被写成$|z|$,根据定义是
$$
\begin{aligned}
z & =a+b i \\
|z| & =\sqrt{a^{2}+b^{2}}
\end{aligned}
$$
例如,$3+ 4i $的范数是5。当我们讨论复数的极坐标表示时,我们会看到为什么会这样。

\section{复共轭}
两个复数的乘积,它们之间唯一的区别是虚数部分的符号,会得到一个特殊的结果:
$$
\begin{aligned}
(a+b i)(a-b i) & =a^{2}-a b i+a b i-b^{2} i^{2} \\
& =a^{2}+b^{2} .
\end{aligned}
$$
这种类型的乘积总是得到一个实数,并用于求解两个复数的商。因为这个实值是一个非常有趣的结果,$a-b i$被称为$z=a+ bi $的复共轭,它可以用一根横线如 $\bar{z}$ 来写,也可以用星号如 $z^{*}$ 来写,这意味着
$$
z z^{*}=a^{2}+b^{2}=|z|^{2} .
$$
举例:
$$
\begin{aligned}
z & =3+4 i \\
z^{*} & =3-4 i \\
z z^{*} & =9+16=25
\end{aligned}
$$

\section{两个复数的商}
复共轭为我们提供了一种用一个复数除以另一个复数的机制。例如,除数
$$
\frac{a_{1}+b_{1} i}{a_{2}+b_{2} i}
$$
通过将分子和分母乘以分母的复共轭$a_{2}-b_{2} i$来得到实分母:
$$
\begin{aligned}
\frac{a_{1}+b_{1} i}{a_{2}+b_{2} i} & =\frac{\left(a_{1}+b_{1} i\right)\left(a_{2}-b_{2} i\right)}{\left(a_{2}+b_{2} i\right)\left(a_{2}-b_{2} i\right)} \\
& =\frac{a_{1} a_{2}-a_{1} b_{2} i+b_{1} a_{2} i-b_{1} b_{2} i^{2}}{a_{2}^{2}+b_{2}^{2}} \\
& =\frac{a_{1} a_{2}+b_{1} b_{2}}{a_{2}^{2}+b_{2}^{2}}+\frac{b_{1} a_{2}-a_{1} b_{2}}{a_{2}^{2}+b_{2}^{2}} i
\end{aligned}
$$
举例,求解
$$
\frac{4+3 i}{3+4 i}
$$
上下同时乘以复数共轭 $3-4 i$ :
$$
\begin{aligned}
\frac{4+3 i}{3+4 i} & =\frac{(4+3 i)(3-4 i)}{(3+4 i)(3-4 i)} \\
& =\frac{12-16 i+9 i-12 i^{2}}{25} \\
& =\frac{24}{25}-\frac{7}{25} i
\end{aligned}
$$

\section{复数的倒数}
要计算$z=a+ bi $的倒数,我们从这里开始
$$
z^{-1}=\frac{1}{z} \text {. }
$$
上下同时乘以$z^{*}$,得到
$$
z^{-1}=\frac{z^{*}}{z z^{*}} .
$$
但我们之前已经证明了$z z^{*}=|z|^{2}$,因此,
$$
\begin{aligned}
z^{-1} & =\frac{z^{*}}{|z|^{2}} \\
& =\left(\frac{a}{a^{2}+b^{2}}\right)-\left(\frac{b}{a^{2}+b^{2}}\right) i .
\end{aligned}
$$
举例来说,$3+ 4i $的倒数是
$$
(3+4 i)^{-1}=\frac{3}{25}-\frac{4}{25} i .
$$
让我们用$3+4 i$乘以它的倒数来测试这个结果:
$$
(3+4 i)\left(\frac{3}{25}-\frac{4}{25} i\right)=\frac{9}{25}-\frac{12}{25} i+\frac{12}{25} i+\frac{16}{25}=1
$$
这证实了结果的正确性。

\section{$i$的平方根}
为了找到$\sqrt{i}$,我们假设根是复根。因此,我们从
$$
\begin{aligned}
i & =(a+b i)(a+b i) \\
& =a^{2}+2 a b i-b^{2} \\
& =a^{2}-b^{2}+2 a b i
\end{aligned}
$$
将实部和虚部分别相等
$$
\begin{array}{r}
a^{2}-b^{2}=0 \\
2 a b=1 .
\end{array}
$$
由此我们推导出
$$
a=b=\frac{\sqrt{2}}{2} \text {. }
$$
因此,根是
$$
\sqrt{i}=\pm \frac{\sqrt{2}}{2}(1+i) \text {. }
$$
让我们通过平方根来测试这个结果,以确保答案是$i$:
$$
\begin{aligned}
\left(\frac{\sqrt{2}}{2}(1+i)\right)\left(\frac{\sqrt{2}}{2}(1+i)\right) & =\frac{1}{2} 2 i=i \\
\left(-\frac{\sqrt{2}}{2}(1+i)\right)\left(-\frac{\sqrt{2}}{2}(1+i)\right) & =\frac{1}{2} 2 i=i .
\end{aligned}
$$
为了完备起见,让我们计算$\sqrt{-i}$:
$$
\begin{aligned}
-i & =(a+b i)(a+b i) \\
& =a^{2}+2 a b i-b^{2} \\
& =a^{2}-b^{2}+2 a b i
\end{aligned}
$$
将实部和虚部分别相等
$$
\begin{aligned}
a^{2}-b^{2} & =0 \\
2 a b & =-1 .
\end{aligned}
$$
由此我们推导出
$$
a=b=\frac{\sqrt{2}}{2} i .
$$
因此,根是
$$
\sqrt{-i}=\pm \frac{\sqrt{2}}{2}(1-i) .
$$
再次,让我们通过平方根来测试这个结果,以确保答案是$-i$:
$$
\begin{aligned}
& \left(\frac{\sqrt{2}}{2}(1-i)\right)\left(\frac{\sqrt{2}}{2}(1-i)\right)=\frac{1}{2}(-2 i)=-i \\
& \left(-\frac{\sqrt{2}}{2}(1-i)\right)\left(-\frac{\sqrt{2}}{2}(1-i)\right)=\frac{1}{2}(-2 i)=-i \text {. }
\end{aligned}
$$
下一章我们将用这些根来研究复数的旋转性质。
\section{域结构}
复数$\mathbb{C}$是一个域,因为它们满足前面为域定义的规则。

\section{有序对}
到目前为止,我们选择用$a+ bi $来表示复数,这样我们可以区分实部和虚部。然而,有一件事我们不能假设是实部总是在前面,虚部在后,因为$b i+a$仍然是一个复数。因此,可以使用两个函数来提取实系数和虚系数,如下所示
$$
\begin{aligned}
& \operatorname{Re}(a+b i)=a \\
& \operatorname{Im}(a+b i)=b
\end{aligned}
$$
这就引出了用有序对表示复数的想法,其中有序是有保证的。

我们现在可以将复数集合$\mathbb{C}$定义为实数有序对$(a, b)$的集合$\mathbb{R}^{2}$,并将加法和乘法的公理重写为:
\begin{align}
z_{1} & =\left(a_{1}, b_{1}\right) \notag\\
z_{2} & =\left(a_{2}, b_{2}\right) \notag\\
z_{1}+z_{2} & =\left(a_{1}+a_{2}, b_{1}+b_{2}\right) \label{Z.1}\\
z_{1} z_{2} & =\left(a_{1} a_{2}-b_{1} b_{2}, b_{1} a_{2}+a_{1} b_{2}\right) .\label{Z.2}
\end{align}
将复数写成有序对是一个伟大的贡献,最早由Hamilton在1833年提出。这种表示法非常简洁,没有任何虚构的术语,可以在任何需要的时候添加。

现在我们将使用公式(3.1)和(3.2)将两个复数相乘。首先,我们将使用常规符号求积,然后使用有序对。
$$
\begin{aligned}
z_{1} & =6+2 i \\
z_{2} & =4+3 i \\
z_{1} z_{2} & =(6+2 i)(4+3 i) \\
& =24+18 i+8 i-6 \\
& =18+26 i .
\end{aligned}
$$
接下来,使用有序对和(3.1)和(3.2):
$$
\begin{aligned}
z_{1} & =(6,2) \\
z_{2} & =(4,3) \\
z_{1} z_{2} & =(6,2)(4,3) \\
& =(24-6,18+8) \\
& =(18,26)
\end{aligned}
$$
这是正确的。

让我们继续发展一个基于有序对的代数,它和复数的代数是一样的。我们从以下写法开始
$$
\begin{aligned}
z & =(a, b) \\
& =(a, 0)+(0, b) \\
& =a(1,0)+b(0,1)
\end{aligned}
$$
它创建了单位有序对$(1,0)$和$(0,1)$。

现在让我们计算乘积$(1,0)(1,0)$:
$$
\begin{aligned}
(1,0)(1,0) & =(1-0,0) \\
& =(1,0)
\end{aligned}
$$
这表明$(1,0)$的行为类似于实数1,即$(1,0)=1$。

接下来,让我们计算乘积$(0,1)(0,1)$:
$$
\begin{aligned}
(0,1)(0,1) & =(0-1,0) \\
& =(-1,0)
\end{aligned}
$$
也就是实数$-1$:
$$
(0,1)^{2}=-1
$$
或
$$
(0,1)=\sqrt{-1} \quad \text { 且是一个虚数 }
$$
这意味着有序对$(a, b)$及其相关规则表示一个复数,即$(a, b) \equiv a+ bi $。

\subsection{乘以标量}
我们已经熟悉了这个规则
$$
\lambda(a, b)=(\lambda a, \lambda b)
$$
这和复数代数是一致的。

\subsection{复共轭}
$z=a+ bi $的共轭定义为$z^{*}=a- bi $,以有序对的形式表示为$z^{*}=(a,-b)$。使用(3.2),我们有
$$
\begin{aligned}
z & =(a, b) \\
z^{*} & =(a,-b) \\
z z^{*} & =(a, b)(a,-b) \\
& =\left(a^{2}+b^{2}, b a-a b\right) \\
& =\left(a^{2}+b^{2}, 0\right) \\
& =a^{2}+b^{2}
\end{aligned}
$$
这是正确的。

\subsection{商}
解析$z_{1} / z_{2}$的技巧是将表达式乘以$z_{2}^{*} / z_{2}^{*}$,使用有序对就是
$$
\begin{aligned}
\frac{z_{1}}{z_{2}} & =\frac{\left(a_{1}, b_{1}\right)}{\left(a_{2}, b_{2}\right)} \\
& =\frac{\left(a_{1}, b_{1}\right)}{\left(a_{2}, b_{2}\right)} \frac{\left(a_{2},-b_{2}\right)}{\left(a_{2},-b_{2}\right)} \\
& =\frac{\left(a_{1} a_{2}+b_{1} b_{2}, b_{1} a_{2}-a_{1} b_{2}\right)}{\left(a_{2}^{2}+b_{2}^{2}, 0\right)} \\
& =\left(\frac{a_{1} a_{2}+b_{1} b_{2}}{a_{2}^{2}+b_{2}^{2}}, \frac{b_{1} a_{2}-a_{1} b_{2}}{a_{2}^{2}+b_{2}^{2}}\right) .
\end{aligned}
$$

\subsection{逆}
我们之前已经证明了$z^{-1}$是
$$
z^{-1}=\frac{z^{*}}{z z^{*}}
$$
使用有序对是
$$
\begin{aligned}
z & =(a, b) \\
z^{-1} & =\frac{(a,-b)}{(a, b)(a,-b)}\\
& =\frac{(a,-b)}{\left(a^{2}+b^{2}, 0\right)} \\
& =\left(\frac{a}{a^{2}+b^{2}}, \frac{-b}{a^{2}+b^{2}}\right)
\end{aligned}
$$
从上面的定义可以明显看出,有序对为表示复数提供了另一种表示法,其中虚数特征嵌入在乘积公理中。我们还将使用有序对定义具有三个虚数的四元数,这些虚数在乘积公理中仍然是隐藏的。

\section{复数的矩阵表示}
由于四元数有矩阵表示,也许我们应该研究复数的矩阵表示。我们可以推断,复数的矩阵$\mathbf{C}$是另外两个矩阵的和,分别表示实的$\mathbf{R}$和虚的$\mathbf{I}$部分:
$$
\mathbf{C}=\mathbf{R}+\mathbf{I}
$$
它又可以写成
$$
\mathbf{C}=a \hat{\mathbf{R}}+b \hat{\mathbf{I}} \quad a, b \in \mathbb{R}
$$
其中 $\hat{\mathbf{R}} \equiv 1$ 且 $\hat{\mathbf{I}} \equiv i$。

与1等价的矩阵是$2 \times 2$ 单位矩阵:
$$
\left[\begin{array}{ll}
1 & 0 \\
0 & 1
\end{array}\right] \text {. }
$$
虽然我只是暗示$ i $可以被视为某种旋转算子,但这是一种完美的可视化方法。在第四章中,我们将发现复数乘以$i$可以有效地旋转$90^{\circ}$。所以目前,它可以用$90^{\circ}$的旋转矩阵表示:
$$
\left[\begin{array}{cc}
\cos 90^{\circ} & -\sin 90^{\circ} \\
\sin 90^{\circ} & \cos 90^{\circ}
\end{array}\right]=\left[\begin{array}{cc}
0 & -1 \\
1 & 0
\end{array}\right]
$$
我们可以这样写:
$$
\left[\begin{array}{cc}
a & -b \\
b & a
\end{array}\right]=a\left[\begin{array}{ll}
1 & 0 \\
0 & 1
\end{array}\right]+b\left[\begin{array}{cc}
0 & -1 \\
1 & 0
\end{array}\right]
$$
注意,表示 $i$ 平方到$-1$的矩阵:
$$
\left[\begin{array}{cc}
0 & -1 \\
1 & 0
\end{array}\right]\left[\begin{array}{cc}
0 & -1 \\
1 & 0
\end{array}\right]=-1\left[\begin{array}{ll}
1 & 0 \\
0 & 1
\end{array}\right]
$$
现在让我们用矩阵表示法来表示用于复数的所有算术运算。
\subsection{加减法}
两个复数相加或相减写作如下:
$$
\begin{aligned}
z_{1} & =a_{1}+b_{1} i \\
z_{2} & =a_{2}+b_{2} i \\
z_{1} & =\left[\begin{array}{cc}
a_{1} & -b_{1} \\
b_{1} & a_{1}
\end{array}\right] \\
z_{2} & =\left[\begin{array}{ll}
a_{2} & -b_{2} \\
b_{2} & a_{2}
\end{array}\right] \\
z_{1} \pm z_{2} & =\left[\begin{array}{ll}
a_{1} & -b_{1} \\
b_{1} & a_{1}
\end{array}\right] \pm\left[\begin{array}{ll}
a_{2} & -b_{2} \\
b_{2} & a_{2}
\end{array}\right] \\
& =\left[\begin{array}{ll}
a_{1} \pm a_{2} & -\left(b_{1} \pm b_{2}\right) \\
b_{1} \pm b_{2} & a_{1} \pm a_{2}
\end{array}\right]
\end{aligned}
$$

\subsection{乘积}
两个复数的乘积计算如下:
$$
\begin{aligned}
z_{1} & =a_{1}+b_{1} i \\
z_{2} & =a_{2}+b_{2} i \\
z_{1} z_{2} & =\left[\begin{array}{cc}
a_{1} & -b_{1} \\
b_{1} & a_{1}
\end{array}\right]\left[\begin{array}{cc}
a_{2} & -b_{2} \\
b_{2} & a_{2}
\end{array}\right] \\
& =\left[\begin{array}{cc}
a_{1} a_{2}-b_{1} b_{2} & -\left(a_{1} b_{2}+b_{1} a_{2}\right) \\
a_{1} b_{2}+b_{1} a_{2} & a_{1} a_{2}-b_{1} b_{2}
\end{array}\right]
\end{aligned}
$$

\subsection{范数的平方}
范数的平方就是矩阵的行列式:
$$
\begin{aligned}
z & =a+b i \\
|z|^{2} & =\left|\begin{array}{cc}
a & -b \\
b & a
\end{array}\right|=a^{2}+b^{2} .
\end{aligned}
$$

\subsection{复数共轭}
复数z的复共轭表示为
$$
\begin{aligned}
z & =a+b i \\
z^{*} & =a-b i \\
& =\left[\begin{array}{cc}
a & b \\
-b & a
\end{array}\right]
\end{aligned}
$$

并且乘积 $z z^{*}=a^{2}+b^{2}$ :

$$
\begin{aligned}
z z^{*} & =\left[\begin{array}{cc}
a & -b \\
b & a
\end{array}\right]\left[\begin{array}{cc}
a & b \\
-b & a
\end{array}\right] \\
& =\left[\begin{array}{cc}
a^{2}+b^{2} & 0 \\
0 & a^{2}+b^{2}
\end{array}\right] .
\end{aligned}
$$

\subsection{逆}
$2 \times 2$矩阵$\mathbf{A}$的逆由如下给出
$$
\begin{aligned}
\mathbf{A} & =\left[\begin{array}{ll}
a_{11} & a_{12} \\
a_{21} & a_{22}
\end{array}\right] \\
\mathbf{A}^{-1} & =\frac{1}{a_{11} a_{22}-a_{12} a_{21}}\left[\begin{array}{cc}
a_{22} & -a_{12} \\
-a_{21} & a_{12}
\end{array}\right]
\end{aligned}
$$

因此,$z$ 的逆由如下给出
$$
\begin{aligned}
z & =a+b i \\
z & =\left[\begin{array}{cc}
a & -b \\
b & a
\end{array}\right] \\
z^{-1} & =\frac{1}{a^{2}+b^{2}}\left[\begin{array}{cc}
a & b \\
-b & a
\end{array}\right] \\
& =\frac{a}{a^{2}+b^{2}}-\frac{b}{a^{2}+b^{2}} i .
\end{aligned}
$$

\subsection{商}
两个复数的商计算如下:
$$
\begin{aligned}
z_{1} & =a_{1}+b_{1} i \\
z_{2} & =a_{2}+b_{2} i \\
\frac{z_{1}}{z_{2}} & =z_{1} z_{2}^{-1} \\
& =\left[\begin{array}{cc}
a_{1} & -b_{1} \\
b_{1} & a_{1}
\end{array}\right] \frac{1}{a_{2}^{2}+b_{2}^{2}}\left[\begin{array}{cc}
a_{2} & b_{2} \\
-b_{2} & a_{2}
\end{array}\right] \\
& =\frac{1}{a_{2}^{2}+b_{2}^{2}}\left[\begin{array}{cc}
a_{1} a_{2}+b_{1} b_{2} & -\left(b_{1} a_{2}-a_{1} b_{2}\right) \\
b_{1} a_{2}-a_{1} b_{2} & a_{1} a_{2}+b_{1} b_{2}
\end{array}\right] .
\end{aligned}
$$

\section{总结}
在本章中,我们证明了复数的集合是一个域,因为它们满足闭合性、结合律、分配性、单位元和逆的要求。我们还证明了复数与有序对之间存在一一对应关系,复数可以表示为矩阵,这使得我们可以用矩阵运算来计算所有的复数运算。

如果这是你第一次遇到复数,你可能会发现它们一方面很奇怪,另一方面又很神奇。仅仅通过声明$i$的平方等于$-1$的存在,就开辟了一个统一数学领域的新数字系统。

\subsection{运算符总结}
\subsubsection*{定义}
$\mathbb{I}$是虚数的集合:$b i \in \mathbb{I}, b \in \mathbb{R}, i ^{2}=-1$。$\mathbb{C}$是复数的集合,是一个域。
$$
\begin{aligned}
z & =a+b i \quad a \in \mathbb{R}, \quad b i \in \mathbb{I}, \quad z \in \mathbb{C} \\
& =\left[\begin{array}{cc}
a & -b \\
b & a
\end{array}\right]
\end{aligned}
$$

\subsubsection*{有序对}
$$
a+b i \equiv(a, b)
$$

\subsubsection*{加减法}
$$
\begin{aligned}
z_{1} & =a_{1}+b_{1} i \\
z_{2} & =a_{2}+b_{2} i \\
z_{1} \pm z_{2} & =a_{1} \pm a_{2}+\left(b_{1} \pm b_{2}\right) i \\
& =\left[\begin{array}{lc}
a_{1} \pm a_{2} & -\left(b_{1} \pm b_{2}\right) \\
b_{1} \pm b_{2} & a_{1} \pm a_{2}
\end{array}\right]
\end{aligned}
$$

\subsubsection*{乘积}
$$
\begin{aligned}
z_{1} z_{2} & =\left(a_{1}+b_{1} i\right)\left(a_{2}+b_{2} i\right) \\
& =\left(a_{1} a_{2}-b_{1} b_{2}\right)+\left(a_{1} b_{2}+b_{1} a_{2}\right) i \\
& =\left[\begin{array}{cc}
a_{1} a_{2}-b_{1} b_{2} & -\left(a_{1} b_{2}+b_{1} a_{2}\right) \\
a_{1} b_{2}+b_{1} a_{2} & a_{1} a_{2}-b_{1} b_{2}
\end{array}\right]
\end{aligned}
$$

\subsubsection*{平方}
$$
\begin{aligned}
z^{2} & =(a+b i)^{2} \\
& =\left(a^{2}-b^{2}\right)+2 a b i \\
& =\left[\begin{array}{cc}
a^{2}-b^{2} & -2 a b \\
2 a b & a^{2}-b^{2}
\end{array}\right]
\end{aligned}
$$

\subsubsection*{范数}
$$
\begin{aligned}
z & =a+b i \\
|z| & =\sqrt{a^{2}+b^{2}} \\
|z|^{2} & =\left|\begin{array}{cc}
a & -b \\
b & a
\end{array}\right|
\end{aligned}
$$

\subsubsection*{复共轭}
$$
\begin{aligned}
z & =a+b i \\
z^{*} & =a-b i \\
& =\left[\begin{array}{cc}
a & b \\
-b & a
\end{array}\right]
\end{aligned}
$$

\subsubsection*{商}
$$
\begin{aligned}
\frac{z_{1}}{z_{2}} & =\frac{a_{1}+b_{1} i}{a_{2}+b_{2} i} \\
& =\frac{a_{1} a_{2}+b_{1} b_{2}}{a_{2}^{2}+b_{2}^{2}}+\frac{b_{1} a_{2}-a_{1} b_{2}}{a_{2}^{2}+b_{2}^{2}} i \\
& =z_{1} z_{2}^{-1} \\
& =\left[\begin{array}{cc}
a_{1} & -b_{1} \\
b_{1} & a_{1}
\end{array}\right] \frac{1}{a_{2}^{2}+b_{2}^{2}}\left[\begin{array}{cc}
a_{2} & b_{2} \\
-b_{2} & a_{2}
\end{array}\right] \\
& =\frac{1}{a_{2}^{2}+b_{2}^{2}}\left[\begin{array}{cc}
a_{1} a_{2}+b_{1} b_{2} & -\left(b_{1} a_{2}-a_{1} b_{2}\right) \\
b_{1} a_{2}-a_{1} b_{2} & a_{1} a_{2}+b_{1} b_{2}
\end{array}\right]
\end{aligned}
$$

\subsubsection*{逆}
$$
\begin{aligned}
z & =a+b i \\
\frac{1}{z} & =\frac{z^{*}}{|z|^{2}} \\
& =\left(\frac{a}{a^{2}+b^{2}}\right)-\left(\frac{b}{a^{2}+b^{2}}\right) i \\
& =\frac{1}{a^{2}+b^{2}}\left[\begin{array}{cc}
a & b \\
-b & a
\end{array}\right]
\end{aligned}
$$

\subsubsection*{$\pm i$ 的平方根}
$$
\begin{aligned}
\sqrt{i} & =\pm \frac{\sqrt{2}}{2}(1+i) \\
& =\pm \frac{\sqrt{2}}{2}\left[\begin{array}{cc}
1 & -1 \\
1 & 1
\end{array}\right] \\
\sqrt{-i} & =\pm \frac{\sqrt{2}}{2}(1-i) \\
& =\pm \frac{\sqrt{2}}{2}\left[\begin{array}{cc}
1 & 1 \\
-1 & 1
\end{array}\right]
\end{aligned}
$$

\section{样例}
下面是一些进一步使用上述思想的示例。在某些情况下,包括测试来确认结果。


\begin{example}
    $z_{1}$ 和 $z_{2}$ 的加减:
    $$
    \begin{aligned}
    z_{1} & =12+6 i \\
    z_{2} & =10-4 i \\
    z_{1}+z_{2} & =22+2 i \\
    z_{1}-z_{2} & =2+10 i .
    \end{aligned}
    $$
\end{example}

\begin{example}
    计算乘积 $z_{1} z_{2}$ :
    $$
    \begin{aligned}
    z_{1} z_{2} & =(12+6 i)(10-4 i) \\
    & =144+12 i .
    \end{aligned}
    $$
\end{example}

\begin{example}
    $z_{1}$乘以 $i$ :
    $$
    \begin{aligned}
    z_{1} i & =(12+6 i) i \\
    & =-6+12 i
    \end{aligned}
    $$
\end{example}

\begin{example}
    计算如下的范数:
    $$
    \begin{aligned}
    |5+12 i| & =\sqrt{5^{2}+12^{2}}=13 \\
    |\pm 1| & =1 \\
    |\pm i| & =1 \\
    |1 \pm i| & =\sqrt{2} .
    \end{aligned}
    $$
\end{example}

\begin{example}
    计算下列数的共轭复数:
    $$
    \begin{aligned}
    (2+3 i)^{*} & =2-3 i \\
    1^{*} & =1 \\
    i^{*} & =-i \\
    (-i)^{*} & =i .
    \end{aligned}
    $$
\end{example}

\begin{example}
    例6 计算商$(2+ 3i) /(3+ 4i)$。
    $$
    \begin{aligned}
    \frac{2+3 i}{3+4 i} & =\frac{(2+3 i)}{(3+4 i)} \frac{(3-4 i)}{(3-4 i)} \\
    & =\frac{6-8 i+9 i+12}{25} \\
    & =\frac{18}{25}+\frac{1}{25} i
    \end{aligned}
    $$
    
    测试:
    $$
    \begin{aligned}
    (3+4 i)\left(\frac{18}{25}+\frac{1}{25} i\right) & =\frac{54}{25}+\frac{3}{25} i+\frac{72}{25} i-\frac{4}{25} \\
    & =2+3 i .
    \end{aligned}
    $$
\end{example}

\begin{example}
    $2+3 i$除以$i$。
    $$
    \begin{aligned}
    \frac{2+3 i}{i} & =\frac{(2+3 i)}{i} \frac{(-i)}{(-i)} \\
    & =\frac{-2 i+3}{1} \\
    & =3-2 i .
    \end{aligned}
    $$
    
    测试:
    $$
    i(3-2 i)=2+3 i
    $$
\end{example}

\begin{example}
    $2+3 i$除以$-i$。
    $$
    \begin{aligned}
    \frac{2+3 i}{-i} & =\frac{(2+3 i)}{-i} \frac{(i)}{(i)} \\
    & =\frac{2 i-3}{1} \\
    & =-3+2 i .
    \end{aligned}
    $$
    
    测试:
    $$
    -i(-3+2 i)=2+3 i
    $$
\end{example}

\begin{example}
    计算 $2+3 i$ 的逆。
    $$
    \begin{aligned}
    \frac{1}{2+3 i} & =\frac{1}{(2+3 i)} \frac{(2-3 i)}{(2-3 i)} \\
    & =\frac{2-3 i}{13} \\
    & =\frac{2}{13}-\frac{3}{13} i
    \end{aligned}
    $$
    
    测试:
    $$
    (2+3 i)\left(\frac{2}{13}-\frac{3}{13} i\right)=\frac{4}{13}-\frac{6}{13} i+\frac{6}{13} i+\frac{9}{13}=1
    $$
\end{example}

\begin{example}
    计算 $i$ 的逆。
    $$
    \begin{aligned}
    \frac{1}{i} & =\frac{1}{i} \frac{(-i)}{(-i)} \\
    & =\frac{-i}{1}=-i .
    \end{aligned}
    $$
    
    测试:
    $$
    i(-i)=1
    $$
\end{example}

\begin{example}
    计算 $-i$ 的逆。
    $$
    \begin{aligned}
    \frac{1}{-i} & =\frac{1}{-i} \frac{(i)}{(i)} \\
    & =\frac{i}{1}=i
    \end{aligned}
    $$
    
    测试:
    $$
    -i i=1 .
    $$
\end{example}
