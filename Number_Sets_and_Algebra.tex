\chap{数集与代数}
\section{介绍}
在本章中,我们将回顾数集的一些基本思想,以及如何用算术和代数方法处理它们。我们简要地看一下表达式和方程,以及用于构造和求值的规则。这些反过来又揭示了用所谓的复数扩展日常数字的需要。

本章的第二部分用于定义组、环和域。

\section{数集}
\subsection{自然数}
自然数是整数1,2,3,4等,根据定义(DIN 5473),自然数的集合和零$\{0,1,2,3,4,\ldots\}$由符号$\mathbb{N}$表示,我们使用:
$$
\mathbb{N}=\{0,1,2,3,4, \ldots\} .
$$
该语句
$$
k \in \mathbb{N}
$$
意味着$k$属于集合$\mathbb{N}$,其中$\in$表示属于,或者换句话说,$k$是一个自然数。我们在本书中使用这种符号,以确保对所使用的数值量的类型没有混淆。

$ \mathbb {N} ^{*}$用于表示集合$\{1,2,3,4,\ldots \} $。

\subsection{实数}
科学计算使用大量的数学对象,如标量、向量和矩阵。标量只有一个数值,而矢量有两个或两个以上的数字来编码矢量的大小和方向。矩阵是一个矩形数组的数字,可能有各种各样的属性。

十进制数构成由$\mathbb{R}$标识的实数集。这样的数字是有符号的,可以组织成一条线,延伸到$-\infty$和$+\infty$,其中包括0。无限的概念很奇怪,德国数学家康托(Georg Cantor, 1845-1918)研究过这个概念。康托尔还发明了集合论,证明了实数比自然数多。幸运的是,我们不需要在本书中使用这些概念。

\subsection{整数}
整数集$\mathbb{Z}$包含自然数及其负数:
$$
\mathbb{Z}=\{\ldots,-3,-2,-1,0,1,2,3, \ldots\} .
$$
$\mathbb{Z}$代表zahlen——德语“数字”的意思。


\subsection{有理数}
有理数的集合是$\mathbb{Q}$,包含如下形式的数:
$$
\frac{a}{b}
$$
其中 $a, b \in \mathbb{Z}$ 且 $b \neq 0$.

\section{算术运算}
我们使用加法、减法、乘法和除法运算来操作数字,其结果是闭的还是不闭的,或者是未定义的,这取决于底层集合。例如,当我们将两个自然数相加时,结果总是另一个自然数,因此,运算是封闭的:
$$
3+4=7 .
$$
然而,当我们减去两个自然数时,结果不一定是一个自然数。例如,尽管
$$
6-2=4
$$
是一个封闭的操作,
$$
2-6=-4
$$
不是封闭的,因为$-4$不是自然数集合中的一员。两个自然数的乘积通常是一个封闭的运算,但是除法会引起一些问题。首先,将一个偶数除以2是一个封闭运算:
$$
16 / 2=8 .
$$
然而,将一个奇数自然数除以一个偶数自然数得到一个十进制数:
$$
7 / 2=3.5
$$
并且不封闭,因为$3.5$不属于自然数集合。这是用集合语言写的
$$
3.5 \notin \mathbb{N}
$$
其中 $\notin$ 表示不属于.

将任何数字乘以零都得到零——这是一个封闭运算;然而,任何数字除以零都是没有定义的,必须排除。

实数没有任何与自然数相关的问题,并且在加法、乘法和除法上有闭包:
$$
\begin{aligned}
a+b & =c & & a, b, c \in \mathbb{R} \\
a b & =c & & a, b, c \in \mathbb{R} \\
a / b & =c & & a, b, c \in \mathbb{R} \text { and } b \neq 0 .
\end{aligned}
$$
注意,$a b$是$a \times b$的简写。


\section{公理}
当我们构造代数表达式时,我们使用称为公理的特定定律。对于加法和乘法,我们知道数字的分组对最终结果没有影响:例如$2+(4+6)=(2+4)+6$和$2 \times(3 \times 4)=$ $(2 \times 3) \times 4$。这是结合律公理,表示为:
$$
\begin{aligned}
a+(b+c) & =(a+b)+c \\
a(b c) & =(a b) c .
\end{aligned}
$$
我们也知道,当加或乘时,顺序对最终结果没有影响:例如$2+6=6+2$和$2 \times6=6 \times2$。这是交换公理,表示为:
$$
\begin{aligned}
a+b & =b+a \\
a b & =b a .
\end{aligned}
$$
代数表达式包含各种各样的乘积,涉及一个实数和一串实数,它们服从分配律:
$$
\begin{aligned}
a(b+c) & =a b+a c \\
(a+b)(c+d) & =a c+a d+b c+b d .
\end{aligned}
$$
我们回顾这些公理的原因是,它们不应该被视为刻在数学石头上的,而适用于所有被发明的东西。因为当我们讲到四元数时,我们会发现它们不遵守交换公理,这并不奇怪。如果你用过矩阵,你就会知道矩阵乘法也是不可交换的,但是是结合的。
\section{表达式}
使用上述公理,我们能够构造各种表达式,例如:
$$
\begin{aligned}
& a(2+c)-d / e+a-10 \\
& g /(a c-b d)+h /(d e-f g) .
\end{aligned}
$$
我们还使用符号来提高一个量的幂,如$n^{2}$。这个符号引入了另一组观察结果:
$$
\begin{aligned}
a^{n} a^{m} & =a^{n+m} \\
\frac{a^{n}}{a^{m}} & =a^{n-m} \\
\left(a^{n}\right)^{m} & =a^{n m} \\
\frac{a^{n}}{a^{n}} & =a^{0}=1 \\
\frac{1}{a^{n}} & =a^{-1} \\
a^{1 / n} & =\sqrt[n]{a} .
\end{aligned}
$$
接下来,我们必须包括各种各样的函数,比如平方根、正弦和余弦,这些函数看起来相当简单。但我们必须警惕他们。例如,按惯例,$\sqrt{16}=4$。但是,$x^{2}=16$有两个解:$\pm \sqrt{16}=\pm 4$。然而,$\sqrt{-16}$不存在自然数或实数解。因此,如果$a<0$,表达式$\sqrt{a}$就没有实根。

类似地,在处理正弦和余弦等三角函数时,我们必须记住,这些函数的值范围在$-1$和$+1$之间,包括0,这意味着如果将它们用作分母,结果可能是未定义的。例如,如果$\sin \alpha=0$,则此表达式未定义
$$
\frac{a}{\sin \alpha} \text {. }
$$

\section{等式}
接下来,我们来到方程,我们将表达式的值赋给变量。在大多数情况下,任务是直接的,并导致一个真实的结果,如
$$
x^{2}-16=0
$$
其中$x=\pm 4$。但有趣的是,只要把符号颠倒过来
$$
x^{2}+16=0
$$
我们创建了一个没有实解的方程。然而,有一个复杂的解决方案,这是第三章的主题。

\section{有序对}
$(a, b)$是一个有两个坐标或投影项的对象,其中第一个或左边的项与第二个或右边的项是不同的。例如,$(a, b)$与$(b, a)$是不同的,除非$a=b$。也许有序对的最好例子是$(x, y)$,它表示平面上的一个点,其中元素的顺序总是$x$-坐标后面跟着$y$-坐标。

在计算机图形学中,有序对和有序三元组被广泛用于表示平面上的点$(x, y)$,空间中的点$(x, y, z)$,以及诸如$(r, g, b)$和$(h, s, v)$等颜色值。在这些例子中,字段都是实值。没有什么可以阻止我们使用有序对来开发一个代数,它的行为就像另一个代数一样,我们将在第三章对复数和第五章对四元数这样做。现在,让我们探索一些可以操纵有序对的方法。

假设我们选择将一个通用的有序对描述为
$$
a=\left(a_{1}, a_{2}\right) \quad a_{1}, a_{2} \in \mathbb{R} .
$$
我们将定义两个对象的加法为
$$
\begin{aligned}
a & =\left(a_{1}, a_{2}\right) \\
b & =\left(b_{1}, b_{2}\right) \\
a+b & =\left(a_{1}+b_{1}, a_{2}+b_{2}\right) .
\end{aligned}
$$
举例:
$$
\begin{aligned}
a & =(2,3) \\
b & =(4,5) \\
a+b & =(6,8) .
\end{aligned}
$$
我们将乘积定义为
$$
a b=\left(a_{1} b_{1}, a_{2} b_{2}\right)
$$
使用上面的值,结果是什么
$$
a b=(8,15) .
$$
记住,我们说了算,规则是我们定的。

另一个规则将控制有序对如何响应标量乘法。例如:
$$
\begin{aligned}
\lambda\left(a_{1}, a_{2}\right) & =\left(\lambda a_{1}, \lambda a_{2}\right) \quad \lambda \in \mathbb{R} \\
3(2,3) & =(6,9) .
\end{aligned}
$$
有了上面的规则,我们就可以写了
$$
\begin{aligned}
\left(a_{1}, a_{2}\right) & =\left(a_{1}, 0\right)+\left(0, a_{2}\right) \\
& =a_{1}(1,0)+a_{2}(0,1)
\end{aligned}
$$
如果我们用乘法法则对$(1,0)$和$(0,1)$进行平方,我们得到
$$
\begin{aligned}
& (1,0)^{2}=(1,0) \\
& (0,1)^{2}=(0,1)
\end{aligned}
$$
这表明它们表现得像实数,这并不出人意料。

这似乎不是很有用,但是等着看在复数和四元数的上下文中会发生什么。

\section{群,环,域}
数学家们使用一系列令人眼花缭乱的名字来识别他们的发明,这些发明似乎每天都在出现。甚至“四元数”这个名字也不是原创的,在历史上经常出现在“士兵的四元数”的语境中:

“罗马人派出四人组或四人组成夜间警卫……”[19]。

在不太正式的情况下,让我们探索更多与本书所包含的思想相关的数学结构。

\subsection{群}
我们已经讨论了集合的概念,以及属于集合意味着什么。我们还发现,当我们对集合的成员应用某些算术运算时,我们可以确保闭包、非闭包或结果未定义。

当将集合与算术运算结合在一起时,可以方便地创建另一个实体:群,即集合,以及描述集合元素如何组合的公理。集合可能包含数字、矩阵、向量、四元数、多项式等,并在下面表示为$a,b$和$c$。

公理使用'$o$'符号来表示任何二进制操作,如$+,-, \times$。一个群是由一个集合和一个二元运算组成的。例如,我们可能希望在加法下形成一组整数:$(\mathbb{Z},+)$,或者我们可能希望检查四元数是否在乘法操作下形成一组:$(\mathbb{H}, \times)$。

要成为一个群,对于集合$S$,所有下列公理必须成立。特别地,在中必须存在一个特殊的单位元素$e \in S$,并且对于每一个$a \in S$,在中必须存在一个逆元素$a^{-1} \in S$,从而满足下列公理:

\begin{align*}
    \begin{aligned}
        & \textbf{闭合: }        && a \circ b \in S && a,b \in S.\\
        & \textbf{结合律: }  && (a \circ b) \circ c=a \circ(b \circ c) && a, b, c \in S.\\
        & \textbf{单位元:}    && a \circ e=e \circ a=a && a, e \in S.\\
        & \textbf{可逆元素: }    && a  \circ a^{-1}=a^{-1} \circ a=e && a, a^{-1}, e \in S.
    \end{aligned}
\end{align*}
我们将一个组描述为$(S, \circ)$,其中$S$是集合,' 0 '是操作。例如$(\mathbb{Z},+)$是一组进行加法运算的整数,$(\mathbb{R}, \times)$是一组进行乘法运算的实数。

让我们通过三个例子将这些公理生动地展现出来。

$(\mathbb{Z},+)$:整数$\mathbb{Z}$在加法运算下形成一个组:
$$
\begin{aligned}
\text { 闭合: } & -23+24=1 \\
\text { 结合律: } & (2+3)+4=2+(3+4)=9 \\
\text { 单位元: } & 2+0=0+2=2 \\
\text { 可逆: } & 2+(-2)=(-2)+2=0 .
\end{aligned}
$$
$(\mathbb{Z}, \times)$:整数$\mathbb{Z}$在乘法下不构成群:
$$
\begin{aligned}
\text { 闭合: } & -2 \times 4=-8 \\
\text { 结合律: } & (2 \times 3) \times 4=2 \times(3 \times 4)=24 \\
\text { 单位元: } & 2 \times 1=1 \times 2=2 \\
\text { 可逆性: } & 2^{-1}=0.5 \quad(0.5 \notin \mathbb{Z}) .
\end{aligned}
$$
并且,整数0没有逆。

$(\mathbb{Q}, \times)$:非零有理数在乘法下构成一个群:
$$
\begin{aligned}
\text { 闭合: } & \frac{2}{5} \times \frac{2}{3}=\frac{4}{15} \\
\text { 交换律: } & \left(\frac{2}{5} \times \frac{2}{3}\right) \times \frac{1}{2}=\frac{2}{5} \times\left(\frac{2}{3} \times \frac{1}{2}\right)=\frac{2}{15} \\
\text { 单位元: } & \frac{2}{3} \times \frac{1}{1}=\frac{1}{1} \times \frac{2}{3}=\frac{2}{3} \\
\text { 可逆性: } & \frac{2}{3} \times \frac{3}{2}=\frac{1}{1} \quad\left(\text { where } \frac{3}{2}=\left(\frac{2}{3}\right)^{-1}\right)
\end{aligned}
$$

\subsection{阿贝尔群}
最后,以挪威数学家Neils Henrik Abel(1802-1829)的名字命名的阿贝尔群,是一个元素的顺序不影响结果的群,即群是可交换的。因此有五个公理:闭合性、结合性、单位元、逆元和交换性:
$$
\textbf{Commutativity:} \qquad a \circ b=b \circ a \qquad a, b \in S
$$
例如,整数集在普通加法$(\mathbb{Z},+)$下形成一个阿贝尔群。然而,由于三维旋转一般不交换,三维空间中所有旋转的集合形成了一个非交换群。

\subsection{环}
环是一个扩展的群,在这里我们有一组可以相加和相乘的对象,遵循一些精确的公理。有实数圈、复数圈、整数圈、矩阵圈、方程圈、多项式圈等。环被正式定义为一个系统,其中$(S,+)$和$(S, \times)$为阿贝尔群,分配公理如下:
$$
\begin{array}{lll}
\textbf { 加法结合律: } & a+(b+c)=(a+b)+c & a, b, c \in S . \\
\textbf { 乘法结合律: } & a \times(b \times c)=(a \times b) \times c & a, b, c \in S . \\
\textbf { 分配律: } & a \times(b+c)=(a \times b)+(a \times c) & \text { 和 } \\
& (a+b) \times c=(a \times c)+(b \times c) & a, b, c \in S .
\end{array}
$$
例如,我们已经知道整数$\mathbb{Z}$在加法运算下形成了一个群,但它们也形成了一个环,因为集合满足上述公理:
$$
\begin{aligned}
& 2 \times(3 \times 4)=(2 \times 3) \times 4 \\
& 2 \times(3+4)=(2 \times 3)+(2 \times 4) \\
& (2+3) \times 4=(2 \times 4)+(3 \times 4) .
\end{aligned}
$$

\subsection{域}
虽然环支持加法和乘法,但它们不一定支持除法。然而,由于除法是如此重要的算术运算,因此创建域是为了支持除法,但有一个限制条件:不允许除0。这样我们就有了实数$\mathbb{R}$、有理数$\mathbb{Q}$以及复数$\mathbb{C}$的域。然而,我们将发现四元数不构成域,但它们确实形成了所谓的除法环。

因此,每个域都是一个环,但并不是每个环都是一个场。

\subsection{除法环}
除法环或除法代数是一个环,其中每个元素都有一个逆元素,前提是该元素非零。代数也支持非交换乘法。下面是除法环$(S,+, \times)$的正式描述:
\begin{align*}
    \begin{aligned}
        &\textbf{加法结合律: } & (a+b)+c=a+(b+c) && a, b, c \in S \\
        &\textbf{加法交换律: } & a+b=b+a && a, b \in S\\
        &\textbf{加法单位元0 :}    & 0+a=a+0 && a, 0 \in S\\
        &\textbf{加法可逆性: }       & a+(-a)=(-a)+a=0 && a,-a \in S\\
        &\textbf{乘法结合律:} & (a \times b) \times c=a \times(b \times c) && a, b, c \in S\\
        &\textbf{乘法单位元 1:} &1 \times a=a \times 1 && a, 1 \in S\\
        &\textbf{乘法可逆性:} & a \times a^{-1}=a^{-1} \times a=1 && a, a^{-1} \in S, a \neq 0\\
        &\textbf{分配律:} & a \times(b+c)=(a \times b)+(a \times c) && \text{和}\\
        &&(b+c) \times a=(b \times a)+(c \times a) && a, b, c \in S
    \end{aligned}
\end{align*}

1878年,德国数学家费迪南·格奥尔格·弗罗本乌斯(Ferdinand Georg Frobenius, 1849-1917)证明了只有三个符合结合律的除法代数:实数$\mathbb{R}$、复数$\mathbb{C}$和四元数$\mathbb{H}$。

\section{总结}
本章的目的是提醒你代数基础上的公理化系统,以及算术运算的结果如何可以是开的、闭的或未定义的。也许有序对、集合、组、域和环的一些概念是新的,它们被包括在内,因为这种符号经常与四元数关联使用。

当我们考虑复数的代数和后来的四元数时,所有这些想法都会再次出现。

\subsection{定义总结}
\subsubsection*{有序对}
一个具有两个可区分组件的对象:$(a, b)$使得$(a, b) \neq(b, a)$除非$a=b$。

\subsubsection*{集合}
定义:集合是对象的集合。

符号:$k \in \mathbb{Z}$表示$k$属于集合$\mathbb{Z}$。
\begin{align*}
    \begin{aligned}
        \mathbb{C} &\text{: 复数的集合}\\
        \mathbb{H} &\text{: 四元数集合}\\
        \mathbb{N} &\text{: 自然数集合}\\
        \mathbb{Q} &\text{: 有理数集合}\\
        \mathbb{R} &\text{: 实数集合}\\      
        \mathbb{Z} &\text{: 整数集。}\\
    \end{aligned}
\end{align*}

\subsubsection*{群}
定义:群$(S, \circ)$是一个集合$S$和一个二进制操作' $\circ$ '以及定义闭包、结合性、单位元和可逆元素的公理。
\begin{align*}
    \begin{aligned}
        &\textbf{闭合性:   } && a \circ b \in S && a, b \in S.\\
        &\textbf{交换律:   } && (a \circ b) \circ c=a \circ(b \circ c) && a, b, c \in S.\\
        &\textbf{单位元:   } && a \circ e=e \circ a=a && a, e \in S.\\
        &\textbf{可逆性:   } && a \circ a^{-1}=a^{-1} \circ a=e. && a, a^{-1}, e \in S.
    \end{aligned}
\end{align*}

\subsubsection*{环}
定义:环是一组元素可以加/减或乘的组合,使用一些精确的公理:
\begin{align*}
    \begin{aligned}
        & \textbf{加法结合律:} && a+(b+c)=(a+b)+c && a, b, c \in S.\\
        & \textbf{乘法结合律:} && a \times(b \times c)=(a \times b) \times c && a, b, c \in S.\\
        & \textbf{分配律:} && a \times(b+c)=(a \times b)+(a \times c) && \text{和}\\
&&& (a+b) \times c=(a \times c)+(b \times c) && a, b, c \in S.
    \end{aligned}
\end{align*}


\subsubsection*{域}
定义:域是一个支持除法的环。

\subsubsection*{除法环}
除法环的每个元素都有一个逆元素,条件是该元素非零。代数也支持非交换乘法。
\begin{align*}
    \begin{aligned}
        &\textbf{加法结合律:}    && (a+b)+c=a+(b+c) && a, b, c \in S . \\
        &\textbf{加法交换律:}    && a+b=b+a && a, b \in S . \\
        &\textbf{加法单位元0 :}  && 0+a=a+0 && a, 0 \in S . \\
        &\textbf{加法可逆性:}    && a+(-a)=(-a)+a=0 && a,-a \in S . \\
        &\textbf{乘法结合律:}    && (a \times b) \times c=a \times(b \times c) && a, b, c \in S . \\
        &\textbf{乘法单位元1:}   && 1 \times a=a \times 1 && a, 1 \in S . \\
        &\textbf{乘法可逆性:}    && a \times a^{-1}=a^{-1} \times a=1 && a, a^{-1} \in S, a \neq 0 . \\
        &\textbf{分配律:}        && a \times(b+c)=(a \times b)+(a \times c) && \text {和 } \\
        &                    &&(b+c) \times a=(b \times a)+(c \times a) && a, b, c \in S .
    \end{aligned}
\end{align*}

