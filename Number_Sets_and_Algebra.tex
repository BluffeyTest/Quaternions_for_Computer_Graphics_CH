\chap{数集与代数}
\section{介绍}
在本章中,我们将回顾数集的一些基本思想,以及如何用算术和代数方法处理它们。我们简要地看一下表达式和方程,以及用于构造和求值的规则。这些反过来又揭示了用所谓的复数扩展日常数字的需要。

本章的第二部分用于定义组、环和域。

\section{数集}
\subsection{自然数}
自然数是整数1,2,3,4等,根据定义(DIN 5473),自然数的集合和零$\{0,1,2,3,4,\ldots\}$由符号$\mathbb{N}$表示,我们使用:
$$
\mathbb{N}=\{0,1,2,3,4, \ldots\} .
$$
该语句
$$
k \in \mathbb{N}
$$
意味着$k$属于集合$\mathbb{N}$,其中$\in$表示属于,或者换句话说,$k$是一个自然数。我们在本书中使用这种符号,以确保对所使用的数值量的类型没有混淆。

$ \mathbb {N} ^{*}$用于表示集合$\{1,2,3,4,\ldots \} $。

\subsection{实数}
科学计算使用大量的数学对象,如标量、向量和矩阵。标量只有一个数值,而矢量有两个或两个以上的数字来编码矢量的大小和方向。矩阵是一个矩形数组的数字,可能有各种各样的属性。

十进制数构成由$\mathbb{R}$标识的实数集。这样的数字是有符号的,可以组织成一条线,延伸到$-\infty$和$+\infty$,其中包括0。无限的概念很奇怪,德国数学家康托(Georg Cantor, 1845-1918)研究过这个概念。康托尔还发明了集合论,证明了实数比自然数多。幸运的是,我们不需要在本书中使用这些概念。

\subsection{整数}
整数集$\mathbb{Z}$包含自然数及其负数:
$$
\mathbb{Z}=\{\ldots,-3,-2,-1,0,1,2,3, \ldots\} .
$$
$\mathbb{Z}$代表zahlen——德语“数字”的意思。


\subsection{有理数}
有理数的集合是$\mathbb{Q}$,包含如下形式的数:
$$
\frac{a}{b}
$$
其中 $a, b \in \mathbb{Z}$ 且 $b \neq 0$.

\section{算术运算}
我们使用加法、减法、乘法和除法运算来操作数字,其结果是闭的还是不闭的,或者是未定义的,这取决于底层集合。例如,当我们将两个自然数相加时,结果总是另一个自然数,因此,运算是封闭的:
$$
3+4=7 .
$$
然而,当我们减去两个自然数时,结果不一定是一个自然数。例如,尽管
$$
6-2=4
$$
是一个封闭的操作,
$$
2-6=-4
$$
不是封闭的,因为$-4$不是自然数集合中的一员。两个自然数的乘积通常是一个封闭的运算,但是除法会引起一些问题。首先,将一个偶数除以2是一个封闭运算:
$$
16 / 2=8 .
$$
然而,将一个奇数自然数除以一个偶数自然数得到一个十进制数:
$$
7 / 2=3.5
$$
并且不封闭,因为$3.5$不属于自然数集合。这是用集合语言写的
$$
3.5 \notin \mathbb{N}
$$
其中 $\notin$ 表示不属于.

将任何数字乘以零都得到零——这是一个封闭运算;然而,任何数字除以零都是没有定义的,必须排除。

实数没有任何与自然数相关的问题,并且在加法、乘法和除法上有闭包:
$$
\begin{aligned}
a+b & =c & & a, b, c \in \mathbb{R} \\
a b & =c & & a, b, c \in \mathbb{R} \\
a / b & =c & & a, b, c \in \mathbb{R} \text { and } b \neq 0 .
\end{aligned}
$$
注意,$a b$是$a \times b$的简写。


\section{公理}
当我们构造代数表达式时,我们使用称为公理的特定定律。对于加法和乘法,我们知道数字的分组对最终结果没有影响:例如$2+(4+6)=(2+4)+6$和$2 \times(3 \times 4)=$ $(2 \times 3) \times 4$。这是结合律公理,表示为:
$$
\begin{aligned}
a+(b+c) & =(a+b)+c \\
a(b c) & =(a b) c .
\end{aligned}
$$
我们也知道,当加或乘时,顺序对最终结果没有影响:例如$2+6=6+2$和$2 \times6=6 \times2$。这是交换公理,表示为:
$$
\begin{aligned}
a+b & =b+a \\
a b & =b a .
\end{aligned}
$$
代数表达式包含各种各样的乘积,涉及一个实数和一串实数,它们服从分配律:
$$
\begin{aligned}
a(b+c) & =a b+a c \\
(a+b)(c+d) & =a c+a d+b c+b d .
\end{aligned}
$$
我们回顾这些公理的原因是,它们不应该被视为刻在数学石头上的,而适用于所有被发明的东西。因为当我们讲到四元数时,我们会发现它们不遵守交换公理,这并不奇怪。如果你用过矩阵,你就会知道矩阵乘法也是不可交换的,但是是结合的。
\section{表达式}
使用上述公理,我们能够构造各种表达式,例如:
$$
\begin{aligned}
& a(2+c)-d / e+a-10 \\
& g /(a c-b d)+h /(d e-f g) .
\end{aligned}
$$
我们还使用符号来提高一个量的幂,如$n^{2}$。这个符号引入了另一组观察结果:
$$
\begin{aligned}
a^{n} a^{m} & =a^{n+m} \\
\frac{a^{n}}{a^{m}} & =a^{n-m} \\
\left(a^{n}\right)^{m} & =a^{n m} \\
\frac{a^{n}}{a^{n}} & =a^{0}=1 \\
\frac{1}{a^{n}} & =a^{-1} \\
a^{1 / n} & =\sqrt[n]{a} .
\end{aligned}
$$
接下来,我们必须包括各种各样的函数,比如平方根、正弦和余弦,这些函数看起来相当简单。但我们必须警惕他们。例如,按惯例,$\sqrt{16}=4$。但是,$x^{2}=16$有两个解:$\pm \sqrt{16}=\pm 4$。然而,$\sqrt{-16}$不存在自然数或实数解。因此,如果$a<0$,表达式$\sqrt{a}$就没有实根。

类似地,在处理正弦和余弦等三角函数时,我们必须记住,这些函数的值范围在$-1$和$+1$之间,包括0,这意味着如果将它们用作分母,结果可能是未定义的。例如,如果$\sin \alpha=0$,则此表达式未定义
$$
\frac{a}{\sin \alpha} \text {. }
$$

\section{等式}
接下来,我们来到方程,我们将表达式的值赋给变量。在大多数情况下,任务是直接的,并导致一个真实的结果,如
$$
x^{2}-16=0
$$
其中$x=\pm 4$。但有趣的是,只要把符号颠倒过来
$$
x^{2}+16=0
$$
我们创建了一个没有实解的方程。然而,有一个复杂的解决方案,这是第三章的主题。

\section{有序对}
$(a, b)$是一个有两个坐标或投影项的对象,其中第一个或左边的项与第二个或右边的项是不同的。例如,$(a, b)$与$(b, a)$是不同的,除非$a=b$。也许有序对的最好例子是$(x, y)$,它表示平面上的一个点,其中元素的顺序总是$x$-坐标后面跟着$y$-坐标。

在计算机图形学中,有序对和有序三元组被广泛用于表示平面上的点$(x, y)$,空间中的点$(x, y, z)$,以及诸如$(r, g, b)$和$(h, s, v)$等颜色值。在这些例子中,字段都是实值。没有什么可以阻止我们使用有序对来开发一个代数,它的行为就像另一个代数一样,我们将在第三章对复数和第五章对四元数这样做。现在,让我们探索一些可以操纵有序对的方法。

假设我们选择将一个通用的有序对描述为
$$
a=\left(a_{1}, a_{2}\right) \quad a_{1}, a_{2} \in \mathbb{R} .
$$
我们将定义两个对象的加法为
$$
\begin{aligned}
a & =\left(a_{1}, a_{2}\right) \\
b & =\left(b_{1}, b_{2}\right) \\
a+b & =\left(a_{1}+b_{1}, a_{2}+b_{2}\right) .
\end{aligned}
$$
举例:
$$
\begin{aligned}
a & =(2,3) \\
b & =(4,5) \\
a+b & =(6,8) .
\end{aligned}
$$
我们将乘积定义为
$$
a b=\left(a_{1} b_{1}, a_{2} b_{2}\right)
$$
使用上面的值,结果是什么
$$
a b=(8,15) .
$$
记住,我们说了算,规则是我们定的。

另一个规则将控制有序对如何响应标量乘法。例如:
$$
\begin{aligned}
\lambda\left(a_{1}, a_{2}\right) & =\left(\lambda a_{1}, \lambda a_{2}\right) \quad \lambda \in \mathbb{R} \\
3(2,3) & =(6,9) .
\end{aligned}
$$
有了上面的规则,我们就可以写了
$$
\begin{aligned}
\left(a_{1}, a_{2}\right) & =\left(a_{1}, 0\right)+\left(0, a_{2}\right) \\
& =a_{1}(1,0)+a_{2}(0,1)
\end{aligned}
$$
如果我们用乘法法则对$(1,0)$和$(0,1)$进行平方,我们得到
$$
\begin{aligned}
& (1,0)^{2}=(1,0) \\
& (0,1)^{2}=(0,1)
\end{aligned}
$$
这表明它们表现得像实数,这并不出人意料。

这似乎不是很有用,但是等着看在复数和四元数的上下文中会发生什么。

\section{群,环,域}
数学家们使用一系列令人眼花缭乱的名字来识别他们的发明,这些发明似乎每天都在出现。甚至“四元数”这个名字也不是原创的,在历史上经常出现在“士兵的四元数”的语境中:

“罗马人派出四人组或四人组成夜间警卫……”[19]。

在不太正式的情况下,让我们探索更多与本书所包含的思想相关的数学结构。

\subsection{群}
我们已经讨论了集合的概念,以及属于集合意味着什么。我们还发现,当我们对集合的成员应用某些算术运算时,我们可以确保闭包、非闭包或结果未定义。

当将集合与算术运算结合在一起时,可以方便地创建另一个实体:群,即集合,以及描述集合元素如何组合的公理。集合可能包含数字、矩阵、向量、四元数、多项式等,并在下面表示为$a,b$和$c$。

公理使用'$o$'符号来表示任何二进制操作,如$+,-, \times$。一个群是由一个集合和一个二元运算组成的。例如,我们可能希望在加法下形成一组整数:$(\mathbb{Z},+)$,或者我们可能希望检查四元数是否在乘法操作下形成一组:$(\mathbb{H}, \times)$。

要成为一个群,对于集合$S$,所有下列公理必须成立。特别地,在中必须存在一个特殊的单位元素$e \in S$,并且对于每一个$a \in S$,在中必须存在一个逆元素$a^{-1} \in S$,从而满足下列公理:

\begin{align*}
    \begin{aligned}
        & \textbf{闭合: }        && a \circ b \in S && a,b \in S.\\
        & \textbf{结合律: }  && (a \circ b) \circ c=a \circ(b \circ c) && a, b, c \in S.\\
        & \textbf{单位元:}    && a \circ e=e \circ a=a && a, e \in S.\\
        & \textbf{可逆元素: }    && a  \circ a^{-1}=a^{-1} \circ a=e && a, a^{-1}, e \in S.
    \end{aligned}
\end{align*}
我们将一个组描述为$(S, \circ)$,其中$S$是集合,' 0 '是操作。例如$(\mathbb{Z},+)$是一组进行加法运算的整数,$(\mathbb{R}, \times)$是一组进行乘法运算的实数。

让我们通过三个例子将这些公理生动地展现出来。

$(\mathbb{Z},+)$:整数$\mathbb{Z}$在加法运算下形成一个组:
$$
\begin{aligned}
\text { 闭合: } & -23+24=1 \\
\text { 结合律: } & (2+3)+4=2+(3+4)=9 \\
\text { 单位元: } & 2+0=0+2=2 \\
\text { 可逆: } & 2+(-2)=(-2)+2=0 .
\end{aligned}
$$
$(\mathbb{Z}, \times)$:整数$\mathbb{Z}$在乘法下不构成群:
$$
\begin{aligned}
\text { 闭合: } & -2 \times 4=-8 \\
\text { 结合律: } & (2 \times 3) \times 4=2 \times(3 \times 4)=24 \\
\text { 单位元: } & 2 \times 1=1 \times 2=2 \\
\text { 可逆性: } & 2^{-1}=0.5 \quad(0.5 \notin \mathbb{Z}) .
\end{aligned}
$$
并且,整数0没有逆。

$(\mathbb{Q}, \times)$:非零有理数在乘法下构成一个群:
$$
\begin{aligned}
\text { 闭合: } & \frac{2}{5} \times \frac{2}{3}=\frac{4}{15} \\
\text { 交换律: } & \left(\frac{2}{5} \times \frac{2}{3}\right) \times \frac{1}{2}=\frac{2}{5} \times\left(\frac{2}{3} \times \frac{1}{2}\right)=\frac{2}{15} \\
\text { 单位元: } & \frac{2}{3} \times \frac{1}{1}=\frac{1}{1} \times \frac{2}{3}=\frac{2}{3} \\
\text { 可逆性: } & \frac{2}{3} \times \frac{3}{2}=\frac{1}{1} \quad\left(\text { where } \frac{3}{2}=\left(\frac{2}{3}\right)^{-1}\right)
\end{aligned}
$$

\subsection{阿贝尔群}
最后,以挪威数学家Neils Henrik Abel(1802-1829)的名字命名的阿贝尔群,是一个元素的顺序不影响结果的群,即群是可交换的。因此有五个公理:闭合性、结合性、单位元、逆元和交换性:
$$
\textbf{Commutativity:} \qquad a \circ b=b \circ a \qquad a, b \in S
$$
例如,整数集在普通加法$(\mathbb{Z},+)$下形成一个阿贝尔群。然而,由于三维旋转一般不交换,三维空间中所有旋转的集合形成了一个非交换群。

\subsection{环}
环是一个扩展的群,在这里我们有一组可以相加和相乘的对象,遵循一些精确的公理。有实数圈、复数圈、整数圈、矩阵圈、方程圈、多项式圈等。环被正式定义为一个系统,其中$(S,+)$和$(S, \times)$为阿贝尔群,分配公理如下:
$$
\begin{array}{lll}
\textbf { 加法结合律: } & a+(b+c)=(a+b)+c & a, b, c \in S . \\
\textbf { 乘法结合律: } & a \times(b \times c)=(a \times b) \times c & a, b, c \in S . \\
\textbf { 分配律: } & a \times(b+c)=(a \times b)+(a \times c) & \text { 和 } \\
& (a+b) \times c=(a \times c)+(b \times c) & a, b, c \in S .
\end{array}
$$
例如,我们已经知道整数$\mathbb{Z}$在加法运算下形成了一个群,但它们也形成了一个环,因为集合满足上述公理:
$$
\begin{aligned}
& 2 \times(3 \times 4)=(2 \times 3) \times 4 \\
& 2 \times(3+4)=(2 \times 3)+(2 \times 4) \\
& (2+3) \times 4=(2 \times 4)+(3 \times 4) .
\end{aligned}
$$

\subsection{域}
虽然环支持加法和乘法,但它们不一定支持除法。然而,由于除法是如此重要的算术运算,因此创建域是为了支持除法,但有一个限制条件:不允许除0。这样我们就有了实数$\mathbb{R}$、有理数$\mathbb{Q}$以及复数$\mathbb{C}$的域。然而,我们将发现四元数不构成域,但它们确实形成了所谓的除法环。

因此,每个域都是一个环,但并不是每个环都是一个场。

\subsection{除法环}
除法环或除法代数是一个环,其中每个元素都有一个逆元素,前提是该元素非零。代数也支持非交换乘法。下面是除法环$(S,+, \times)$的正式描述:

Additive associativity: $\quad(a+b)+c=a+(b+c)$

$a, b, c \in S$

Additive commutativity:

$a+b=b+a$

$a, b \in S$.

Additive identity 0 :

$0+a=a+0$

$a, 0 \in S$.

Additive inverse:

$a+(-a)=(-a)+a=0$

$a,-a \in S$.

Multiplicative associativity: $(a \times b) \times c=a \times(b \times c)$

$a, b, c \in S$.

Multiplicative identity 1:

$1 \times a=a \times 1$

$a, 1 \in S$.

Multiplicative inverse:

$a \times a^{-1}=a^{-1} \times a=1$

$a, a^{-1} \in S, a \neq 0$.

Distributivity:

$a \times(b+c)=(a \times b)+(a \times c)$ and

$(b+c) \times a=(b \times a)+(c \times a) \quad a, b, c \in S$.

In 1878 the German mathematician, Ferdinand Georg Frobenius (1849-1917), proved that there are only three associative division algebras: real numbers $\mathbb{R}$, complex numbers $\mathbb{C}$, and quaternions $\mathbb{H}$.

\section{Summary}
The objective of this chapter was to remind you of the axiomatic systems underlying algebra, and how the results of arithmetic operations can be open, closed, or undefined. Perhaps some of the ideas of ordered pairs, sets, groups, fields and rings are new, and they have been included as this notation is often used in association with quaternions.

All of these ideas emerge again when we consider the algebra of complex numbers and later on, quaternions.

\subsubsection{Summary of Definitions}
\section{Ordered pair}
An object with two distinguishable components: $(a, b)$ such that $(a, b) \neq(b, a)$ unless $a=b$.

\section{Set}
Definition: A set is a collection of objects.

Notation: $k \in \mathbb{Z}$ means $k$ belongs to the set $\mathbb{Z}$.
$\mathbb{C}$ : Set of complex numbers
$\mathbb{H}$ : Set of quaternions
$\mathbb{N}$ : Set of natural numbers
$\mathbb{Q}$ : Set of rational numbers
$\mathbb{R}:$ Set of real numbers
$\mathbb{Z}$ : Set of integers.

\section{Group}
Definition: A group $(S, \circ)$ is a set $S$ and a binary operation ' $\circ$ ' and the axioms defining closure, associativity, an identity element, and an inverse element.
Closure:
$a \circ b \in S$
$a, b \in S$.
Associativity:
$(a \circ b) \circ c=a \circ(b \circ c)$
$a, b, c \in S$.
Identity element:
$a \circ e=e \circ a=a$
$a, e \in S$.
Inverse element:
$a \circ a^{-1}=a^{-1} \circ a=e$.
$a, a^{-1}, e \in S$.

\section{Ring}
Definition: A ring is a group whose elements can be added/subtracted and multiplied, using some precise axioms:
Additive associativity:
$a+(b+c)=(a+b)+c$
$a, b, c \in S$.
Multiplicative associativity: $a \times(b \times c)=(a \times b) \times c$
$a, b, c \in S$.
Distributivity:
$a \times(b+c)=(a \times b)+(a \times c)$ and
$(a+b) \times c=(a \times c)+(b \times c) \quad a, b, c \in S$.

\section{Field}
Definition: A field is a ring that supports division.

\section{Division ring}
Every element of a division ring has an inverse element, with the proviso that the element is non-zero. The algebra also supports non-commutative multiplication.

Additive associativity:

Additive commutativity:

Additive identity 0 :

Additive inverse:

Multiplicative associativity: $(a \times b) \times c=a \times(b \times c)$

Multiplicative identity 1: $\quad 1 \times a=a \times 1$

Multiplicative inverse:

Distributivity:

$$
\begin{array}{ll}
(a+b)+c=a+(b+c) & a, b, c \in S . \\
a+b=b+a & a, b \in S . \\
0+a=a+0 & a, 0 \in S . \\
a+(-a)=(-a)+a=0 & a,-a \in S . \\
(a \times b) \times c=a \times(b \times c) & a, b, c \in S . \\
1 \times a=a \times 1 & a, 1 \in S . \\
a \times a^{-1}=a^{-1} \times a=1 & a, a^{-1} \in S, a \neq 0 . \\
a \times(b+c)=(a \times b)+(a \times c) & \text { and } \\
(b+c) \times a=(b \times a)+(c \times a) & a, b, c \in S .
\end{array}
$$
