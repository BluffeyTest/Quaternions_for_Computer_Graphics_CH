\chap{空间中的四元数}
\section{介绍}
在本章中,我们将展示如何使用四元数来围绕任意轴旋转向量。我们首先回顾了一些与四元数相关的历史,特别是本杰明·奥林德·罗德里格斯(Benjamin Olinde Rodrigues)的作用,他发现了旋转变换中半角的重要性。

对于特定的四元数积,当四元数表示为
$$
q=[\cos \theta, \sin \theta \mathbf{v}]
$$

一个向量围绕轴$\mathbf{v}$旋转一个角度$\theta$。但是,我们会发现,对于三重四元数的乘积,当四元数表示为
$$
q=\left[\cos \frac{1}{2} \theta, \sin \frac{1}{2} \theta \mathbf{v}\right]
$$

一个向量围绕轴$\mathbf{v}$旋转一个角度$\theta$。这种半角表示是罗德里格斯发现的。

关于组合代数的简短部分揭示了四元数是相当特殊的,并告诉我们为什么 Hamilton 不能识别基于超复数$z=s+ i+ bj $的代数。

然后,我们检查了各种四元数乘积,以发现它们的旋转性质。这从两个正交四元数开始,然后转向使用三重$q p q^{-1}$的一般情况,其中$q$是一个单位范数四元数,$p$是一个纯四元数。

本章介绍了两种将四元数乘积表示为矩阵的方法,这两种方法依次编码特征向量和特征值。最后,我们研究如何插值四元数。

我们继续将四元数表示为有序对,用斜体小写字母表示四元数,用粗体小写字母表示向量。

\section{一点历史}
本杰明·罗德里格斯(Benjamin Olinde Rodrigues, 1795-1851)出生于法国波尔多。他在巴黎学习,并于1816年获得博士学位,时年21岁。他论文的主题是求解 Legendre 多项式, Rodrigues 提出了一个解,这个解至今仍被称为 Rodrigues 公式。

虽然他从事政治和银行业,但他的博士研究证实他不仅仅是一个“娱乐”数学家,因为在1840年,他在《纯数学年鉴》(Annales de Mathématiques Pures et Appliquées)上发表了一篇关于变换群[20]的数学论文。该文包含了一个公式,其描述了一个几何结构,即两个连续的围绕不同的轴的旋转,等价于第三个围绕另一个轴的旋转。今天,我们知道这种对应称为Euler-Rodrigues参数化。欧拉在1775年已经证明了一次旋转可以表示围绕不同轴的两次连续旋转,但没有提供代数解。

如果我们将一个关于轴向量$\mathbf{a}$的旋转$\alpha$表示为$\mathbf{R}_{\alpha, \mathbf{a}}$,那么Rodrigues提供了一个解决方案
$$
\mathbf{R}_{\gamma, \mathbf{n}}=\mathbf{R}_{\alpha, \mathbf{l}} \mathbf{R}_{\beta, \mathbf{m}}
$$

形式为
\begin{align}
& \cos \frac{1}{2} \gamma=\cos \frac{1}{2} \alpha \cos \frac{1}{2} \beta-\sin \frac{1}{2} \alpha \sin \frac{1}{2} \beta \mathbf{l} \cdot \mathbf{m} \\
& \sin \frac{1}{2} \gamma \mathbf{n}=\sin \frac{1}{2} \alpha \cos \frac{1}{2} \beta \mathbf{l}+\cos \frac{1}{2} \alpha \sin \frac{1}{2} \beta \mathbf{m}+\sin \frac{1}{2} \alpha \sin \frac{1}{2} \beta \mathbf{l} \times \mathbf{m} .
\end{align}

Rodrigues没有使用(7.1)和(7.2)中使用的向量符号,因为这还没有被Hamilton定义,但他确实使用了这些向量积的代数等价物。图$7.1$显示了罗德里格斯使用的由轴和旋转角组成的球形三角形。

\begin{figure}[h!]
    \centering
    \includegraphics[max width=0.5\textwidth]{2023_01_16_a848224efad29cd66460g-105}
    \caption[short]{显示$\mathbf{l}$、$\mathbf{m}$和$\mathbf{n}$的Rodrigues球面三角形}
\end{figure}

式(7.1)和式(7.2)包含了一些四元数乘积所熟悉的特征,这些特征在下面的分析中变得很明显。我们从定义四元数开始

$$
\begin{aligned}
q_{l} & =\left[\cos \frac{1}{2} \alpha, \sin \frac{1}{2} \alpha \mathbf{l}\right] \\
q_{m} & =\left[\cos \frac{1}{2} \beta, \sin \frac{1}{2} \beta \mathbf{m}\right] \\
q_{n} & =\left[\cos \frac{1}{2} \gamma, \sin \frac{1}{2} \gamma \mathbf{n}\right]
\end{aligned}
$$

且乘积形式为
\begin{align}
q_{n}  =& q_{l} q_{m} \notag\\
=& \left[\cos \frac{1}{2} \alpha, \sin \frac{1}{2} \alpha \mathbf{l}\right]\left[\cos \frac{1}{2} \beta, \sin \frac{1}{2} \beta \mathbf{m}\right] \notag\\
=& \left[\cos \frac{1}{2} \alpha \cos \frac{1}{2} \beta-\sin \frac{1}{2} \alpha \sin \frac{1}{2} \beta \mathbf{l} \cdot \mathbf{m},\right. \notag\\
& \left.\sin \frac{1}{2} \alpha \cos \frac{1}{2} \beta \mathbf{l}+\cos \frac{1}{2} \alpha \sin \frac{1}{2} \beta \mathbf{m}+\sin \frac{1}{2} \alpha \sin \frac{1}{2} \beta \mathbf{l} \times \mathbf{m}\right] \notag\\
\cos \frac{1}{2} \gamma = & \cos \frac{1}{2} \alpha \cos \frac{1}{2} \beta-\sin \frac{1}{2} \alpha \sin \frac{1}{2} \beta \mathbf{l} \cdot \mathbf{m} \\
\sin \frac{1}{2} \gamma \mathbf{n} =& \sin \frac{1}{2} \alpha \cos \frac{1}{2} \beta \mathbf{l}+\cos \frac{1}{2} \alpha \sin \frac{1}{2} \beta \mathbf{m}+\sin \frac{1}{2} \alpha \sin \frac{1}{2} \beta \mathbf{l} \times \mathbf{m}
\end{align}

其中(7.3)和(7.4)分别与(7.1)和(7.2)相同。虽然Rodrigues没有发明四元数的形式
$$
q=s+a i+b j+c k,
$$

他在 Hamilton 之前就发现了四元数积的系数。这就是生活!\footnote{译注,法语:C'est la vie!}

Hamilton 在1843年10月发明了四元数,同年12月,他的朋友、爱尔兰数学家约翰·托马斯·格雷夫斯(John Thomas Graves, 1806-1870)发明了八度音阶,最终被称为八元数。英国数学家亚瑟·凯利(Arthur Cayley, 1821-1895)也对 Hamilton 的四元数很感兴趣,并于1845年独立发明了八元数。八元数最终被称为Cayley数而不是八元数,只是因为Graves直到1848年才发表了他的结果——比Cayley晚了3年!

正如四元数可以用复数的有序对来定义一样,八度或八元数也可以用四元数的有序对来定义。

\subsection{组合代数}
当一个特定的定律构成一个代数的基础时,它被称为组合代数。例如,我们知道在普通算术中
$$
a^{2} b^{2}=(a b)^{2} \qquad a, b \in \mathbb{R}
$$
比如
$$
3^{2} 4^{2}=12^{2}
$$
其中平方定律就是组合定律。

我们在第四章中发现,对于两个复数:
$$
\begin{aligned}
\left|z_{1}\right|\left|z_{2}\right| & =\left|z_{1} z_{2}\right| \qquad z_{1}, z_{2} \in \mathbb{C} \\
\left|z_{1}\right|^{2}\left|z_{2}\right|^{2} & =\left|z_{1} z_{2}\right|^{2} .
\end{aligned}
$$
举个例子, 给出
$$
\begin{aligned}
& z_{1}=a_{1}+b_{1} i \\
& z_{2}=a_{2}+b_{2} i
\end{aligned}
$$
然后
$$
\left(a_{1}^{2}+b_{1}^{2}\right)\left(a_{2}^{2}+b_{2}^{2}\right)=\left(a_{1} a_{2}-b_{1} b_{2}\right)^{2}+\left(a_{1} b_{2}+a_{2} b_{1}\right)^{2}
$$
这是一个2平方定律。

在第5章中,我们注意到对于两个四元数:
$$
\left|q_{a}\right|^{2}\left|q_{b}\right|^{2}=\left|q_{a} q_{b}\right|^{2} \quad q_{a}, q_{b} \in \mathbb{H} .
$$
举个例子, 给出
$$
\begin{aligned}
q_{a} & =\left[s_{a}, x_{a} \mathbf{i}+y_{a} \mathbf{j}+z_{a} \mathbf{k}\right] \\
q_{b} & =\left[s_{b}, x_{b} \mathbf{i}+y_{b} \mathbf{j}+z_{b} \mathbf{k}\right]
\end{aligned}
$$
然后
$$
\begin{aligned}
\left(s_{a}^{2}+x_{a}^{2}+y_{a}^{2}+z_{a}^{2}\right)\left(s_{b}^{2}+x_{b}^{2}+y_{b}^{2}+z_{b}^{2}\right)= & \left(s_{a} s_{b}-x_{a} x_{b}-y_{a} y_{b}-z_{a} z_{b}\right)^{2} \\
& +\left(s_{a} x_{b}+s_{b} x_{a}+y_{a} z_{b}-y_{b} z_{a}\right)^{2} \\
& +\left(s_{a} y_{b}+s_{b} y_{a}+z_{a} x_{b}-z_{b} x_{a}\right)^{2} \\
& +\left(s_{a} z_{b}+s_{b} z_{a}+x_{a} y_{b}-x_{b} y_{a}\right)^{2}
\end{aligned}
$$
这是四方定律。

除了复数,四元数在数学系统中占据着中心位置,今天有四个这样的组合代数:实$\mathbb{R}$、复$\mathbb{C}$、四元数$\mathbb{H}$和八元数$\mathbb{O}$,它们遵循$n$-平方的恒等式来计算它们的规范。1898年,德国数学家阿道夫·赫维茨(Adolf Hurwitz, 1859-1919)证明,只有当$n$等于1、2、4和8时,$n$的平方和与$n$的平方和的乘积才等于$n$的平方和,其中$n$用实数、复数、四元数和八元数表示。这就是所谓的“赫维茨定理”或“1,2,4,8定理”。没有其他系统是可能的,这表明四元数在数学领域是多么重要。因此,Hamilton对三元体系的探索是徒劳的,因为不存在三平方恒等式。

现在让我们研究如何使用四元数来围绕任意轴旋转向量。

\section{四元数乘积}
四元数$q$是标量$s$和向量$\mathbf{v}$的并集:
$$
q=[s, \mathbf{v}] \quad s \in \mathbb{R}, \mathbf{v} \in \mathbb{R}^{3}
$$

如果我们用$\mathbf{v}$的分量来表示它,我们有
$$
q=[s, x \mathbf{i}+y \mathbf{j}+z \mathbf{k}] \quad s, x, y, z \in \mathbb{R} .
$$

Hamilton曾希望四元数可以像复数转子一样使用,后者我们在第二章中看到了
$$
\mathbf{R}_{\theta}=\cos \theta+i \sin \theta
$$

将一个复数旋转$\theta$。单位范数四元数$q$可以用来旋转存储为纯四元数$p$的向量吗?是的,但只是在有限的意义上。为了理解这一点,让我们构造一个单位范数四元数$q$和一个纯四元数$p$的乘积。单位范数四元数$q$定义为
\begin{align}
\begin{aligned}
q & =[s, \lambda \hat{\mathbf{v}}] \quad s, \lambda \in \mathbb{R}, \hat{\mathbf{v}} \in \mathbb{R}^{3} \\
|\hat{\mathbf{v}}| & =1 \\
s^{2}+\lambda^{2} & =1
\end{aligned}
\end{align}
纯四元数$p$存储要旋转的向量$\mathbf{p}$:
$$
p=[0, \mathbf{p}] \quad \mathbf{p} \in \mathbb{R}^{3} .
$$

让我们计算乘积$p^{\prime}=q p$,并检查$p^{\prime}$的向量部分,看看$ \mathbf{p}$是否被旋转:
\begin{align}
\begin{aligned}
p^{\prime} & =q p \\
& =[s, \lambda \hat{\mathbf{v}}][0, \mathbf{p}] \\
& =[-\lambda \hat{\mathbf{v}} \cdot \mathbf{p}, s \mathbf{p}+\lambda \hat{\mathbf{v}} \times \mathbf{p}] .
\end{aligned}
\end{align}

我们可以从(7.6)中看到,结果是一个具有标量和向量分量的一般四元数。

\subsection{特殊情况}
上面提到的“狭义”是$\hat{\mathbf{v}}$必须垂直于$\mathbf{p}$,这使得点积项$-\lambda \hat{\mathbf{v}} \cdot \mathbf{p}$在(7.6)中消失,我们只剩下纯四元数
\begin{align}
    p^{\prime}=[0, s \mathbf{p}+\lambda \hat{\mathbf{v}} \times \mathbf{p}] .
\end{align}

图$7.2$说明了这种情况,其中$\mathbf{p}$垂直于$\hat{\mathbf{v}}$,而$\hat{\mathbf{v}} \times \mathbf{p}$垂直于包含$\mathbf{p}$和$\hat{\mathbf{v}}$的平面。现在因为$\hat{\mathbf{v}}$是一个单位向量,$|\mathbf{p}|=|\hat{\mathbf{v}} \times\mathbf{p}|$,这意味着我们有两个正交向量,即$\mathbf{p}$和$\hat{\mathbf{v}} \times\mathbf{p}$,它们的长度相同。因此,要绕$\hat{\mathbf{v}}$旋转$\mathbf{p}$,我们所要做的就是在(7.7)中使$s=\cos \theta$和$ \lambda=\sin \theta$:
$$
\begin{aligned}
p^{\prime} & =\left[0, \mathbf{p}^{\prime}\right] \\
& =[0, \cos \theta \mathbf{p}+\sin \theta \hat{\mathbf{v}} \times \mathbf{p}]
\end{aligned}
$$

\begin{figure}[h!]
    \includegraphics[max width=0.5\textwidth, center]{2023_01_16_a848224efad29cd66460g-109}
    \caption[short]{三个正交向量$\mathbf{p}, \hat{\mathbf{v}}$和$\hat{\mathbf{v}} \times \mathbf{p}$}
\end{figure}
\begin{figure}[h!]
    \includegraphics[max width=0.5\textwidth, center]{2023_01_16_a848224efad29cd66460g-109(1)}
    \caption[short]{向量$2 \mathbf{i}$被四元数$q=\left[\frac{\sqrt{2}}{2}, \frac{\sqrt{2}}{2} \mathbf{k}\right]$旋转$45^{\circ}$}
\end{figure}

例如,要绕$z$-轴旋转一个向量,$q$的向量$\hat{\mathbf{v}}$必须与$z$-轴对齐,如图7.3所示。如果我们使旋转角度$\theta=45^{\circ}$,那么
$$
\begin{aligned}
q & =[s, \lambda \hat{\mathbf{v}}] \\
& =[\cos \theta, \sin \theta \mathbf{k}] \\
& =\left[\frac{\sqrt{2}}{2}, \frac{\sqrt{2}}{2} \mathbf{k}\right]
\end{aligned}
$$
如果要旋转的向量是$\mathbf{p}=2 \mathbf{i}$,则
$$
\begin{aligned}
p & =[0, \mathbf{p}] \\
& =[0,2 \mathbf{i}] .
\end{aligned}
$$

现在有四个值得探索的乘积组合:$q p, pq, q^{-1} p$和$p q^{-1}$。不值得考虑$q p^{-1}$和$p^{-1} q$,因为$p^{-1}$只是颠倒了$\mathbf{p}$的方向。让我们从$q p$开始:
$$
\begin{aligned}
p^{\prime} & =q p \\
& =\left[\frac{\sqrt{2}}{2}, \frac{\sqrt{2}}{2} \mathbf{k}\right][0,2 \mathbf{i}] \\
& =\left[0,2 \frac{\sqrt{2}}{2} \mathbf{i}+2 \frac{\sqrt{2}}{2} \mathbf{k} \times \mathbf{i}\right] \\
& =[0, \sqrt{2} \mathbf{i}+\sqrt{2} \mathbf{j}]
\end{aligned}
$$
即$\mathbf{p}$已经被旋转$45^{\circ}$为$\mathbf{p}^{\prime}=\sqrt{2} \mathbf{i}+\sqrt{2} \mathbf{j}$。

接下来, $p q$ :
$$
\begin{aligned}
p^{\prime} & =p q \\
& =[0,2 \mathbf{i}]\left[\frac{\sqrt{2}}{2}, \frac{\sqrt{2}}{2} \mathbf{k}\right] \\
& =\left[0,2 \frac{\sqrt{2}}{2} \mathbf{i}-2 \frac{\sqrt{2}}{2} \mathbf{k} \times \mathbf{i}\right] \\
& =[0, \sqrt{2} \mathbf{i}-\sqrt{2} \mathbf{j}]
\end{aligned}
$$

即 $\mathbf{p}$ 已经被旋转 $-45^{\circ}$ 到 $\mathbf{p}^{\prime}=\sqrt{2} \mathbf{i}-\sqrt{2} \mathbf{j}$.

接下来,$q^{-1} p$,由于$q$是单位范数四元数,$q^{-1}=q^{*}$:
$$
\begin{aligned}
p^{\prime} & =q^{-1} p \\
& =\left[\frac{\sqrt{2}}{2},-\frac{\sqrt{2}}{2} \mathbf{k}\right][0,2 \mathbf{i}] \\
& =\left[0,2 \frac{\sqrt{2}}{2} \mathbf{i}-2 \frac{\sqrt{2}}{2} \mathbf{k} \times \mathbf{i}\right] \\
& =[0, \sqrt{2} \mathbf{i}-\sqrt{2} \mathbf{j}]
\end{aligned}
$$
即 $\mathbf{p}$ 已经被旋转 $-45^{\circ}$ 到 $\mathbf{p}^{\prime}=\sqrt{2} \mathbf{i}-\sqrt{2} \mathbf{j}$.

最后, $p q^{-1}$ :
$$
\begin{aligned}
p^{\prime} & =p q^{-1} \\
& =[0,2 \mathbf{i}]\left[\frac{\sqrt{2}}{2},-\frac{\sqrt{2}}{2} \mathbf{k}\right] \\
& =\left[0,2 \frac{\sqrt{2}}{2} \mathbf{i}+2 \frac{\sqrt{2}}{2} \mathbf{k} \times \mathbf{i}\right] \\
& =[0, \sqrt{2} \mathbf{i}+\sqrt{2} \mathbf{j}]
\end{aligned}
$$

即 $\mathbf{p}$ 已经被旋转 $45^{\circ}$ 到 $\mathbf{p}^{\prime}=\sqrt{2} \mathbf{i}+\sqrt{2} \mathbf{j}$. 因此,对于正交四元数,$\theta$是旋转角度,则
$$
\begin{aligned}
& q p=p q^{-1} \\
& p q=q^{-1} p
\end{aligned}
$$

\begin{figure}[h!]
    \centering
    \includegraphics[max width=0.5\textwidth]{2023_01_16_a848224efad29cd66460g-111}
    \caption[short]{将向量$\mathbf{p}=2 \mathbf{i}$按四元数$q=[\cos \theta, \sin \theta \hat{\mathbf{v}}]$旋转}
\end{figure}

在继续之前,让我们看看当$\theta=180^{\circ}$时,乘积$q p$会发生什么变化:
$$
\begin{aligned}
p^{\prime} & =q p \\
& =[-1, \mathbf{0}][0,2 \mathbf{i}] \\
& =[0,-2 \mathbf{i}]
\end{aligned}
$$
即$\mathbf{p}$已被旋转$180^{\circ}$到$\mathbf{p}^{\prime}=-2 \mathbf{i}$。

注意,在上述所有乘积中,矢量在旋转过程中都没有缩放。这是因为$q$是一个单位范数四元数。现在让我们看看如果我们改变$\hat{\mathbf{v}}$和$\mathbf{p}$之间的角度会发生什么。让我们将角度减小到$45^{\circ}$,并保留$q$的单位向量,如图7.4所示。因此,
$$
\begin{aligned}
\hat{\mathbf{v}} & =\frac{1}{\sqrt{2}} \mathbf{i}+\frac{1}{\sqrt{2}} \mathbf{k} \\
q & =[\cos \theta, \sin \theta \hat{\mathbf{v}}] \\
p & =[0, \mathbf{p}] .
\end{aligned}
$$
这一次我们必须包含点积项$-\sin \theta \hat{\mathbf{v}} \cdot \mathbf{p}$,因为它不再是零:
\begin{align}
    \begin{aligned}
        p^{\prime} & =q p \\
        & =[\cos \theta, \sin \theta \hat{\mathbf{v}}][0, \mathbf{p}] \\
        & =[-\sin \theta \hat{\mathbf{v}} \cdot \mathbf{p}, \cos \theta \mathbf{p}+\sin \theta \hat{\mathbf{v}} \times \mathbf{p}] .
    \end{aligned}
\end{align}


将$\hat{\mathbf{v}}, \mathbf{p}$和$\theta=45^{\circ}$代入$(7.8)$,我们有
\begin{align}
    \begin{aligned}
        p^{\prime} & =\left[-\frac{\sqrt{2}}{2}\left(\frac{1}{\sqrt{2}} \mathbf{i}+\frac{1}{\sqrt{2}} \mathbf{k}\right) \cdot(2 \mathbf{i}), \frac{\sqrt{2}}{2} 2 \mathbf{i}+\frac{\sqrt{2}}{2}\left(\frac{1}{\sqrt{2}} \mathbf{i}+\frac{1}{\sqrt{2}} \mathbf{k}\right) \times 2 \mathbf{i}\right] \\
        & =[-1, \sqrt{2} \mathbf{i}+\mathbf{j}]
        \end{aligned}
\end{align}


不幸的是,它不再是纯四元数了。它没有被旋转$45^{\circ}$,并且向量的范数减少为$\sqrt{3}$ !向量乘以一个非正交四元数已将一些向量信息转换为四元数的标量分量。

\begin{figure}[h!]
    \centering
    \includegraphics[max width=0.5\textwidth]{2023_01_16_a848224efad29cd66460g-112}
    \caption[short]{向量$2 \mathbf{i}$被旋转$90^{\circ}$为$\mathbf{i}+\sqrt{2} \mathbf{j}+\mathbf{k}$}
\end{figure}

\subsection{一般情况}
别担心。会不会是四元数逆运算用反了?让我们看看如果用$q^{-1}$后乘$q p$会发生什么。

给出
$$
q=\left[\cos \theta, \sin \theta\left(\frac{1}{\sqrt{2}} \mathbf{i}+\frac{1}{\sqrt{2}} \mathbf{k}\right)\right]
$$
接着
$$
\begin{aligned}
q^{-1} & =\left[\cos \theta,-\sin \theta\left(\frac{1}{\sqrt{2}} \mathbf{i}+\frac{1}{\sqrt{2}} \mathbf{k}\right)\right] \\
& =\left[\frac{\sqrt{2}}{2}, \frac{\sqrt{2}}{2}\left(\frac{1}{\sqrt{2}} \mathbf{i}+\frac{1}{\sqrt{2}} \mathbf{k}\right)\right] \\
& =\frac{1}{2}[\sqrt{2},-\mathbf{i}-\mathbf{k}] .
\end{aligned}
$$
因此,将(7.9)乘以$q^{-1}$,得到
\begin{align}
    \begin{aligned}
        q p q^{-1} & =[-1, \sqrt{2} \mathbf{i}+\mathbf{j}] \frac{1}{2}[\sqrt{2},-\mathbf{i}-\mathbf{k}] \\
        & =\frac{1}{2}[-\sqrt{2}-(\sqrt{2} \mathbf{i}+\mathbf{j}) \cdot(-\mathbf{i}-\mathbf{k}), \mathbf{i}+\mathbf{k}+\sqrt{2}(\sqrt{2} \mathbf{i}+\mathbf{j})-\mathbf{i}+\sqrt{2} \mathbf{j}+\mathbf{k}] \\
        & =\frac{1}{2}[-\sqrt{2}+\sqrt{2}, \mathbf{i}+\mathbf{k}+2 \mathbf{i}+\sqrt{2} \mathbf{j}-\mathbf{i}+\sqrt{2} \mathbf{j}+\mathbf{k}] \\
        & =[0, \mathbf{i}+\sqrt{2} \mathbf{j}+\mathbf{k}]
    \end{aligned}
\end{align}
这是一个纯四元数。此外,没有缩放,因为它的范数仍然是2,但向量已经旋转了$90^{\circ}$,而不是$45^{\circ}$,是所需角度的两倍,如图7.5所示。

如果以$q$和$q^{-1}$的纯四元数形式“夹”向量是正确的,那么将$\theta$增加到$90^{\circ}$应该将$ \mathbf{p}=2 \mathbf{i}$旋转$180^{\circ}$到$2 \mathbf{k}$。让我们试试这个。

令$\theta=90^{\circ}$,因此,
$$
\begin{aligned}
q p & =\left[0, \frac{1}{\sqrt{2}} \mathbf{i}+\frac{1}{\sqrt{2}} \mathbf{k}\right][0,2 \mathbf{i}] \\
& =\left[-\frac{2}{\sqrt{2}}, \frac{2}{\sqrt{2}} \mathbf{j}\right]
\end{aligned}
$$
接下来,我们将$q p$后乘$q^{-1}$:
$$
\begin{aligned}
q u p q^{-1} & =\left[-\frac{2}{\sqrt{2}}, \frac{2}{\sqrt{2}} \mathbf{j}\right]\left[0,-\frac{1}{\sqrt{2}} \mathbf{i}-\frac{1}{\sqrt{2}} \mathbf{k}\right] \\
& =[0, \mathbf{i}+\mathbf{k}-\mathbf{i}+\mathbf{k}] \\
& =[0,2 \mathbf{k}]
\end{aligned}
$$
这证实了我们的预测,并表明$q p q^{-1}$可行。现在我们来看看这个两倍角是如何产生的。首先定义一个单位范数四元数$q$:
$$
q=[s, \lambda \hat{\mathbf{v}}]
$$
其中$s^{2}+\lambda^{2}=1$。要旋转的向量$\mathbf{p}$被编码为纯四元数:
$$
p=[0, \mathbf{p}]
$$
且这个四元数的逆$q^{-1}$是
$$
q^{-1}=[s,-\lambda \hat{\mathbf{v}}]
$$
因此,积$q p q^{-1}$为
$$
\begin{aligned}
q p q^{-1}= & {[s, \lambda \hat{\mathbf{v}}][0, \mathbf{p}][s,-\lambda \hat{\mathbf{v}}] } \\
= & {[-\lambda \hat{\mathbf{v}} \cdot \mathbf{p}, s \mathbf{p}+\lambda \hat{\mathbf{v}} \times \mathbf{p}][s,-\lambda \hat{\mathbf{v}}] } \\
= & {\left[-\lambda s \hat{\mathbf{v}} \cdot \mathbf{p}+\lambda s \mathbf{p} \cdot \hat{\mathbf{v}}+\lambda^{2}(\hat{\mathbf{v}} \times \mathbf{p}) \cdot \hat{\mathbf{v}}\right.} \\
& \lambda^{2}(\hat{\mathbf{v}} \cdot \mathbf{p}) \hat{\mathbf{v}}+s^{2} \mathbf{p}+\lambda s \hat{\mathbf{v}} \times \mathbf{p} \\
& \left.-\lambda s \mathbf{p} \times \hat{\mathbf{v}}-\lambda^{2}(\hat{\mathbf{v}} \times \mathbf{p}) \times \hat{\mathbf{v}}\right] \\
= & {\left[\lambda^{2}(\hat{\mathbf{v}} \times \mathbf{p}) \cdot \hat{\mathbf{v}}, \lambda^{2}(\hat{\mathbf{v}} \cdot \mathbf{p}) \hat{\mathbf{v}}+s^{2} \mathbf{p}+2 \lambda s \hat{\mathbf{v}} \times \mathbf{p}-\lambda^{2}(\hat{\mathbf{v}} \times \mathbf{p}) \times \hat{\mathbf{v}}\right] }
\end{aligned}
$$
请注意,
$$
(\hat{\mathbf{v}} \times \mathbf{p}) \cdot \hat{\mathbf{v}}=0
$$
且
$$
(\hat{\mathbf{v}} \times \mathbf{p}) \times \hat{\mathbf{v}}=(\hat{\mathbf{v}} \cdot \hat{\mathbf{v}}) \mathbf{p}-(\mathbf{p} \cdot \hat{\mathbf{v}}) \hat{\mathbf{v}}=\mathbf{p}-(\mathbf{p} \cdot \hat{\mathbf{v}}) \hat{\mathbf{v}}
$$
因此,
\begin{align}
\begin{aligned}
q p q^{-1} & =\left[0, \lambda^{2}(\hat{\mathbf{v}} \cdot \mathbf{p}) \hat{\mathbf{v}}+s^{2} \mathbf{p}+2 \lambda s \hat{\mathbf{v}} \times \mathbf{p}-\lambda^{2} \mathbf{p}+\lambda^{2}(\mathbf{p} \cdot \hat{\mathbf{v}}) \hat{\mathbf{v}}\right] \\
& =\left[0,2 \lambda^{2}(\hat{\mathbf{v}} \cdot \mathbf{p}) \hat{\mathbf{v}}+\left(s^{2}-\lambda^{2}\right) \mathbf{p}+2 \lambda s \hat{\mathbf{v}} \times \mathbf{p}\right]
\end{aligned}
\end{align}
显然,这是一个纯四元数,因为标量分量为零。然而,角度翻倍的来源并不明显。但是看看当我们令$s=\cos \theta$和$\lambda=\sin \theta$时,会发生什么:
$$
\begin{aligned}
q p q^{-1} & =\left[0,2 \sin ^{2} \theta(\hat{\mathbf{v}} \cdot \mathbf{p}) \hat{\mathbf{v}}+\left(\cos ^{2} \theta-\sin ^{2} \theta\right) \mathbf{p}+2 \sin \theta \cos \theta \hat{\mathbf{v}} \times \mathbf{p}\right] \\
& =[0,(1-\cos 2 \theta)(\hat{\mathbf{v}} \cdot \mathbf{p}) \hat{\mathbf{v}}+\cos 2 \theta \mathbf{p}+\sin 2 \theta \hat{\mathbf{v}} \times \mathbf{p}] .
\end{aligned}
$$
双倍角度项出现了!现在,如果我们想让这个乘积实际旋转向量$\theta$,那么我们必须从一开始就把$\theta$减半到$q$:
\begin{align}
    q=\left[\cos \frac{1}{2} \theta, \sin \frac{1}{2} \theta \hat{\mathbf{v}}\right]
\end{align}
这使得
\begin{align}
q p q^{-1}=[0,(1-\cos \theta)(\hat{\mathbf{v}} \cdot \mathbf{p}) \hat{\mathbf{v}}+\cos \theta \mathbf{p}+\sin \theta \hat{\mathbf{v}} \times \mathbf{p}]
\end{align}
积$q p q^{-1}$是Hamilton发现的,但他没有发表结果。Cayley也发现了这种乘积,并于1845年发表了结果。然而,Altmann指出,“在Cayley的文集中,他承认Hamilton优先”[2],这是一个很好的姿态。然而,在Hamilton和Cayley之前认识到半角参数重要性的人是rodrigues,他发表了一个Hamilton没有看到的解决方案,但显然是Cayley看到的。

让我们使用前面的例子来测试(7.13),其中我们围绕四元数$\hat{\mathbf{v}}=(1 / \sqrt{2}) \mathbf{i}+(1 / \sqrt{2}) \mathbf{k}$旋转一个向量$\mathbf{p}=2 \mathbf{i}$, $\theta=90^{\circ}$
$$
\begin{aligned}
q u p q^{-1} & =[0,(1-\cos \theta)(\hat{\mathbf{v}} \cdot \mathbf{p}) \hat{\mathbf{v}}+\cos \theta \mathbf{p}+\sin \theta \hat{\mathbf{v}} \times \mathbf{p}] \\
& =[0,(\hat{\mathbf{v}} \cdot \mathbf{p}) \hat{\mathbf{v}}+\hat{\mathbf{v}} \times \mathbf{p}] \\
& =\left[0, \frac{2}{\sqrt{2}}\left(\frac{1}{\sqrt{2}} \mathbf{i}+\frac{1}{\sqrt{2}} \mathbf{k}\right)+\sqrt{2} \mathbf{j}\right] \\
& =[0, \mathbf{i}+\sqrt{2} \mathbf{j}+\mathbf{k}]
\end{aligned}
$$
这与(7.10)一致。因此,当单位范数四元数采用如下形式时
$$
q=\left[\cos \frac{1}{2} \theta, \sin \frac{1}{2} \theta \hat{\mathbf{v}}\right]
$$
存储待旋转向量的纯四元数采用这种形式
$$
p=[0, \mathbf{p}]
$$
纯四元数
$$
p^{\prime}=q p q^{-1}
$$
存储旋转后的向量$\mathbf{p}^{\prime}$。我们来看看为什么这个乘积保留了旋转矢量的大小。
$$
\begin{aligned}
\left|p^{\prime}\right| & =|q p|\left|q^{-1}\right| \\
& =|q||p|\left|q^{-1}\right| \\
& =|q|^{2}|p|
\end{aligned}
$$
且如果$q$是一个单位范数四元数,$|q|=1$,则$\left|p^{\prime}\right|=|p|$。

你可能想知道,如果乘积颠倒到$q^{-1} p q$会发生什么?一种猜测是旋转顺序颠倒了,但让我们看看代数分析证实了什么。
$$
\begin{aligned}
q^{-1} p q= & {[s,-\lambda \hat{\mathbf{v}}][0, \mathbf{p}][s, \lambda \hat{\mathbf{v}}] } \\
= & {[\lambda \hat{\mathbf{v}} \cdot \mathbf{p}, s \mathbf{p}-\lambda \hat{\mathbf{v}} \times \mathbf{p}][s, \lambda \hat{\mathbf{v}}] } \\
= & {[\lambda s \hat{\mathbf{v}} \cdot \mathbf{p}-\lambda s \mathbf{p} \cdot \hat{\mathbf{v}}} \\
& \left.\lambda^{2} \hat{\mathbf{v}} \times \mathbf{p} \cdot \hat{\mathbf{v}}+\lambda^{2} \hat{\mathbf{v}} \cdot \mathbf{p} \hat{\mathbf{v}}+s^{2} \mathbf{p}-\lambda s \hat{\mathbf{v}} \times \mathbf{p}+\lambda s \mathbf{p} \times \hat{\mathbf{v}}-\lambda^{2} \hat{\mathbf{v}} \times \mathbf{p} \times \hat{\mathbf{v}}\right] \\
= & {\left[\lambda^{2}(\hat{\mathbf{v}} \times \mathbf{p}) \cdot \hat{\mathbf{v}}, \lambda^{2}(\hat{\mathbf{v}} \cdot \mathbf{p}) \hat{\mathbf{v}}+s^{2} \mathbf{p}-2 \lambda s \hat{\mathbf{v}} \times \mathbf{p}-\lambda^{2}(\hat{\mathbf{v}} \times \mathbf{p}) \times \hat{\mathbf{v}}\right] }
\end{aligned}
$$
又一次
$$
(\hat{\mathbf{v}} \times \mathbf{p}) \cdot \hat{\mathbf{v}}=0
$$
且
$$
(\hat{\mathbf{v}} \times \mathbf{p}) \times \hat{\mathbf{v}}=\mathbf{p}-(\mathbf{p} \cdot \hat{\mathbf{v}}) \hat{\mathbf{v}}
$$
因此,
$$
\begin{aligned}
q^{-1} p q & =\left[0, \lambda^{2}(\hat{\mathbf{v}} \cdot \mathbf{p}) \hat{\mathbf{v}}+s^{2} \mathbf{p}-2 \lambda s \hat{\mathbf{v}} \times \mathbf{p}-\lambda^{2} \mathbf{p}+\lambda^{2}(\mathbf{p} \cdot \hat{\mathbf{v}}) \hat{\mathbf{v}}\right] \\
& =\left[0,2 \lambda^{2}(\hat{\mathbf{v}} \cdot \mathbf{p}) \hat{\mathbf{v}}+\left(s^{2}-\lambda^{2}\right) \mathbf{p}-2 \lambda s \hat{\mathbf{v}} \times \mathbf{p}\right]
\end{aligned}
$$
再次,让我们令$s=\cos \theta$和$\lambda=\sin \theta$:
$$
q^{-1} p q=[0,(1-\cos 2 \theta)(\hat{\mathbf{v}} \cdot \mathbf{p}) \hat{\mathbf{v}}+\cos 2 \theta \mathbf{p}-\sin 2 \theta \hat{\mathbf{v}} \times \mathbf{p}]
$$
则唯一改变$q p q^{-1}$的是叉乘项的符号,叉乘项的方向是相反的。但是,我们必须记住通过将$\theta$减半来补偿角度加倍:
\begin{align}
q^{-1} p q=[0,(1-\cos \theta)(\hat{\mathbf{v}} \cdot \mathbf{p}) \hat{\mathbf{v}}+\cos \theta \mathbf{p}-\sin \theta \hat{\mathbf{v}} \times \mathbf{p}]
\end{align}
让我们看看当我们使用(7.14)绕四元数的向量$\hat{\mathbf{v}}=(1 / \sqrt{2}) \mathbf{i}+(1 / \sqrt{2}) \mathbf{k}$旋转$\mathbf{p}=2 \mathbf{i}, 90^{\circ}$会发生什么:
$$
\begin{aligned}
q^{-1} p q & =\left[0, \frac{2}{\sqrt{2}}\left(\frac{1}{\sqrt{2}} \mathbf{i}+\frac{1}{\sqrt{2}} \mathbf{k}\right)-\sqrt{2} \mathbf{j}\right] \\
& =[0, \mathbf{i}-\sqrt{2} \mathbf{j}+\mathbf{k}]
\end{aligned}
$$
它将$\mathbf{p}$绕四元数的向量顺时针旋转$90^{\circ}$。因此,转子$q p q^{-1}$逆时针旋转一个向量,$q^{-1} p q$顺时针旋转一个向量:
$$
\begin{aligned}
q p q^{-1} & =[0,(1-\cos \theta)(\hat{\mathbf{v}} \cdot \mathbf{p}) \hat{\mathbf{v}}+\cos \theta \mathbf{p}+\sin \theta \hat{\mathbf{v}} \times \mathbf{p}] \\
q^{-1} p q & =[0,(1-\cos \theta)(\hat{\mathbf{v}} \cdot \mathbf{p}) \hat{\mathbf{v}}+\cos \theta \mathbf{p}-\sin \theta \hat{\mathbf{v}} \times \mathbf{p}]
\end{aligned}
$$

\begin{figure}[h!]
    \centering
    \includegraphics[max width=0.5\textwidth]{2023_01_16_a848224efad29cd66460g-116}
    \caption[short]{点$P(0,1,1)$围绕$y$轴旋转$90^{\circ}$到$P^{\prime}(1,1,0)$}
\end{figure}

让我们计算另一个例子。考虑图$7.6$中的点$P(0,1,1)$,它将围绕$y$轴旋转$90^{\circ}$。我们可以看到旋转后的点$P^{\prime}$的坐标为$(1,1,0)$,我们将用代数方法确认。点$P$在纯四元数中由它的位置向量$\mathbf{P}$表示
$$
p=[0, \mathbf{p}] \text {. }
$$
旋转轴为$\hat{\mathbf{v}}=\mathbf{j}$,要旋转的向量为$\mathbf{p}=\mathbf{j}+\mathbf{k}$。因此,
$$
\begin{aligned}
\operatorname{qpq}^{-1} & =[0,(1-\cos \theta)(\hat{\mathbf{v}} \cdot \mathbf{p}) \hat{\mathbf{v}}+\cos \theta \mathbf{p}+\sin \theta \hat{\mathbf{v}} \times \mathbf{p}] \\
& =[0, \mathbf{j} \cdot(\mathbf{j}+\mathbf{k}) \mathbf{j}+\mathbf{j} \times(\mathbf{j}+\mathbf{k})] \\
& =[0, \mathbf{i}+\mathbf{j}]
\end{aligned}
$$
且确认$P$确实被旋转到$(1,1,0)$。

现在我们来研究一下这个乘积是如何用矩阵形式表示的。

\section{矩阵形式的四元数}
在发现了一个向量方程来表示三重$q p q^{-1}$之后,让我们继续把它转换成一个矩阵。我们将探讨两种方法:第一种是简单的向量方法,它将表示$q p q^{-1}$的向量方程直接转换为矩阵形式。第二种方法使用矩阵代数来开发一个相当巧妙的解决方案。

\subsection{向量方法}
对于向量法,将单位范数四元数描述为
$$
\begin{aligned}
q & =[s, \mathbf{v}] \\
& =[s, x \mathbf{i}+y \mathbf{j}+z \mathbf{k}]
\end{aligned}
$$
其中
$$
s^{2}+|\mathbf{v}|^{2}=1
$$
而纯四元数为
$$
\begin{aligned}
p & =[0, \mathbf{p}] \\
& =\left[0, x_{p} \mathbf{i}+y_{p} \mathbf{j}+z_{p} \mathbf{k}\right] .
\end{aligned}
$$
计算$q p q^{-1}$的简单方法是使用(7.11)并将$|\mathbf{v}|$替换为$\lambda$:
$$
\begin{aligned}
q p q^{-1} & =\left[0,2 \lambda^{2}(\hat{\mathbf{v}} \cdot \mathbf{p}) \hat{\mathbf{v}}+\left(s^{2}-\lambda^{2}\right) \mathbf{p}+2 \lambda s \hat{\mathbf{v}} \times \mathbf{p}\right] \\
& =\left[0,2|\mathbf{v}|^{2}(\hat{\mathbf{v}} \cdot \mathbf{p}) \hat{\mathbf{v}}+\left(s^{2}-|\mathbf{v}|^{2}\right) \mathbf{p}+2|\mathbf{v}| s \hat{\mathbf{v}} \times \mathbf{p}\right]
\end{aligned}
$$
接下来,我们用$\mathbf{v}$替换$|\mathbf{v}| \hat{\mathbf{v}}$:
$$
q p q^{-1}=\left[0,2(\mathbf{v} \cdot \mathbf{p}) \mathbf{v}+\left(s^{2}-|\mathbf{v}|^{2}\right) \mathbf{p}+2 s \mathbf{v} \times \mathbf{p}\right]
$$
最后,由于我们在工作中使用的是单位规范的四元数,以防止缩放。
$$
s^{2}+|\mathbf{v}|^{2}=1
$$
且
$$
s^{2}-|\mathbf{v}|^{2}=2 s^{2}-1
$$
因此,
$$
q p q^{-1}=\left[0,2(\mathbf{v} \cdot \mathbf{p}) \mathbf{v}+\left(2 s^{2}-1\right) \mathbf{p}+2 s \mathbf{v} \times \mathbf{p}\right]
$$
如果我们令$p^{\prime}=q p q^{-1}$,这是一个纯四元数,我们有
$$
\begin{aligned}
p^{\prime} & =q p q^{-1} \\
& =\left[0, \mathbf{p}^{\prime}\right] \\
& =\left[0,2(\mathbf{v} \cdot \mathbf{p}) \mathbf{v}+\left(2 s^{2}-1\right) \mathbf{p}+2 s \mathbf{v} \times \mathbf{p}\right] \\
\mathbf{p}^{\prime} & =2(\mathbf{v} \cdot \mathbf{p}) \mathbf{v}+\left(2 s^{2}-1\right) \mathbf{p}+2 s \mathbf{v} \times \mathbf{p} .
\end{aligned}
$$
我们只对旋转向量$\mathbf{p}^{\prime}$感兴趣,它由三个项$2(\mathbf{v} \cdot \mathbf{p}) \mathbf{v}$, $\left(2s ^{2}-1\right) \mathbf{p}$和$ 2s \mathbf{v} \times \mathbf{p}$组成,它们可以由三个单独的矩阵表示并求和。
$$
\begin{aligned}
2(\mathbf{v} \cdot \mathbf{p}) \mathbf{v} & =2\left(x x_{p}+y y_{p}+z z_{p}\right)(x \mathbf{i}+y \mathbf{j}+z \mathbf{k}) \\
& =\left[\begin{array}{lll}
2 x^{2} & 2 x y & 2 x z \\
2 x y & 2 y^{2} & 2 y z \\
2 x z & 2 y z & 2 z^{2}
\end{array}\right]\left[\begin{array}{l}
x_{p} \\
y_{p} \\
z_{p}
\end{array}\right] \\
\end{aligned}
$$

$$
\begin{aligned}
\left(2 s^{2}-1\right) \mathbf{p} & =\left(2 s^{2}-1\right) x_{p} \mathbf{i}+\left(2 s^{2}-1\right) y_{p} \mathbf{j}+\left(2 s^{2}-1\right) z_{p} \mathbf{k} \\
& =\left[\begin{array}{ccc}
2 s^{2}-1 & 0 & 0 \\
0 & 2 s^{2}-1 & 0 \\
0 & 0 & 2 s^{2}-1
\end{array}\right]\left[\begin{array}{l}
x_{p} \\
y_{p} \\
z_{p}
\end{array}\right]\\
2 s \mathbf{v} \times \mathbf{p} & =2 s\left(\left(y z_{p}-z y_{p}\right) \mathbf{i}+\left(z x_{p}-x z_{p}\right) \mathbf{j}+\left(x y_{p}-y x_{p}\right) \mathbf{k}\right) \\
& =\left[\begin{array}{ccc}
0 & -2 s z & 2 s y \\
2 s z & 0 & -2 s x \\
-2 s y & 2 s x & 0
\end{array}\right]\left[\begin{array}{l}
x_{p} \\
y_{p} \\
z_{p}
\end{array}\right] .
\end{aligned}
$$
把这些矩阵加起来:
\begin{align}
\mathbf{p}^{\prime}=\left[\begin{array}{ccc}
2\left(s^{2}+x^{2}\right)-1 & 2(x y-s z) & 2(x z+s y) \\
2(x y+s z) & 2\left(s^{2}+y^{2}\right)-1 & 2(y z-s x) \\
2(x z-s y) & 2(y z+s x) & 2\left(s^{2}+z^{2}\right)-1
\end{array}\right]\left[\begin{array}{l}
x_{p} \\
y_{p} \\
z_{p}
\end{array}\right]
\end{align}
或
\begin{align}
\mathbf{p}^{\prime}=\left[\begin{array}{ccc}
1-2\left(y^{2}+z^{2}\right) & 2(x y-s z) & 2(x z+s y) \\
2(x y+s z) & 1-2\left(x^{2}+z^{2}\right) & 2(y z-s x) \\
2(x z-s y) & 2(y z+s x) & 1-2\left(x^{2}+y^{2}\right)
\end{array}\right]\left[\begin{array}{l}
x_{p} \\
y_{p} \\
z_{p}
\end{array}\right]
\end{align}
其中
$$
\left[0, \mathbf{p}^{\prime}\right]=q p q^{-1}
$$
现在让我们反转乘积。要计算$q^{-1} p q$的向量部分,我们所要做的就是将$2s \mathbf{v} \times\mathbf{p}$的符号颠倒:
\begin{align}
\mathbf{p}^{\prime}=\left[\begin{array}{ccc}
2\left(s^{2}+x^{2}\right)-1 & 2(x y+s z) & 2(x z-s y) \\
2(x y-s z) & 2\left(s^{2}+y^{2}\right)-1 & 2(y z+s x) \\
2(x z+s y) & 2(y z-s x) & 2\left(s^{2}+z^{2}\right)-1
\end{array}\right]\left[\begin{array}{l}
x_{p} \\
y_{p} \\
z_{p}
\end{array}\right]
\end{align}
或
\begin{align}
\mathbf{p}^{\prime}=\left[\begin{array}{ccc}
1-2\left(y^{2}+z^{2}\right) & 2(x y+s z) & 2(x z-s y) \\
2(x y-s z) & 1-2\left(x^{2}+z^{2}\right) & 2(y z+s x) \\
2(x z+s y) & 2(y z-s x) & 1-2\left(x^{2}+y^{2}\right)
\end{array}\right]\left[\begin{array}{l}
x_{p} \\
y_{p} \\
z_{p}
\end{array}\right]
\end{align}
其中
$$
\left[0, \mathbf{p}^{\prime}\right]=q^{-1} p q .
$$
观察出(7.17)是(7.15)的转置,(7.18)是(7.16)的转置。

\subsection{矩阵方法}
$$
\begin{aligned}
q_{a} & =\left[s_{a}, x_{a} \mathbf{i}+y_{a} \mathbf{j}+z_{a} \mathbf{k}\right] \\
q_{b} & =\left[s_{b}, x_{b} \mathbf{i}+y_{b} \mathbf{j}+z_{b} \mathbf{k}\right]
\end{aligned}
$$
它们的乘积是
$$
\begin{aligned}
q_{a} q_{b}=&\left[s_{a}, x_{a} \mathbf{i}+y_{a} \mathbf{j}+z_{a} \mathbf{k}\right]\left[s_{b}, x_{b} \mathbf{i}+y_{b} \mathbf{j}+z_{b} \mathbf{k}\right] \\
=&\left[s_{a} s_{b}-x_{a} x_{b}-y_{a} y_{b}-z_{a} z_{b},\right. \\
& s_{a}\left(x_{b} \mathbf{i}+y_{b} \mathbf{j}+z_{b} \mathbf{k}\right) \\
& +s_{b}\left(x_{a} \mathbf{i}+y_{a} \mathbf{j}+z_{a} \mathbf{k}\right) \\
& \left.+\left(y_{a} z_{b}-y_{b} z_{a}\right) \mathbf{i}+\left(x_{b} z_{a}-x_{a} z_{b}\right) \mathbf{j}+\left(x_{a} y_{b}-x_{b} y_{a}\right) \mathbf{k}\right] \\
=&\left[s_{a} s_{b}-x_{a} x_{b}-y_{a} y_{b}-z_{a} z_{b}\right. \text {, } \\
& \left(s_{a} x_{b}+s_{b} x_{a}+y_{a} z_{b}-y_{b} z_{a}\right) \mathbf{i} \\
& +\left(s_{a} y_{b}+s_{b} y_{a}+x_{b} z_{a}-x_{a} z_{b}\right) \mathbf{j} \\
& \left.+\left(s_{a} z_{b}+s_{b} z_{a}+x_{a} y_{b}-x_{b} y_{a}\right) \mathbf{k}\right] \\
=&\begin{bmatrix}s_{a} & -x_{a} & -y_{a} & -z_{a} \\x_{a} & s_{a} & -z_{a} & y_{a} \\y_{a} & z_{a} & s_{a} & -x_{a} \\z_{a} & -y_{a} & x_{a} & s_{a}\end{bmatrix}\begin{bmatrix}s_{b} \\x_{b} \\y_{b} \\z_{b}\end{bmatrix}=\mathbf{A} q_{b} .
\end{aligned}
$$
在这个阶段,我们有四元数$q_{a}$表示为矩阵$\mathbf{A}$,四元数$q_{b}$表示为列向量。现在让我们在不改变结果的情况下,将$q_{b}$设为矩阵,$q_{a}$设为列向量:
$$
q_{a} q_{b}=\begin{bmatrix}
s_{b} & -x_{b} & -y_{b} & -z_{b} \\
x_{b} & s_{b} & z_{b} & -y_{b} \\
y_{b} & -z_{b} & s_{b} & x_{b} \\
z_{b} & y_{b} & -x_{b} & s_{b}
\end{bmatrix}\begin{bmatrix}
s_{a} \\
x_{a} \\
y_{a} \\
z_{a}
\end{bmatrix}=\mathbf{B} q_{a}
$$
现在我们有两种方法来计算$q_{a} q_{b}$,我们需要一种方法来区分这两个矩阵。设$\mathbf{L}$是保留从左到右四元数序列的矩阵,$\mathbf{R}$是将序列从右到左反转的矩阵:
$$
\begin{aligned}
& q_{a} q_{b}=\mathbf{L}\left(q_{a}\right) q_{b}= {\left[\begin{array}{cccc}
s_{a} & -x_{a} & -y_{a} & -z_{a} \\
x_{a} & s_{a} & -z_{a} & y_{a} \\
y_{a} & z_{a} & s_{a} & -x_{a} \\
z_{a} & -y_{a} & x_{a} & s_{a}
\end{array}\right]\left[\begin{array}{c}
s_{b} \\
x_{b} \\
y_{b} \\
z_{b}
\end{array}\right] } \\
\end{aligned}
$$
$$
\begin{aligned}
& q_{a} q_{b}=\mathbf{R}\left(q_{b}\right) q_{a}=\left[\begin{array}{cccc}
s_{b} & -x_{b} & -y_{b} & -z_{b} \\
x_{b} & s_{b} & z_{b} & -y_{b} \\
y_{b} & -z_{b} & s_{b} & x_{b} \\
z_{b} & y_{b} & -x_{b} & s_{b}
\end{array}\right]\left[\begin{array}{c}
s_{a} \\
x_{a} \\
y_{a} \\
z_{a}
\end{array}\right] .
\end{aligned}
$$
请记住$\mathbf{L}\left(q_{a}\right) q_{b}=\mathbf{R}\left(q_{b}\right) q_{a}$,因为这是理解下一阶段的核心。此外,如果你不能理解第一篇阅读中的论点,也不要感到惊讶。作者花了很多小时痛苦地试图破译原始算法,这个解释已经扩展,以确保您不会遭受同样的经历!

首先,让我们使用矩阵$\mathbf{L}$和$\mathbf{R}$将四元数积$q_{a} q_{c} q_{b}$重新排列到$q_{a} q_{b} q_{c}$,即将$q_{c}$从中间移到右边。我们从四元数积$q_{a} q_{c} q_{b}$开始,并将其分为$q_{a} q_{c}$和$q_{b}$两部分。我们可以这样做,因为四元数代数是符合结合律的:
$$
q_{a} q_{c} q_{b}=\left(q_{a} q_{c}\right) q_{b}
$$
前面已经演示过,可以用$\mathbf{L}\left(q_{a}\right) q_{c}$替换乘积$q_{a} q_{c}$
$$
q_{a} q_{c} q_{b}=\mathbf{L}\left(q_{a}\right) q_{c} q_{b}
$$
我们现在有另外两个部分:$\mathbf{L}\left(q_{a}\right) q_{c}$和$q_{b}$,它们可以使用$\mathbf{R}$来反转,而不会影响结果:
$$
q_{a} q_{c} q_{b}=\mathbf{L}\left(q_{a}\right) q_{c} q_{b}=\mathbf{R}\left(q_{b}\right) \mathbf{L}\left(q_{a}\right) q_{c}
$$
这样就实现了将$q_{c}$移到右边的目标。但最重要的结果是,矩阵$\mathbf{R}\left(q_{b}\right)$和$\mathbf{L}\left(q_{a}\right)$可以相乘,形成一个对$q_{c}$操作的矩阵。

现在让我们重复相同的过程来重新排列乘积$q p q^{-1}$。目标是将$p$从$q$和$q^{-1}$的中间移到右边。这样做的原因是将$q$和$q^{-1}$以两个矩阵的形式放在一起,它们可以相乘成一个矩阵。

我们从四元数积$q p q^{-1}$开始,并将其分为$q p$和$q^{-1}$两部分
$$
q p q^{-1}=(q p) q^{-1}
$$
乘积$ qp $可以替换为$\mathbf{L}(q) p$:
$$
q p q^{-1}=\mathbf{L}(q) p q^{-1}
$$
我们现在有另外两个部分:$\mathbf{L}(q) p$和$q^{-1}$,可以使用$\mathbf{R}$反转而不影响结果:
$$
q p q^{-1}=\mathbf{L}(q) p q^{-1}=\mathbf{R}\left(q^{-1}\right) \mathbf{L}(q) p
$$
这就实现了把p移到右边的目标。

下一步是计算$\mathbf{L}(q)$和$\mathbf{R}\left(q^{-1}\right)$使用$q=[s, x \mathbf{i}+y \mathbf{j}+z \mathbf{k}]$。

$\mathbf{L}(q)$很简单,因为它与$\mathbf{L}\left(q_{a}\right)$相同:
$$
\mathbf{L}(q)=\left[\begin{array}{cccc}
s & -x & -y & -z \\
x & s & -z & y \\
y & z & s & -x \\
z & -y & x & s
\end{array}\right]
$$
$\mathbf{R}\left(q^{-1}\right)$也很简单,但需要将原始定义中的$q_{b}$转换为$q^{-1}$,这是通过反转向量分量的符号来实现的:
$$
\mathbf{R}\left(q^{-1}\right)=\left[\begin{array}{cccc}
s & x & y & z \\
-x & s & -z & y \\
-y & z & s & -x \\
-z & -y & x & s
\end{array}\right]
$$

\begin{figure}[h!]
    \centering
    \includegraphics[max width=0.5\textwidth]{2023_01_16_a848224efad29cd66460g-121}
    \caption[short]{ti点$P(0,1,1)$围绕$y$轴旋转$90^{\circ}$到$P^{\prime}(1,1,0)$tle}
\end{figure}

现在我们可以写出
$$
\begin{aligned}
q p q^{-1} & =\mathbf{R}\left(q^{-1}\right) \mathbf{L}(q) p \\
& =\left[\begin{array}{cccc}
s & x & y & z \\
-x & s & -z & y \\
-y & z & s & -x \\
-z & -y & x & s
\end{array}\right]\left[\begin{array}{cccc}
s & -x & -y & -z \\
x & s & -z & y \\
y & z & s & -x \\
z & -y & x & s
\end{array}\right]\left[\begin{array}{c}
0 \\
x_{p} \\
y_{p} \\
z_{p}
\end{array}\right] \\
& =\left[\begin{array}{cccc}
1 & 0 & 0 & 0 \\
0 & 1-2\left(y^{2}+z^{2}\right) & 2(x y-s z) & 2(x z+s y) \\
0 & 2(x y+s z) & 1-2\left(x^{2}+z^{2}\right) & 2(y z-s x) \\
0 & 2(x z-s y) & 2(y z+s x) & 1-2\left(x^{2}+y^{2}\right)
\end{array}\right]\left[\begin{array}{c}
0 \\
x_{p} \\
y_{p} \\
z_{p}
\end{array}\right] .
\end{aligned}
$$
如果忽略第一行和第一列,矩阵将计算$\mathbf{p}^{\prime}$:
$$
\mathbf{p}^{\prime}=\left[\begin{array}{ccc}
1-2\left(y^{2}+z^{2}\right) & 2(x y-s z) & 2(x z+s y) \\
2(x y+s z) & 1-2\left(x^{2}+z^{2}\right) & 2(y z-s x) \\
2(x z-s y) & 2(y z+s x) & 1-2\left(x^{2}+y^{2}\right)
\end{array}\right]\left[\begin{array}{l}
x_{p} \\
y_{p} \\
z_{p}
\end{array}\right]
$$
这和(7.16)是一样的!

\subsection{几何验证}
让我们通过围绕$y$轴旋转点$(0,1,1), 90^{\circ}$来说明(7.15)的作用,如图7.7所示。四元数采用这种形式
$$
q=\left[\cos \frac{1}{2} \theta, \sin \frac{1}{2} \theta \hat{\mathbf{v}}\right]
$$
这意味着$\theta=90^{\circ}$和$\hat{\mathbf{v}}=\mathbf{j}$,因此,
$$
q=\left[\cos 45^{\circ}, \sin 45^{\circ} \hat{\mathbf{j}}\right]
$$
因此
$$
s=\frac{\sqrt{2}}{2}, \quad x=0, \quad y=\frac{\sqrt{2}}{2}, \quad z=0 .
$$
在(7.15)中代入这些值给出
$$
\begin{aligned}
\mathbf{p}^{\prime} & =\left[\begin{array}{ccc}
2\left(s^{2}+x^{2}\right)-1 & 2(x y-s z) & 2(x z+s y) \\
2(x y+s z) & 2\left(s^{2}+y^{2}\right)-1 & 2(y z-s x) \\
2(x z-s y) & 2(y z+s x) & 2\left(s^{2}+z^{2}\right)-1
\end{array}\right]\left[\begin{array}{l}
x_{p} \\
y_{p} \\
z_{p}
\end{array}\right] \\
{\left[\begin{array}{l}
1 \\
1 \\
0
\end{array}\right] } & =\left[\begin{array}{ccc}
0 & 0 & 1 \\
0 & 1 & 0 \\
-1 & 0 & 0
\end{array}\right]\left[\begin{array}{l}
0 \\
1 \\
1
\end{array}\right]
\end{aligned}
$$
其中$(0,1,1)$被旋转为$(1,1,0)$,这是正确的。

现在我们有了一个变换,可以让一个点绕任意轴旋转,这个轴与原点相交而没有欧拉变换带来的万向节锁定问题。

在继续之前,让我们再看一个例子。让我们对一个向量$\mathbf{v}=\mathbf{i}+\mathbf{k}$通过原点执行$180^{\circ}$旋转。首先,我们会故意忘记把这个向量转换成单位向量,只是为了看看最终矩阵会发生什么。四元数采用这种形式
$$
q=\left[\cos \frac{1}{2} \theta, \sin \frac{1}{2} \theta \hat{\mathbf{v}}\right]
$$
但是我们将使用指定的$\mathbf{v}$。因此,当使用$\theta=180^{\circ}$
$$
s=0, \quad x=1, \quad y=0, \quad z=1 .
$$
在(7.15)中代入这些值给出
$$
\begin{aligned}
\mathbf{p}^{\prime} & =\left[\begin{array}{ccc}
2\left(s^{2}+x^{2}\right)-1 & 2(x y-s z) & 2(x z+s y) \\
2(x y+s z) & 2\left(s^{2}+y^{2}\right)-1 & 2(y z-s x) \\
2(x z-s y) & 2(y z+s x) & 2\left(s^{2}+z^{2}\right)-1
\end{array}\right]\left[\begin{array}{l}
x_{p} \\
y_{p} \\
z_{p}
\end{array}\right] \\
& =\left[\begin{array}{ccc}
1 & 0 & 2 \\
0 & -1 & 0 \\
2 & 0 & 1
\end{array}\right]\left[\begin{array}{l}
1 \\
0 \\
0
\end{array}\right]
\end{aligned}
$$
它看起来一点也不像旋转矩阵,这提醒我们用单位向量来表示轴是多么重要。让我们重复这些计算,将向量归一化为$\hat{\mathbf{v}}=\frac{1}{\sqrt{2}} \mathbf{i}+\frac{1}{\sqrt{2}} \mathbf{k}$:
$$
s=0, \quad x=\frac{1}{\sqrt{2}}, \quad y=0, \quad z=\frac{1}{\sqrt{2}} .
$$
在(7.15)中代入这些值给出
$$
\begin{aligned}
\mathbf{p}^{\prime} & =\left[\begin{array}{ccc}
2\left(s^{2}+x^{2}\right)-1 & 2(x y-s z) & 2(x z+s y) \\
2(x y+s z) & 2\left(s^{2}+y^{2}\right)-1 & 2(y z-s x) \\
2(x z-s y) & 2(y z+s x) & 2\left(s^{2}+z^{2}\right)-1
\end{array}\right]\left[\begin{array}{l}
x_{p} \\
y_{p} \\
z_{p}
\end{array}\right] \\
{\left[\begin{array}{l}
0 \\
0 \\
1
\end{array}\right] } & =\left[\begin{array}{ccc}
0 & 0 & 1 \\
0 & -1 & 0 \\
1 & 0 & 0
\end{array}\right]\left[\begin{array}{l}
1 \\
0 \\
0
\end{array}\right]
\end{aligned}
$$
它不仅看起来像一个旋转矩阵,而且行列式为1,并将点$(1,0,0)$旋转到$(0,0,1)$,如图7.8所示。

\begin{figure}[h!]
    \centering
    \includegraphics[max width=0.5\textwidth]{2023_01_16_a848224efad29cd66460g-123}
    \caption[short]{点$(1,0,0)$围绕向量$\hat{\mathbf{v}}$旋转$180^{\circ}$到$(0,0,1)$}
\end{figure}

\section{多个旋转}
假设一个向量或参考系受到$q_{1}$和$q_{2}$指定的两次旋转。有一种诱惑是将两个四元数转换为各自的矩阵并将矩阵相乘。然而,这并不是结合旋转的最有效的方法。最好将旋转累积为四元数,然后在需要时转换为矩阵符号。

为了说明这一点,考虑纯四元数$p$受到第一个四元数$q_{1}$的影响:
$$
q_{1} p q_{1}^{-1}
$$
后面跟着第二个四元数$q_{2}$
$$
q_{2}\left(q_{1} p q_{1}^{-1}\right) q_{2}^{-1}
$$
这可以表示为
$$
\left(q_{2} q_{1}\right) p\left(q_{2} q_{1}\right)^{-1} .
$$
可以相应地添加额外的四元数。让我们用两个例子来说明这一点。

为了简单起见,第一个四元数$q_{1}$围绕$y$-轴旋转$30^{\circ}$:
$$
q_{1}=\left[\cos 15^{\circ}, \sin 15^{\circ} \mathbf{j}\right] .
$$
第二个四元数$q_{2}$也围绕$y$轴旋转$60^{\circ}$:
$$
q_{2}=\left[\cos 30^{\circ}, \sin 30^{\circ} \mathbf{j}\right] .
$$
两个四元数一起围绕$y$轴旋转$90^{\circ}$。为了累积这些旋转,我们将它们相乘:
$$
\begin{aligned}
q_{1} q_{2} & =\left[\cos 15^{\circ}, \sin 15^{\circ} \mathbf{j}\right]\left[\cos 30^{\circ}, \sin 30^{\circ} \mathbf{j}\right] \\
& =\left[\cos 15^{\circ} \cos 30^{\circ}-\sin 15^{\circ} \sin 30^{\circ}, \cos 15^{\circ} \sin 30^{\circ} \mathbf{j}+\cos 30^{\circ} \sin 15^{\circ} \mathbf{j}\right] \\
& =\left[\frac{\sqrt{2}}{2}, \frac{\sqrt{2}}{2} \mathbf{j}\right]
\end{aligned}
$$
它是一个四元数,围绕y轴旋转。使用矩阵(7.15)我们有
$$
\begin{aligned}
\mathbf{p}^{\prime} & =\left[\begin{array}{ccc}
2\left(s^{2}+x^{2}\right)-1 & 2(x y-s z) & 2(x z+s y) \\
2(x y+s z) & 2\left(s^{2}+y^{2}\right)-1 & 2(y z-s x) \\
2(x z-s y) & 2(y z+s x) & 2\left(s^{2}+z^{2}\right)-1
\end{array}\right]\left[\begin{array}{l}
x_{p} \\
y_{p} \\
z_{p}
\end{array}\right] \\
& =\left[\begin{array}{ccc}
0 & 0 & 1 \\
0 & 1 & 0 \\
-1 & 0 & 0
\end{array}\right]\left[\begin{array}{c}
x_{p} \\
y_{p} \\
z_{p}
\end{array}\right]
\end{aligned}
$$
它围绕y轴旋转$90^{\circ}$。

第二个例子,让我们求四元数的值。第一个四元数$q_{1}$围绕$x$-轴旋转$90^{\circ}$, $q_{2}$围绕$y$-轴旋转$90^{\circ}$:

$$
\begin{aligned}
q_{1} & =\left[\frac{\sqrt{2}}{2}, \frac{\sqrt{2}}{2} \mathbf{i}\right] \\
q_{2} & =\left[\frac{\sqrt{2}}{2}, \frac{\sqrt{2}}{2} \mathbf{j}\right] \\
p & =[0, \mathbf{i}+\mathbf{j}]
\end{aligned}
$$
因此,
$$
\begin{aligned}
q_{2} q_{1} & =\left[\frac{\sqrt{2}}{2}, \frac{\sqrt{2}}{2} \mathbf{j}\right]\left[\frac{\sqrt{2}}{2}, \frac{\sqrt{2}}{2} \mathbf{i}\right] \\
& =\left[\frac{1}{2}, \frac{\sqrt{2}}{2} \frac{\sqrt{2}}{2} \mathbf{i}+\frac{\sqrt{2}}{2} \frac{\sqrt{2}}{2} \mathbf{j}-\frac{1}{2} \mathbf{k}\right] \\
& =\left[\frac{1}{2}, \frac{1}{2} \mathbf{i}+\frac{1}{2} \mathbf{j}-\frac{1}{2} \mathbf{k}\right] \\
\left(q_{2} q_{1}\right)^{-1} & =\left[\frac{1}{2},-\frac{1}{2} \mathbf{i}-\frac{1}{2} \mathbf{j}+\frac{1}{2} \mathbf{k}\right] \\
\left(q_{2} q_{1}\right) p & =\left[\frac{1}{2}, \frac{1}{2} \mathbf{i}+\frac{1}{2} \mathbf{j}-\frac{1}{2} \mathbf{k}\right][0, \mathbf{i}+\mathbf{j}] \\
& =\left[-\frac{1}{2}-\frac{1}{2}, \frac{1}{2}(\mathbf{i}+\mathbf{j})+\frac{1}{2} \mathbf{i}-\frac{1}{2} \mathbf{j}\right] \\
& =[-1, \mathbf{i}] \\
\left(q_{2} q_{1}\right) p\left(q_{2} q_{1}\right)^{-1} & =[-1, \mathbf{i}]\left[\frac{1}{2},-\frac{1}{2} \mathbf{i}-\frac{1}{2} \mathbf{j}+\frac{1}{2} \mathbf{k}\right] \\
& =\left[-\frac{1}{2}+\frac{1}{2}, \frac{1}{2} \mathbf{i}+\frac{1}{2} \mathbf{j}-\frac{1}{2} \mathbf{k}+\frac{1}{2} \mathbf{i}-\frac{1}{2} \mathbf{j}-\frac{1}{2} \mathbf{k}\right] \\
& =[0, \mathbf{i}-\mathbf{k}] .
\end{aligned}
$$
因此,点$(1,1,0)$被旋转到$(1,0,-1)$,这是正确的。

\section{特征值和特征向量}
Although there is no doubt that (7.15) is a rotation matrix, we can secure further evidence by calculating its eigenvalue and eigenvector. The eigenvalue should be $\theta$ where

$$
\operatorname{Tr}\left(q p q^{-1}\right)=1+2 \cos \theta
$$

and $\operatorname{Tr}$ is the trace function, which is the sum of the diagonal elements of a matrix.

The trace of $(7.15)$ is

$$
\begin{aligned}
\operatorname{Tr}\left(q p q^{-1}\right) & =2\left(s^{2}+x^{2}\right)-1+2\left(s^{2}+y^{2}\right)-1+2\left(s^{2}+z^{2}\right)-1 \\
& =4 s^{2}+2\left(s^{2}+x^{2}+y^{2}+z^{2}\right)-3 \\
& =4 s^{2}-1 \\
& =4 \cos ^{2} \frac{1}{2} \theta-1 \\
& =4 \cos \theta+4 \sin ^{2} \frac{1}{2} \theta-1 \\
& =4 \cos \theta+2-2 \cos \theta-1 \\
& =1+2 \cos \theta
\end{aligned}
$$

and

$$
\cos \theta=\frac{1}{2}\left(\operatorname{Tr}\left(q p q^{-1}\right)-1\right) .
$$

To compute the eigenvector of (7.15) we use the three equations derived in Appendix:

$$
\begin{aligned}
& x_{v}=\left(a_{22}-1\right)\left(a_{33}-1\right)-a_{23} a_{32} \\
& y_{v}=\left(a_{33}-1\right)\left(a_{11}-1\right)-a_{31} a_{13} \\
& z_{v}=\left(a_{11}-1\right)\left(a_{22}-1\right)-a_{12} a_{21} .
\end{aligned}
$$

Therefore,

$$
\begin{aligned}
x_{v} & =\left(2\left(s^{2}+y^{2}\right)-2\right)\left(2\left(s^{2}+z^{2}\right)-2\right)-2(y z-s x) 2(y z+s x) \\
& =4\left(s^{2}+y^{2}-1\right)\left(s^{2}+z^{2}-1\right)-4\left(y^{2} z^{2}-s^{2} x^{2}\right) \\
& =4\left(\left(x^{2}+z^{2}\right)\left(x^{2}+y^{2}\right)-y^{2} z^{2}+s^{2} x^{2}\right) \\
& =4\left(x^{4}+x^{2} y^{2}+x^{2} z^{2}+z^{2} y^{2}-y^{2} z^{2}+s^{2} x^{2}\right) \\
& =4 x^{2}\left(s^{2}+x^{2}+y^{2}+z^{2}\right) \\
& =4 x^{2} .
\end{aligned}
$$

Similarly, $y_{v}=4 y^{2}$ and $z_{v}=4 z^{2}$, which confirm that the eigenvector has components associated with the quaternion's vector. The square terms should be no surprise, as the triple $q p q^{-1}$ includes the product of three quaternions. Let's test these formulae with the matrix associated with Fig. 7.8, which rotates a point $180^{\circ}$ about the vector $\hat{\mathbf{v}}=\frac{1}{\sqrt{2}} \mathbf{i}+\frac{1}{\sqrt{2}} \mathbf{k}$ :

$$
\mathbf{M}=\left[\begin{array}{lll}
a_{11} & a_{12} & a_{13} \\
a_{21} & a_{22} & a_{23} \\
a_{31} & a_{32} & a_{33}
\end{array}\right]=\left[\begin{array}{ccc}
0 & 0 & 1 \\
0 & -1 & 0 \\
1 & 0 & 0
\end{array}\right]
$$

therefore,

$$
\begin{aligned}
& x_{v}=-2 \times-1-0=2 \\
& y_{v}=-1 \times-1-1 \times 1=0 \\
& z_{v}=-1 \times-2-0=2
\end{aligned}
$$

which confirms that the eigenvector is $2 \mathbf{i}+2 \mathbf{k}$.

Next, $\operatorname{Tr}(\mathbf{M})=-1$, therefore

$$
\begin{aligned}
\cos \theta & =\frac{1}{2}\left(\operatorname{Tr}\left(q p q^{-1}\right)-1\right) \\
& =\frac{1}{2}((-1)-1) \\
& =-1 \\
\theta & =\pm 180^{\circ}
\end{aligned}
$$

which agrees with the previous results.

\section{绕偏移轴旋转}
Now that we have a matrix to represent a quaternion rotor, we can employ it to resolve problems such as rotating a point about an off-set axis using the same techniques associated with normal rotation transforms. For example, in Chap. 6 we used the following notation

$$
\left[\begin{array}{c}
x^{\prime} \\
y^{\prime} \\
z^{\prime} \\
1
\end{array}\right]=\mathbf{T}_{t_{x}, 0, t_{z}} \mathbf{R}_{\beta, y} \mathbf{T}_{-t_{x}, 0,-t_{z}}\left[\begin{array}{c}
x \\
y \\
z \\
1
\end{array}\right]
$$

to rotate a point about a fixed axis parallel with the $y$-axis. Therefore, by substituting the matrix $q p q^{-1}$ for $\mathbf{R}_{\beta, y}$ we have

$$
\left[\begin{array}{c}
x^{\prime} \\
y^{\prime} \\
z^{\prime} \\
1
\end{array}\right]=\mathbf{T}_{t_{x}, 0, t_{z}}\left(q p q^{-1}\right) \mathbf{T}_{-t_{x}, 0,-t_{z}}\left[\begin{array}{c}
x \\
y \\
z \\
1
\end{array}\right] .
$$

Let's test this by rotating our unit cube $90^{\circ}$ about the vertical axis intersecting vertices 4 and 6 as shown in Fig. 7.9. (a)

\begin{center}
\includegraphics[max width=\textwidth]{2023_01_16_a848224efad29cd66460g-127}
\end{center}

(b)

\begin{center}
\includegraphics[max width=\textwidth]{2023_01_16_a848224efad29cd66460g-127(1)}
\end{center}

Fig. 7.9 The cube is rotated $90^{\circ}$ about the axis intersecting vertices 4 and 6

The unit-norm quaternion to achieve this is

$$
q=\left[\cos 45^{\circ}, \sin 45^{\circ} \mathbf{j}\right]
$$

with the pure quaternion

$$
p=[0, \mathbf{p}]
$$

Consequently,

$$
s=\frac{\sqrt{2}}{2}, \quad x=0, \quad y=\frac{\sqrt{2}}{2}, \quad z=0
$$

and using (7.15) in a homogeneous form we have

$$
\begin{aligned}
\mathbf{p}^{\prime} & =\left[\begin{array}{cccc}
2\left(s^{2}+x^{2}\right)-1 & 2(x y-s z) & 2(x z+s y) & 0 \\
2(x y+s z) & 2\left(s^{2}+y^{2}\right)-1 & 2(y z-s x) & 0 \\
2(x z-s y) & 2(y z+s x) & 2\left(s^{2}+z^{2}\right)-1 & 0 \\
0 & 0 & 1
\end{array}\right]\left[\begin{array}{c}
x_{p} \\
y_{p} \\
z_{p} \\
1
\end{array}\right] \\
& =\left[\begin{array}{cccc}
0 & 0 & 1 & 0 \\
0 & 1 & 0 & 0 \\
-1 & 0 & 0 & 0 \\
0 & 0 & 0 & 1
\end{array}\right]\left[\begin{array}{c}
x_{p} \\
y_{p} \\
z_{p} \\
1
\end{array}\right]
\end{aligned}
$$

The other two matrices are

$$
\begin{aligned}
\mathbf{T}_{-t_{x}, 0,0} & =\left[\begin{array}{cccc}
1 & 0 & 0 & -1 \\
0 & 1 & 0 & 0 \\
0 & 0 & 1 & 0 \\
0 & 0 & 0 & 1
\end{array}\right] \\
\mathbf{T}_{t_{x}, 0,0} & =\left[\begin{array}{llll}
1 & 0 & 0 & 1 \\
0 & 1 & 0 & 0 \\
0 & 0 & 1 & 0 \\
0 & 0 & 0 & 1
\end{array}\right] .
\end{aligned}
$$

Multiplying these three matrices together creates

$$
\left[\begin{array}{cccc}
0 & 0 & 1 & 1 \\
0 & 1 & 0 & 0 \\
-1 & 0 & 0 & 1 \\
0 & 0 & 0 & 1
\end{array}\right]
$$

Although not mathematically correct, the following statement shows the matrix (7.19) and the array of coordinates representing a unit cube, followed by the rotated cube's coordinates.

$$
\begin{gathered}
{\left[\begin{array}{cccc}
0 & 0 & 1 & 1 \\
0 & 1 & 0 & 0 \\
-1 & 0 & 0 & 1 \\
0 & 0 & 0 & 1
\end{array}\right]\left[\begin{array}{llllllll}
0 & 0 & 0 & 0 & 1 & 1 & 1 & 1 \\
0 & 0 & 1 & 1 & 0 & 0 & 1 & 1 \\
0 & 1 & 0 & 1 & 0 & 1 & 0 & 1 \\
1 & 1 & 1 & 1 & 1 & 1 & 1 & 1
\end{array}\right]} \\
=\left[\begin{array}{llllllll}
1 & 2 & 1 & 2 & 1 & 2 & 1 & 2 \\
0 & 0 & 1 & 1 & 0 & 0 & 1 & 1 \\
1 & 1 & 1 & 1 & 0 & 0 & 0 & 0 \\
1 & 1 & 1 & 1 & 1 & 1 & 1 & 1
\end{array}\right]
\end{gathered}
$$

These coordinates are confirmed by Fig. 7.9.

\section{参考系}
The product $q p q^{-1}$ is used for rotating points about the vector associated with the quaternion $q$, whereas the triple $q^{-1} p q$ can be used for rotating points about the same vector in the opposite direction. But this reverse rotation is also equivalent to a change of frame of reference. To demonstrate this, consider the problem of rotating the frame of reference $180^{\circ}$ about $\mathbf{i}+\mathbf{k}$ as shown in Fig. 7.10. The unitnorm quaternion for such a rotation is

$$
\begin{aligned}
q & =\left[\cos 90^{\circ}, \sin 90^{\circ}\left(\frac{1}{\sqrt{2}} \mathbf{i}+\frac{1}{\sqrt{2}} \mathbf{k}\right)\right] \\
& =\left[0, \frac{\sqrt{2}}{2} \mathbf{i}+\frac{\sqrt{2}}{2} \mathbf{k}\right] .
\end{aligned}
$$

Consequently,

$$
s=0, \quad x=\frac{\sqrt{2}}{2}, \quad y=0, \quad z=\frac{\sqrt{2}}{2} .
$$

Substituting these values in (7.17) we obtain

$$
\begin{aligned}
q^{-1} p q & =\left[\begin{array}{ccc}
2\left(s^{2}+x^{2}\right)-1 & 2(x y+s z) & 2(x z-s y) \\
2(x y-s z) & 2\left(s^{2}+y^{2}\right)-1 & 2(y z+s x) \\
2(x z+s y) & 2(y z-s x) & 2\left(s^{2}+z^{2}\right)-1
\end{array}\right]\left[\begin{array}{l}
x_{p} \\
y_{p} \\
z_{p}
\end{array}\right] \\
& =\left[\begin{array}{ccc}
0 & 0 & 1 \\
0 & -1 & 0 \\
1 & 0 & 0
\end{array}\right]\left[\begin{array}{l}
x_{p} \\
y_{p} \\
z_{p}
\end{array}\right]
\end{aligned}
$$

(a)

\begin{center}
\includegraphics[max width=\textwidth]{2023_01_16_a848224efad29cd66460g-129}
\end{center}

(b)

\begin{center}
\includegraphics[max width=\textwidth]{2023_01_16_a848224efad29cd66460g-129(1)}
\end{center}

Fig. 7.10 The frame is rotated $180^{\circ}$ about the vector $\mathbf{i}+\mathbf{k}$

which, if used to process the coordinates of our unit cube, produces

$$
\begin{gathered}
{\left[\begin{array}{ccc}
0 & 0 & 1 \\
0 & -1 & 0 \\
1 & 0 & 0
\end{array}\right]\left[\begin{array}{cccccccc}
0 & 0 & 0 & 0 & 1 & 1 & 1 & 1 \\
0 & 0 & 1 & 1 & 0 & 0 & 1 & 1 \\
0 & 1 & 0 & 1 & 0 & 1 & 0 & 1
\end{array}\right]} \\
=\left[\begin{array}{cccccccc}
0 & 1 & 0 & 1 & 0 & 1 & 0 & 1 \\
0 & 0 & -1 & -1 & 0 & 0 & -1 & -1 \\
0 & 0 & 0 & 0 & 1 & 1 & 1 & 1
\end{array}\right] .
\end{gathered}
$$

This scenario is shown in Fig. $7.10$.

\section{插值四元数}
Like vectors, quaternions can be interpolated to compute an in-between quaternion. However, whereas two interpolated vectors results in a third vector that is readily visualised, two interpolated quaternions results in a third quaternion that acts as a rotor, and is not immediately visualised.

The spherical interpolant for vectors is

$$
\mathbf{v}=\frac{\sin (1-t) \theta}{\sin \theta} \mathbf{v}_{1}+\frac{\sin t \theta}{\sin \theta} \mathbf{v}_{2}
$$

where $\theta$ is the angle between the vectors, and requires no modification for quaternions:

$$
q=\frac{\sin (1-t) \theta}{\sin \theta} q_{1}+\frac{\sin t \theta}{\sin \theta} q_{2}
$$

So, given

$$
\begin{aligned}
& q_{1}=\left[s_{1}, x_{1} \mathbf{i}+y_{1} \mathbf{j}+z_{1} \mathbf{k}\right] \\
& q_{2}=\left[s_{2}, x_{2} \mathbf{i}+y_{2} \mathbf{j}+z_{2} \mathbf{k}\right]
\end{aligned}
$$

$\theta$ is obtained by taking the 4D dot product of $q_{1}$ and $q_{2}$ : Fig. 7.11 The point $(0,1,1)$ is rotated $90^{\circ}$ about the vector $\mathbf{v}_{1}$ to $(1,1,0)$
\includegraphics[max width=\textwidth, center]{2023_01_16_a848224efad29cd66460g-130}

Fig. 7.12 The point $(0,1,1)$ is rotated $90^{\circ}$ about the vector $\mathbf{v}_{2}$ to $(0,-1,1)$

$$
\begin{aligned}
\cos \theta & =\frac{q_{1} \cdot q_{2}}{\left|q_{1}\right|\left|q_{2}\right|} \\
& =\frac{s_{1} s_{2}+x_{1} x_{2}+y_{1} y_{2}+z_{1} z_{2}}{\left|q_{1}\right|\left|q_{2}\right|}
\end{aligned}
$$

and if we are working with unit-norm quaternions, then

$$
\cos \theta=s_{1} s_{2}+x_{1} x_{2}+y_{1} y_{2}+z_{1} z_{2} .
$$

Let's use (7.20) in a scenario with two simple unit-norm quaternions.

Figure $7.11$ shows one such scenario where the point $(0,1,1)$ is rotated $90^{\circ}$ about $\mathbf{v}_{1}$, the axis of $q_{1}$. Figure $7.12$ shows another scenario where the same point $(0,1,1)$ is rotated $90^{\circ}$ about $\mathbf{v}_{2}$, the axis of $q_{2}$. The quaternions are

$$
\begin{aligned}
& q_{1}=\left[\cos 45^{\circ}, \sin 45^{\circ} \mathbf{j}\right]=\left[\frac{\sqrt{2}}{2}, \frac{\sqrt{2}}{2} \mathbf{j}\right] \\
& q_{2}=\left[\cos 45^{\circ}, \sin 45^{\circ} \mathbf{i}\right]=\left[\frac{\sqrt{2}}{2}, \frac{\sqrt{2}}{2} \mathbf{i}\right] .
\end{aligned}
$$

Therefore, using (7.21)

$$
\begin{aligned}
\cos \theta & =\frac{\sqrt{2}}{2} \frac{\sqrt{2}}{2}=0.5 \\
\theta & =60^{\circ} .
\end{aligned}
$$

Before proceeding, let's compute the matrices for the two quaternion products. For $q_{1}$ :

$$
s=\frac{\sqrt{2}}{2}, \quad x=0, \quad y=\frac{\sqrt{2}}{2}, \quad z=0
$$

which when substituted in (7.15) gives

$$
\begin{aligned}
\mathbf{p}_{1}^{\prime} & =\left[\begin{array}{ccc}
2\left(s^{2}+x^{2}\right)-1 & 2(x y-s z) & 2(x z+s y) \\
2(x y+s z) & 2\left(s^{2}+y^{2}\right)-1 & 2(y z-s x) \\
2(x z-s y) & 2(y z+s x) & 2\left(s^{2}+z^{2}\right)-1
\end{array}\right]\left[\begin{array}{l}
x_{p} \\
y_{p} \\
z_{p}
\end{array}\right] \\
& =\left[\begin{array}{ccc}
0 & 0 & 1 \\
0 & 1 & 0 \\
-1 & 0 & 0
\end{array}\right]\left[\begin{array}{l}
x_{p} \\
y_{p} \\
z_{p}
\end{array}\right] .
\end{aligned}
$$

Substituting the coordinates $(0,1,1)$ in $(7.22)$ gives

$$
\left[\begin{array}{l}
1 \\
1 \\
0
\end{array}\right]=\left[\begin{array}{ccc}
0 & 0 & 1 \\
0 & 1 & 0 \\
-1 & 0 & 0
\end{array}\right]\left[\begin{array}{l}
0 \\
1 \\
1
\end{array}\right]
$$

which is correct.

For $q_{2}$ :

$$
s=\frac{\sqrt{2}}{2}, \quad x=\frac{\sqrt{2}}{2}, \quad y=0, \quad z=0
$$

which when substituted in $(7.15)$ gives

$$
\begin{aligned}
\mathbf{p}_{2}^{\prime} & =\left[\begin{array}{ccc}
2\left(s^{2}+x^{2}\right)-1 & 2(x y-s z) & 2(x z+s y) \\
2(x y+s z) & 2\left(s^{2}+y^{2}\right)-1 & 2(y z-s x) \\
2(x z-s y) & 2(y z+s x) & 2\left(s^{2}+z^{2}\right)-1
\end{array}\right]\left[\begin{array}{l}
x_{p} \\
y_{p} \\
z_{p}
\end{array}\right] \\
& =\left[\begin{array}{ccc}
1 & 0 & 0 \\
0 & 0 & -1 \\
0 & 1 & 0
\end{array}\right]\left[\begin{array}{c}
x_{p} \\
y_{p} \\
z_{p}
\end{array}\right] .
\end{aligned}
$$

Substituting the coordinates $(0,1,1)$ in $(7.23)$ gives

$$
\left[\begin{array}{c}
0 \\
-1 \\
1
\end{array}\right]=\left[\begin{array}{ccc}
1 & 0 & 0 \\
0 & 0 & -1 \\
0 & 1 & 0
\end{array}\right]\left[\begin{array}{l}
0 \\
1 \\
1
\end{array}\right]
$$

which is also correct.

Using (7.20) with $t=0.5$ computes a mid-way position for an interpolated quaternion, with its vector at $45^{\circ}$ between the $x$ - and $y$-axes, as shown in Fig. 7.13. We already know that $\theta=60^{\circ}$, therefore $\sin \theta=\sqrt{3} / 2$ :

$$
\begin{aligned}
q & =\frac{\sin (1-t) \theta}{\sin \theta} q_{1}+\frac{\sin t \theta}{\sin \theta} q_{2} \\
& =\frac{\sin \frac{1}{2} 60^{\circ}}{\sin 60^{\circ}}\left[\frac{\sqrt{2}}{2}, \frac{\sqrt{2}}{2} \mathbf{j}\right]+\frac{\sin \frac{1}{2} 60^{\circ}}{\sin 60^{\circ}}\left[\frac{\sqrt{2}}{2}, \frac{\sqrt{2}}{2} \mathbf{i}\right]
\end{aligned}
$$

Fig. 7.13 The point $(0,1,1)$ is rotated $90^{\circ}$ about the vector $\mathbf{v}$ to $(1,0,1)$

\begin{center}
\includegraphics[max width=\textwidth]{2023_01_16_a848224efad29cd66460g-132}
\end{center}

$$
\begin{aligned}
& =\frac{1}{\sqrt{3}}\left[\frac{\sqrt{2}}{2}, \frac{\sqrt{2}}{2} \mathbf{j}\right]+\frac{1}{\sqrt{3}}\left[\frac{\sqrt{2}}{2}, \frac{\sqrt{2}}{2} \mathbf{i}\right] \\
& =\left[\frac{\sqrt{2}}{\sqrt{3}}, \frac{1}{\sqrt{6}} \mathbf{i}+\frac{1}{\sqrt{6}} \mathbf{j}\right]
\end{aligned}
$$

where

$$
s=\frac{\sqrt{2}}{\sqrt{3}}, \quad x=\frac{1}{\sqrt{6}}, \quad y=\frac{1}{\sqrt{6}}, \quad z=0
$$

which when substituted in $(7.15)$ gives

$$
\begin{aligned}
\mathbf{p}^{\prime} & =\left[\begin{array}{ccc}
2\left(s^{2}+x^{2}\right)-1 & 2(x y-s z) & 2(x z+s y) \\
2(x y+s z) & 2\left(s^{2}+y^{2}\right)-1 & 2(y z-s x) \\
2(x z-s y) & 2(y z+s x) & 2\left(s^{2}+z^{2}\right)-1
\end{array}\right]\left[\begin{array}{l}
x_{p} \\
y_{p} \\
z_{p}
\end{array}\right] \\
& =\left[\begin{array}{ccc}
\frac{2}{3} & \frac{1}{3} & \frac{2}{3} \\
\frac{1}{3} & \frac{2}{3} & -\frac{2}{3} \\
-\frac{2}{3} & \frac{2}{3} & \frac{1}{3}
\end{array}\right]\left[\begin{array}{l}
x_{p} \\
y_{p} \\
z_{p}
\end{array}\right] .
\end{aligned}
$$

Substituting the coordinates $(0,1,1)$ in $(7.24)$ gives

$$
\left[\begin{array}{l}
1 \\
0 \\
1
\end{array}\right]=\left[\begin{array}{ccc}
\frac{2}{3} & \frac{1}{3} & \frac{2}{3} \\
\frac{1}{3} & \frac{2}{3} & -\frac{2}{3} \\
-\frac{2}{3} & \frac{2}{3} & \frac{1}{3}
\end{array}\right]\left[\begin{array}{l}
0 \\
1 \\
1
\end{array}\right]
$$

which gives the point $(1,0,1)$.

One of the reasons for using a spherical interpolant is that it linearly interpolates the angle between the two unit-norm quaternions, which creates a constant-angular velocity between them. However, one of the problems with visualising quaternions is that they reside in a four-dimensional space and create a hyper-sphere with a radius equal to the quaternion's norm. With our 3D brains, this is difficult to visualise. Nevertheless, we can convince ourselves into thinking we see what is going on with a simple sketch, as shown in Fig. 7.14, where we see part of the hyper-sphere and two quaternions $q_{1}$ and $q_{2}$. In this example, the angle $\phi$ is a constant angle between two values of the interpolant $t$. The spherical interpolant also ensures that the norm Fig. 7.14 Spherical interpolation between $q_{1}$ and $q_{2}$
\includegraphics[max width=\textwidth, center]{2023_01_16_a848224efad29cd66460g-133}

Fig. 7.15 Sketch showing the actions of the interpolated quaternions of the interpolated quaternion remains constant at unity, preventing any unwanted scaling.

Figure $7.15$ provides another sketch to help visualise what is going on. For example, when $t=0$, the interpolated quaternion is $q_{1}$ which rotates the point $(0,1,1)$ to $(1,1,0)$, and when $t=1$, the interpolated quaternion is $q_{2}$ which rotates the point $(0,1,1)$ to $(0,-1,1)$. When $t=0.5$, the interpolated quaternion rotates the point $(0,1,1)$ to $(1,0,1)$ as computed above. Two other curves show what happens for $t=0.25$ and $t=0.75$.

A natural consequence of the interpolant is that the angle of rotation is $90^{\circ}$ for $t=0$ and $t=1$, but for $t=0.5$ the angle of rotation (eigenvalue) is approximately $70.5^{\circ}$. Corresponding angles arise for other values of $t$.

\section{将旋转矩阵转换为四元数}
The matrix transform equivalent to $q p q^{-1}$ is

$$
\begin{aligned}
q p q^{-1} & =\left[\begin{array}{ccc}
2\left(s^{2}+x^{2}\right)-1 & 2(x y-s z) & 2(x z+s y) \\
2(x y+s z) & 2\left(s^{2}+y^{2}\right)-1 & 2(y z-s x) \\
2(x z-s y) & 2(y z+s x) & 2\left(s^{2}+z^{2}\right)-1
\end{array}\right]\left[\begin{array}{l}
x_{p} \\
y_{p} \\
z_{p}
\end{array}\right] \\
& =\left[\begin{array}{lll}
a_{11} & a_{12} & a_{13} \\
a_{21} & a_{22} & a_{23} \\
a_{31} & a_{32} & a_{33}
\end{array}\right]\left[\begin{array}{l}
x_{p} \\
y_{p} \\
z_{p}
\end{array}\right] .
\end{aligned}
$$

Inspection of the matrix shows that by combining various elements we can isolate the terms of a quaternion $s, x, y, z$. For example, by adding the terms $a_{11}+a_{22}+a_{33}$ we obtain

$$
\begin{aligned}
a_{11}+a_{22}+a_{33} & =\left(2\left(s^{2}+x^{2}\right)-1\right)+\left(2\left(s^{2}+y^{2}\right)-1\right)+\left(2\left(s^{2}+z^{2}\right)-1\right) \\
& =6 s^{2}+2\left(x^{2}+y^{2}+z^{2}\right)-3 \\
& =4 s^{2}-1
\end{aligned}
$$

therefore,

$$
s=\pm \frac{1}{2} \sqrt{1+a_{11}+a_{22}+a_{33}} .
$$

To isolate $x, y$ and $z$ we employ

$$
\begin{aligned}
& x=\frac{1}{4 s}\left(a_{32}-a_{23}\right) \\
& y=\frac{1}{4 s}\left(a_{13}-a_{31}\right) \\
& z=\frac{1}{4 s}\left(a_{21}-a_{12}\right) .
\end{aligned}
$$

We can confirm their correctness using the matrix (7.25):

$$
\begin{aligned}
& \left[\begin{array}{lll}a_{11} & a_{12} & a_{13} \\a_{21} & a_{22} & a_{23} \\a_{31} & a_{32} & a_{33}\end{array}\right]=\left[\begin{array}{ccc}\frac{2}{3} & \frac{1}{3} & \frac{2}{3} \\\frac{1}{3} & \frac{2}{3} & -\frac{2}{3} \\-\frac{2}{3} & \frac{2}{3} & \frac{1}{3}\end{array}\right] \\
& s=\pm \frac{1}{2} \sqrt{1+a_{11}+a_{22}+a_{33}}=\pm \frac{1}{2} \sqrt{1+\frac{2}{3}+\frac{2}{3}+\frac{1}{3}}=\frac{\sqrt{2}}{\sqrt{3}} \\
& x=\frac{1}{4 s}\left(a_{32}-a_{23}\right)=\frac{\sqrt{3}}{4 \sqrt{2}}\left(\frac{2}{3}+\frac{2}{3}\right)=\frac{1}{\sqrt{6}} \\
& y=\frac{1}{4 s}\left(a_{13}-a_{31}\right)=\frac{\sqrt{3}}{4 \sqrt{2}}\left(\frac{2}{3}+\frac{2}{3}\right)=\frac{1}{\sqrt{6}} \\
& z=\frac{1}{4 s}\left(a_{21}-a_{12}\right)=\frac{\sqrt{3}}{4 \sqrt{2}}\left(\frac{1}{3}-\frac{1}{3}\right)=0
\end{aligned}
$$

which agree with the original values.

Say, for example, the value of $s$ had been close to zero, this could have made the values of $x, y, z$ unreliable. Consequently, other combinations are available:

$$
\begin{aligned}
& x=\pm \frac{1}{2} \sqrt{1+a_{11}-a_{22}-a_{33}} \\
& y=\frac{1}{4 x}\left(a_{12}+a_{21}\right) \\
& z=\frac{1}{4 x}\left(a_{13}+a_{31}\right)
\end{aligned}
$$

$$
\begin{aligned}
s & =\frac{1}{4 x}\left(a_{32}-a_{23}\right) \\
y & =\pm \frac{1}{2} \sqrt{1-a_{11}+a_{22}-a_{33}} \\
x & =\frac{1}{4 y}\left(a_{12}+a_{21}\right) \\
z & =\frac{1}{4 y}\left(a_{23}+a_{32}\right) \\
s & =\frac{1}{4 y}\left(a_{13}-a_{31}\right) \\
z & =\pm \frac{1}{2} \sqrt{1-a_{11}-a_{22}+a_{33}} \\
x & =\frac{1}{4 z}\left(a_{13}+a_{31}\right) \\
y & =\frac{1}{4 z}\left(a_{23}+a_{32}\right) \\
s & =\frac{1}{4 z}\left(a_{21}-a_{12}\right) .
\end{aligned}
$$

\section{欧拉角转换到四元数}
In Chap. 6 we discovered that the rotation transforms $\mathbf{R}_{\alpha, x}, \mathbf{R}_{\beta, y}$ and $\mathbf{R}_{\gamma, z}$ can be combined to create twelve triple combinations to represent a composite rotation. Now let's see how such a transform is represented by a quaternion.

To demonstrate the technique we must choose one of the twelve combinations, then the same technique can be used to convert other combinations. For example, let's choose the sequence $\mathbf{R}_{\gamma, z} \mathbf{R}_{\beta, y} \mathbf{R}_{\alpha, x}$ where the equivalent quaternions are

$$
\begin{aligned}
& q_{x}=\left[\cos \frac{1}{2} \alpha, \sin \frac{1}{2} \alpha \mathbf{i}\right] \\
& q_{y}=\left[\cos \frac{1}{2} \beta, \sin \frac{1}{2} \beta \mathbf{j}\right] \\
& q_{z}=\left[\cos \frac{1}{2} \gamma, \sin \frac{1}{2} \gamma \mathbf{k}\right]
\end{aligned}
$$

and

$$
q=q_{z} q_{y} q_{x}
$$

Expanding (7.26):

$$
\begin{aligned}
& q=\left[\cos \frac{1}{2} \gamma, \sin \frac{1}{2} \gamma \mathbf{k}\right]\left[\cos \frac{1}{2} \beta, \sin \frac{1}{2} \beta \mathbf{j}\right]\left[\cos \frac{1}{2} \alpha, \sin \frac{1}{2} \alpha \mathbf{i}\right] \\
& =\left[\cos \frac{1}{2} \gamma \cos \frac{1}{2} \beta,\right. \\
& \left.\cos \frac{1}{2} \gamma \sin \frac{1}{2} \beta \mathbf{j}+\cos \frac{1}{2} \beta \sin \frac{1}{2} \gamma \mathbf{k}-\sin \frac{1}{2} \gamma \sin \frac{1}{2} \beta \mathbf{i}\right]\left[\cos \frac{1}{2} \alpha, \sin \frac{1}{2} \alpha \mathbf{i}\right] \\
& =\left[\cos \frac{1}{2} \gamma \cos \frac{1}{2} \beta \cos \frac{1}{2} \alpha+\sin \frac{1}{2} \gamma \sin \frac{1}{2} \beta \sin \frac{1}{2} \alpha\right. \text {, } \\
& \cos \frac{1}{2} \gamma \cos \frac{1}{2} \beta \sin \frac{1}{2} \alpha \mathbf{i}+\cos \frac{1}{2} \alpha \cos \frac{1}{2} \gamma \sin \frac{1}{2} \beta \mathbf{j}+\cos \frac{1}{2} \alpha \cos \frac{1}{2} \beta \sin \frac{1}{2} \gamma \mathbf{k} \\
& \left.-\cos \frac{1}{2} \alpha \sin \frac{1}{2} \gamma \sin \frac{1}{2} \beta \mathbf{i}-\cos \frac{1}{2} \gamma \sin \frac{1}{2} \beta \sin \frac{1}{2} \alpha \mathbf{k}+\cos \frac{1}{2} \beta \sin \frac{1}{2} \gamma \sin \frac{1}{2} \alpha \mathbf{j}\right] \\
& =\left[\cos \frac{1}{2} \gamma \cos \frac{1}{2} \beta \cos \frac{1}{2} \alpha+\sin \frac{1}{2} \gamma \sin \frac{1}{2} \beta \sin \frac{1}{2} \alpha\right. \text {, } \\
& \left(\cos \frac{1}{2} \gamma \cos \frac{1}{2} \beta \sin \frac{1}{2} \alpha-\cos \frac{1}{2} \alpha \sin \frac{1}{2} \gamma \sin \frac{1}{2} \beta\right) \mathbf{i} \\
& \left(\cos \frac{1}{2} \alpha \cos \frac{1}{2} \gamma \sin \frac{1}{2} \beta+\cos \frac{1}{2} \beta \sin \frac{1}{2} \gamma \sin \frac{1}{2} \alpha\right) \mathbf{j} \\
& \left.\left(\cos \frac{1}{2} \alpha \cos \frac{1}{2} \beta \sin \frac{1}{2} \gamma-\cos \frac{1}{2} \gamma \sin \frac{1}{2} \beta \sin \frac{1}{2} \alpha\right) \mathbf{k}\right] \text {. }
\end{aligned}
$$

Now let's place the angles in a consistent sequence:

$$
\begin{aligned}
s & =\cos \frac{1}{2} \gamma \cos \frac{1}{2} \beta \cos \frac{1}{2} \alpha+\sin \frac{1}{2} \gamma \sin \frac{1}{2} \beta \sin \frac{1}{2} \alpha \\
x_{q} & =\cos \frac{1}{2} \gamma \cos \frac{1}{2} \beta \sin \frac{1}{2} \alpha-\sin \frac{1}{2} \gamma \sin \frac{1}{2} \beta \cos \frac{1}{2} \alpha \\
y_{q} & =\cos \frac{1}{2} \gamma \sin \frac{1}{2} \beta \cos \frac{1}{2} \alpha+\sin \frac{1}{2} \gamma \cos \frac{1}{2} \beta \sin \frac{1}{2} \alpha \\
z_{q} & =\sin \frac{1}{2} \gamma \cos \frac{1}{2} \beta \cos \frac{1}{2} \alpha-\cos \frac{1}{2} \gamma \sin \frac{1}{2} \beta \sin \frac{1}{2} \alpha
\end{aligned}
$$

where

$$
q=\left[s, x_{q} \mathbf{i}+y_{q} \mathbf{j}+z_{q} \mathbf{k}\right] .
$$

Let's test (7.27). We start with the three rotation transforms

$$
\begin{aligned}
\mathbf{R}_{\alpha, x} & =\left[\begin{array}{ccc}
1 & 0 & 0 \\
0 & \cos \alpha & -\sin \alpha \\
0 & \sin \alpha & \cos \alpha
\end{array}\right] \\
\mathbf{R}_{\beta, y} & =\left[\begin{array}{ccc}
\cos \beta & 0 & \sin \beta \\
0 & 1 & 0 \\
-\sin \beta & 0 & \cos \beta
\end{array}\right]
\end{aligned}
$$

$$
\mathbf{R}_{\gamma, z}=\left[\begin{array}{ccc}
\cos \gamma & -\sin \gamma & 0 \\
\sin \gamma & \cos \gamma & 0 \\
0 & 0 & 1
\end{array}\right]
$$

Then

$$
\begin{aligned}
& \mathbf{R}_{\gamma, z} \mathbf{R}_{\beta, y} \mathbf{R}_{\alpha, x} \\
&= {\left[\begin{array}{ccc}
\cos \gamma \cos \beta & -\sin \gamma \cos \alpha+\cos \gamma \sin \beta \sin \alpha & \sin \gamma \sin \alpha+\cos \gamma \sin \beta \cos \alpha \\
\sin \gamma \cos \beta & \cos \gamma \cos \alpha+\sin \gamma \sin \beta \sin \alpha & -\cos \gamma \sin \alpha+\sin \gamma \sin \beta \cos \alpha \\
-\sin \beta & \cos \beta \sin \alpha & \cos \beta \cos \alpha
\end{array}\right] . }
\end{aligned}
$$

Let's make $\alpha=\beta=\gamma=90^{\circ}$, then

$$
\mathbf{R}_{90^{\circ}, z} \mathbf{R}_{90^{\circ}, y} \mathbf{R}_{90^{\circ}, x}=\left[\begin{array}{ccc}
0 & 0 & 1 \\
0 & 1 & 0 \\
-1 & 0 & 0
\end{array}\right]
$$

which rotates points $90^{\circ}$ about the $y$-axis:

$$
\left[\begin{array}{l}
1 \\
1 \\
0
\end{array}\right]=\left[\begin{array}{ccc}
0 & 0 & 1 \\
0 & 1 & 0 \\
-1 & 0 & 0
\end{array}\right]\left[\begin{array}{l}
0 \\
1 \\
1
\end{array}\right]
$$

Now let's evaluate (7.27):

$$
\begin{aligned}
s & =\cos \frac{1}{2} \gamma \cos \frac{1}{2} \beta \cos \frac{1}{2} \alpha+\sin \frac{1}{2} \gamma \sin \frac{1}{2} \beta \sin \frac{1}{2} \alpha \\
& =\frac{\sqrt{2}}{2} \frac{\sqrt{2}}{2} \frac{\sqrt{2}}{2}+\frac{\sqrt{2}}{2} \frac{\sqrt{2}}{2} \frac{\sqrt{2}}{2} \\
& =\frac{\sqrt{2}}{2} \\
x_{q} & =\cos \frac{1}{2} \gamma \cos \frac{1}{2} \beta \sin \frac{1}{2} \alpha-\sin \frac{1}{2} \gamma \sin \frac{1}{2} \beta \cos \frac{1}{2} \alpha \\
& =0 \\
y_{q} & =\cos \frac{1}{2} \gamma \sin \frac{1}{2} \beta \cos \frac{1}{2} \alpha+\sin \frac{1}{2} \gamma \cos \frac{1}{2} \beta \sin \frac{1}{2} \alpha \\
& =\frac{\sqrt{2}}{2} \frac{\sqrt{2}}{2} \frac{\sqrt{2}}{2}+\frac{\sqrt{2}}{2} \frac{\sqrt{2}}{2} \frac{\sqrt{2}}{2} \\
& =\frac{\sqrt{2}}{2} \\
z_{q} & =\sin \frac{1}{2} \gamma \cos \frac{1}{2} \beta \cos \frac{1}{2} \alpha-\cos \frac{1}{2} \gamma \sin \frac{1}{2} \beta \sin \frac{1}{2} \alpha \\
& =0
\end{aligned}
$$

and

$$
q=\left[\frac{\sqrt{2}}{2}, \frac{\sqrt{2}}{2} \mathbf{j}\right]
$$

which is a quaternion that also rotates points $90^{\circ}$ about the $y$-axis.

\section{总结}
This chapter has been the focal point of the book where unit-norm quaternions have been used to rotate a vector about a quaternion's vector. It would have been useful if this could have been achieved by the simple product $q p$, like complex numbers. But as we saw, this only works when the quaternion is orthogonal to the vector. The product $q p q^{-1}$ —discovered by Hamilton and Cayley-works for all orientations between a quaternion and a vector. It is also relatively easy to compute. We also saw that the product can be represented as a matrix, which can be integrated with other matrices.

Perhaps one of the most interesting features of quaternions that has emerged in this chapter, is that their imaginary qualities are not required in any calculations, because they are embedded within the algebra.

The spherical interpolant provides a clever way to dynamically change a quaternion's axis and angle of rotation, but can be difficult to visualise as an animated sequence without access to a real-time display system.

The reverse product $q^{-1} p q$ reverses the angle of rotation, and is equivalent to changing the sign of the rotation angle in $q p q^{-1}$. Consequently, it can be used to rotate a frame of reference in the same direction as $q p q^{-1}$.

\subsection{操作符总结}
\subsubsection*{Rotating a point about a vector}
$$
\begin{aligned}
q & =[s, \mathbf{v}] \\
s^{2}+|\mathbf{v}|^{2} & =1 \\
p & =[0, \mathbf{p}] \\
q p q^{-1} & =\left[0,2(\mathbf{v} \cdot \mathbf{p}) \mathbf{v}+\left(2 s^{2}-1\right) \mathbf{p}+2 s \mathbf{v} \times \mathbf{p}\right] \\
q & =\left[\cos \frac{1}{2} \theta, \sin \frac{1}{2} \theta \hat{\mathbf{v}}\right] \\
p & =[0, \mathbf{p}] \\
q p q^{-1} & =[0,(1-\cos \theta)(\hat{\mathbf{v}} \cdot \mathbf{p}) \hat{\mathbf{v}}+\cos \theta \mathbf{p}+\sin \theta \hat{\mathbf{v}} \times \mathbf{p}]
\end{aligned}
$$

\subsubsection*{Rotating a frame about a vector}
$$
q^{-1} p q=[0,(1-\cos \theta)(\hat{\mathbf{v}} \cdot \mathbf{p}) \hat{\mathbf{v}}+\cos \theta \mathbf{p}-\sin \theta \hat{\mathbf{v}} \times \mathbf{p}]
$$

\subsubsection*{Matrix for rotating a point about a vector}
$$
\mathbf{p}^{\prime}=\left[\begin{array}{ccc}
1-2\left(y^{2}+z^{2}\right) & 2(x y-s z) & 2(x z+s y) \\
2(x y+s z) & 1-2\left(x^{2}+z^{2}\right) & 2(y z-s x) \\
2(x z-s y) & 2(y z+s x) & 1-2\left(x^{2}+y^{2}\right)
\end{array}\right]\left[\begin{array}{c}
x_{p} \\
y_{p} \\
z_{p}
\end{array}\right]
$$

\subsubsection*{Matrix for rotating a frame about a vector}
$$
\mathbf{p}^{\prime}=\left[\begin{array}{ccc}
1-2\left(y^{2}+z^{2}\right) & 2(x y+s z) & 2(x z-s y) \\
2(x y-s z) & 1-2\left(x^{2}+z^{2}\right) & 2(y z+s x) \\
2(x z+s y) & 2(y z-s x) & 1-2\left(x^{2}+y^{2}\right)
\end{array}\right]\left[\begin{array}{l}
x_{p} \\
y_{p} \\
z_{p}
\end{array}\right]
$$

\subsubsection*{Matrix for a quaternion product}
$$
\begin{aligned}
& q_{1} q_{2}=\mathbf{L}\left(q_{1}\right) q_{2}= {\left[\begin{array}{cccc}
s_{1} & -x_{1} & -y_{1} & -z_{1} \\
x_{1} & s_{1} & -z_{1} & y_{1} \\
y_{1} & z_{1} & s_{1} & -x_{1} \\
z_{1} & -y_{1} & x_{1} & s_{1}
\end{array}\right]\left[\begin{array}{l}
s_{2} \\
x_{2} \\
y_{2} \\
z_{2}
\end{array}\right] } \\
& q_{1} q_{2}=\mathbf{R}\left(q_{2}\right) q_{1}=\left[\begin{array}{cccc}
s_{2} & -x_{2} & -y_{2} & -z_{2} \\
x_{2} & s_{2} & z_{2} & -y_{2} \\
y_{2} & -z_{2} & s_{2} & x_{2} \\
z_{2} & y_{2} & -x_{2} & s_{2}
\end{array}\right]\left[\begin{array}{l}
s_{1} \\
x_{1} \\
y_{1} \\
z_{1}
\end{array}\right]
\end{aligned}
$$

\subsubsection*{Interpolating two quaternions}
$$
q=\frac{\sin (1-t) \theta}{\sin \theta} q_{1}+\frac{\sin t \theta}{\sin \theta} q_{2}
$$

where

$$
\begin{aligned}
\cos \theta & =\frac{q_{1} \cdot q_{2}}{\left|q_{1}\right|\left|q_{2}\right|} \\
& =\frac{s_{1} s_{2}+x_{1} x_{2}+y_{1} y_{2}+z_{1} z_{2}}{\left|q_{1}\right|\left|q_{2}\right|}
\end{aligned}
$$

\subsubsection*{Quaternion from a rotation matrix}
$$
\begin{aligned}
& s=\pm \frac{1}{2} \sqrt{1+a_{11}+a_{22}+a_{33}} \\
& x=\frac{1}{4 s}\left(a_{32}-a_{23}\right) \\
& y=\frac{1}{4 s}\left(a_{13}-a_{31}\right) \\
& z=\frac{1}{4 s}\left(a_{21}-a_{12}\right) \\
& x=\pm \frac{1}{2} \sqrt{1+a_{11}-a_{22}-a_{33}} \\
& y=\frac{1}{4 x}\left(a_{12}+a_{21}\right) \\
& z=\frac{1}{4 x}\left(a_{13}+a_{31}\right) \\
& s=\frac{1}{4 x}\left(a_{32}-a_{23}\right) \\
& y=\pm \frac{1}{2} \sqrt{1-a_{11}+a_{22}-a_{33}}
\end{aligned}
$$

$$
\begin{aligned}
x & =\frac{1}{4 y}\left(a_{12}+a_{21}\right) \\
z & =\frac{1}{4 y}\left(a_{23}+a_{32}\right) \\
s & =\frac{1}{4 y}\left(a_{13}-a_{31}\right) \\
z & =\pm \frac{1}{2} \sqrt{1-a_{11}-a_{22}+a_{33}} \\
x & =\frac{1}{4 z}\left(a_{13}+a_{31}\right) \\
y & =\frac{1}{4 z}\left(a_{23}+a_{32}\right) \\
s & =\frac{1}{4 z}\left(a_{21}-a_{12}\right)
\end{aligned}
$$

\subsubsection*{Eigenvector and eigenvalue}
$$
\begin{aligned}
x_{v} & =\left(a_{22}-1\right)\left(a_{33}-1\right)-a_{23} a_{32} \\
y_{v} & =\left(a_{33}-1\right)\left(a_{11}-1\right)-a_{31} a_{13} \\
z_{v} & =\left(a_{11}-1\right)\left(a_{22}-1\right)-a_{12} a_{21} \\
\cos \theta & =\frac{1}{2}\left(\operatorname{Tr}\left(q p q^{-1}\right)-1\right)
\end{aligned}
$$

\subsubsection*{Euler angles to quaternion}
Using the transform $\mathbf{R}_{\gamma, z} \mathbf{R}_{\beta, y} \mathbf{R}_{\alpha, x}$ :

$$
\begin{aligned}
s & =\cos \frac{1}{2} \gamma \cos \frac{1}{2} \beta \cos \frac{1}{2} \alpha+\sin \frac{1}{2} \gamma \sin \frac{1}{2} \beta \sin \frac{1}{2} \alpha \\
x_{q} & =\cos \frac{1}{2} \gamma \cos \frac{1}{2} \beta \sin \frac{1}{2} \alpha-\sin \frac{1}{2} \gamma \sin \frac{1}{2} \beta \cos \frac{1}{2} \alpha \\
y_{q} & =\cos \frac{1}{2} \gamma \sin \frac{1}{2} \beta \cos \frac{1}{2} \alpha+\sin \frac{1}{2} \gamma \cos \frac{1}{2} \beta \sin \frac{1}{2} \alpha \\
z_{q} & =\sin \frac{1}{2} \gamma \cos \frac{1}{2} \beta \cos \frac{1}{2} \alpha-\cos \frac{1}{2} \gamma \sin \frac{1}{2} \beta \sin \frac{1}{2} \alpha
\end{aligned}
$$

where

$$
q=\left[s, x_{q} \mathbf{i}+y_{q} \mathbf{j}+z_{q} \mathbf{k}\right]
$$

\section{样例}
Here are some further worked examples that employ the ideas described above.

Example 1 Use $q p$ to rotate $p=[0, \mathbf{j}] 90^{\circ}$ about the $x$-axis.

For this to work $q$ must be orthogonal to $p$ :

$$
\begin{aligned}
q & =[\cos \theta, \sin \theta \mathbf{i}] \\
& =[0, \mathbf{i}]
\end{aligned}
$$

and

$$
\begin{aligned}
p^{\prime} & =q p \\
& =[0, \mathbf{i}][0, \mathbf{j}] \\
& =[0, \mathbf{k}] .
\end{aligned}
$$

Example 2 Use $q p q^{-1}$ to rotate $p=[0, \mathbf{j}] 90^{\circ}$ about the $x$-axis.

For this to work:

$$
\begin{aligned}
q & =\left[\cos \frac{1}{2} \theta, \sin \frac{1}{2} \theta \mathbf{i}\right] \\
& =\left[\frac{\sqrt{2}}{2}, \frac{\sqrt{2}}{2} \mathbf{i}\right]
\end{aligned}
$$

and

$$
\begin{aligned}
p^{\prime} & =q p q^{-1} \\
& =\left[\frac{\sqrt{2}}{2}, \frac{\sqrt{2}}{2} \mathbf{i}\right][0, \mathbf{j}]\left[\frac{\sqrt{2}}{2},-\frac{\sqrt{2}}{2} \mathbf{i}\right] \\
& =\left[0, \frac{\sqrt{2}}{2} \mathbf{j}+\frac{\sqrt{2}}{2} \mathbf{k}\right]\left[\frac{\sqrt{2}}{2},-\frac{\sqrt{2}}{2} \mathbf{i}\right] \\
& =\left[0, \frac{\sqrt{2}}{2}\left(\frac{\sqrt{2}}{2} \mathbf{j}+\frac{\sqrt{2}}{2} \mathbf{k}\right)+\frac{1}{2} \mathbf{j}+\frac{1}{2} \mathbf{k}\right] \\
& =\left[0, \frac{1}{2} \mathbf{j}+\frac{1}{2} \mathbf{k}-\frac{1}{2} \mathbf{j}+\frac{1}{2} \mathbf{k}\right] \\
& =[0, \mathbf{k}]
\end{aligned}
$$

which agrees with the answer for Example 1.

Example 3 Evaluate the triple $q p q^{-1}$ for $p=[0, \mathbf{p}]$ and $q=\left[\cos \frac{1}{2} \theta, \sin \frac{1}{2} \theta \mathbf{v}\right]$, where $\theta=360^{\circ}$.

$$
\begin{aligned}
q & =[-1, \mathbf{0}] \\
q p q^{-1} & =[-1, \mathbf{0}][0, \mathbf{p}][-1, \mathbf{0}] \\
& =[0,-\mathbf{p}][-1, \mathbf{0}] \\
& =[0, \mathbf{p}]
\end{aligned}
$$

which confirms that the vector remains unmoved, as expected.

Example 4 Compute the matrix (7.15) for $q=\left[\frac{1}{2}, \frac{\sqrt{3}}{2} \mathbf{k}\right]$, and find its eigenvector and eigenvalue. From $q$

$$
\begin{aligned}
& s=\frac{1}{2}, \quad x=0, \quad y=0, \quad z=\frac{\sqrt{3}}{2} \\
& \mathbf{p}^{\prime}=\left[\begin{array}{ccc}2\left(s^{2}+x^{2}\right)-1 & 2(x y-s z) & 2(x z+s y) \\2(x y+s z) & 2\left(s^{2}+y^{2}\right)-1 & 2(y z-s x) \\2(x z-s y) & 2(y z+s x) & 2\left(s^{2}+z^{2}\right)-1\end{array}\right]\left[\begin{array}{l}x_{p} \\y_{p} \\z_{p}\end{array}\right] \\
& =\left[\begin{array}{ccc}-\frac{1}{2} & -\frac{\sqrt{3}}{2} & 0 \\\frac{\sqrt{3}}{2} & -\frac{1}{2} & 0 \\0 & 0 & 1\end{array}\right]\left[\begin{array}{l}x_{p} \\y_{p} \\z_{p}\end{array}\right] \text {. }
\end{aligned}
$$

If we plug in the point $(1,0,0)$ it is rotated about the $z$-axis by $120^{\circ}$ :

$$
\left[\begin{array}{c}
-\frac{1}{2} \\
\frac{\sqrt{3}}{2} \\
1
\end{array}\right]=\left[\begin{array}{ccc}
-\frac{1}{2} & -\frac{\sqrt{3}}{2} & 0 \\
\frac{\sqrt{3}}{2} & -\frac{1}{2} & 0 \\
0 & 0 & 1
\end{array}\right]\left[\begin{array}{l}
1 \\
0 \\
0
\end{array}\right]
$$

Using

$$
\begin{aligned}
\cos \theta & =\frac{1}{2}\left(\operatorname{Tr}\left(q p q^{-1}\right)-1\right) \\
& =\frac{1}{2}(0-1) \\
\theta & =120^{\circ}
\end{aligned}
$$

Using

$$
\begin{aligned}
x_{v} & =\left(a_{22}-1\right)\left(a_{33}-1\right)-a_{23} a_{32} \\
& =\left(-\frac{3}{2}\right)(0)-0 \\
& =0 \\
y_{v} & =\left(a_{33}-1\right)\left(a_{11}-1\right)-a_{31} a_{13} \\
& =(0)\left(-\frac{3}{2}\right)-0 \\
& =0 \\
z_{v} & =\left(a_{11}-1\right)\left(a_{22}-1\right)-a_{12} a_{21} \\
& =\left(-\frac{3}{2}\right)\left(-\frac{3}{2}\right)+\frac{\sqrt{3}}{2} \frac{\sqrt{3}}{2} \\
& =3
\end{aligned}
$$

which makes the eigenvector $3 \mathbf{k}$ and the eigenvalue $120^{\circ}$. Example 5 Find the half-way quaternion between $q_{1}=\left[\cos \frac{1}{2} \alpha, \sin \frac{1}{2} \alpha \mathbf{k}\right]$ and $q_{2}=$ $\left[\cos \frac{1}{2} \alpha, \sin \frac{1}{2} \alpha \mathbf{i}\right]$ when $\alpha=90^{\circ}$. Show that it is a unit-norm quaternion, and find its angle of rotation.

The angle between $q_{1}$ and $q_{2}$ is $\theta$ where

$$
\begin{aligned}
\cos \theta & =\frac{s_{1} s_{2}+x_{1} x_{2}+y_{1} y_{2}+z_{1} z_{2}}{\left|q_{1}\right|\left|q_{2}\right|} \\
& =\cos ^{2} \frac{1}{2} \alpha \\
& =0.5 \\
\theta & =60^{\circ} .
\end{aligned}
$$

Using

$$
\begin{aligned}
q & =\frac{\sin (1-t) \theta}{\sin \theta} q_{1}+\frac{\sin t \theta}{\sin \theta} q_{2} \\
& =\frac{\sin 30^{\circ}}{\sin 60^{\circ}}\left[\cos 45^{\circ}, \sin 45^{\circ} \mathbf{k}\right]+\frac{\sin 30^{\circ}}{\sin 60^{\circ}}\left[\cos 45^{\circ}, \sin 45^{\circ} \mathbf{i}\right] \\
& =\frac{1}{\sqrt{3}}\left[\frac{\sqrt{2}}{2}, \frac{\sqrt{2}}{2} \mathbf{k}\right]+\frac{1}{\sqrt{3}}\left[\frac{\sqrt{2}}{2}, \frac{\sqrt{2}}{2} \mathbf{i}\right] \\
& =\left[\frac{\sqrt{2}}{\sqrt{3}}, \frac{\sqrt{2}}{2 \sqrt{3}} \mathbf{i}+\frac{\sqrt{2}}{2 \sqrt{3}} \mathbf{k}\right] \\
& =\left[\frac{2}{\sqrt{6}}, \frac{1}{\sqrt{6}} \mathbf{i}+\frac{1}{\sqrt{6}} \mathbf{k}\right]
\end{aligned}
$$

The norm of $q$ is

$$
\begin{aligned}
|q| & =\left(\frac{2}{\sqrt{6}}\right)^{2}+\left(\frac{1}{\sqrt{6}}\right)^{2}+\left(\frac{1}{\sqrt{6}}\right)^{2} \\
& =\frac{2}{3}+\frac{1}{6}+\frac{1}{6} \\
& =1
\end{aligned}
$$

Therefore, $\cos \frac{1}{2} \alpha=\frac{\sqrt{2}}{\sqrt{3}}$ and $\sin \frac{1}{2} \alpha=\frac{1}{\sqrt{3}}$, and $\alpha \approx 70.5^{\circ}$.

Example 6 Convert the given matrix into a quaternion and identify its function.

$$
\mathbf{M}=\left[\begin{array}{ccc}
0 & 0 & 1 \\
0 & 1 & 0 \\
-1 & 0 & 0
\end{array}\right]
$$

therefore,

$$
\begin{aligned}
s & =\frac{1}{2} \sqrt{1+a_{11}+a_{22}+a_{33}} \\
& =\frac{1}{2} \sqrt{1+0+1+0}=\frac{\sqrt{2}}{2}
\end{aligned}
$$

$$
\begin{aligned}
x & =\frac{1}{4 s}\left(a_{32}-a_{23}\right) \\
& =\frac{\sqrt{2}}{4}(0+0)=0 \\
y & =\frac{1}{4 s}\left(a_{13}-a_{31}\right) \\
& =\frac{\sqrt{2}}{4}(1+1)=\frac{\sqrt{2}}{2} \\
z & =\frac{1}{4 s}\left(a_{21}-a_{12}\right) \\
& =\frac{\sqrt{2}}{4}(0+0)=0
\end{aligned}
$$

which is the quaternion $\left[\frac{\sqrt{2}}{2}, \frac{\sqrt{2}}{2} \mathbf{j}\right]$ which is a rotation of $90^{\circ}$ about the $y$-axis.
